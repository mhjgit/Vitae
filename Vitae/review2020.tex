 \documentclass[prc,amsart,english,superscriptaddress,showpacs,floatfix]{revtex4}
 \usepackage[T1]{fontenc}       % DC-fonts
 \begin{document}


\section*{Contribution from Morten Hjorth-Jensen (May 2019-June 2020)}

\subsection*{Regular Articles in Journals with a Referee System only, after April 30 2019}
\begin{enumerate}
\item Diego A. Torres {\em et al}, {\em Study of medium-spin states of neutron-rich $^{87, 89,91}$Rb isotopes}, European Physics Journal {\bf A55}, 2019, DOI:10.1140/epja/i2019-12839-6
\item  Calvin W. Johnson, Kristina D. Launey, Naftali Auerbach, Sonia Bacca, Bruce R. Barrett, Carl Brune, Mark A. Caprio, Pierre Descouvemont, W. H. Dickhoff, Charlotte Elster, Patrick J. Fasano, Kevin Fossez, Heiko Hergert, Morten Hjorth-Jensen, Linda Hlophe, Baishan Hu, Rodolfo M. Id Betan, Andrea Idini, Sebastian König, Konstantinos Kravvaris, Dean Lee, Jin Lei, Pieter Maris, Alexis Mercenne, Kosho Minomo , {\em et al.}, {\em From bound states to the continuum}, Journal of Physics G, 2020, in press, arXiv:https://arxiv.org/abs/1912.00451 
\item  Robert Solli, Daniel Bazin, Michelle Kuchera, Ryan Strauss, and Morten Hjorth-Jensen, {\em Unsupervised Learning  for Identifying  Events in Active Target  Experiments}, submitted to Nuclear Instruments and Methods in Physics Research Section A: Accelerators, Spectrometers, Detectors and Associated Equipment, 2020
\item Teweldebrhan, A. T. and Burkhart, J. F. and Schuler, T. V. and Hjorth-Jensen, M., {\em Coupled machine learning and the limits of acceptability approach applied in parameter identification for a distributed hydrological model}, Hydrology and Earth System Sciences Discussions, 2020, in press, DOI:10.5194/hess-2019-464
\item  John M. Aiken, Riccardo De Bin, Morten Hjorth-Jensen, Marcos D. Caballero, {\em Predicting time to graduation at a large enrollment American university}, submitted to PLOS ONE, 2020, arXiv:https://arxiv.org/abs/2005.05104
\item Sebastian G. Winther-Larsen, H\aa kon E, Kristiansen, \O yvind Sigmundson Sch\o yen, and Morten Hjorth-Jensen, {\em Time Evolution of Quantum Dot systems}, in preparation for Physical Review X Quantum, 2020.
\end{enumerate}

\subsection*{Books}
\begin{enumerate}
\item Morten Hjorth-Jensen, \emph{Computational Physics, an introduction}, to be published by IOP in 2020. Approx 500 pages

\item Morten Hjorth-Jensen, \emph{Computational Physics, an advanced course}, to be published by IOP in 2021. Approx 400 pages

\end{enumerate}



 \subsection*{Talks at workshops, conferences and institute seminars and organization of meetings}
\begin{enumerate}
\item Lecture on Nuclear Physics at the NS3 school, FRIB, Michigan State University, May 15, 2019. Main organizer Artemis Spyrou. 
\item Main organizer and lecturer of the {\em FRIB TA Summer School, Machine Learning Applied to Nuclear Physics}. Other lecturers Matthew Hirn (MSU), Michelle Kuchera (Davidson College) and Raghuram Ramanujan  (Davidson College). More than 100 participants. Duration: May 20-23, 2019 at FRIB/MSU see \url{https://research.msu.edu/frib-theory-alliance-hosts-summer-school-on-machine-learning/}. In total I gave five one-hour lectures.
\item Online lectures on {\em Machine Learning weeks at MSU-FRIB/NSCL, May 2020}. I lectured to undergraduate, graduate and post-docs at FRIB/MSU from May 18 till May 29 on Machine Learning applied to Nuclear Physics. Two lectures per day and one hour of hands-on sessions. On average between 25-30 particpants per day. All material is available at \url{https://github.com/mhjensen/MachineLearningMSU-FRIB2020}. In total I gave 20 one-hour lectures.
\item Nuclear Talent course on {\em Machine Learning and Data Analysis for Nuclear Physics, a Nuclear TALENT Course at the ECT*, Trento, Italy, June 22 to July 3 2020.}. I was the main organizer and teacher. Other teachers where Daniel Bazin and Sean Liddick FRIB/MSU and   Michelle Kuchera and Raghuram Ramanujan  (Davidson College). The course run over two weeks with two hours of lectures every day and one hour of hands-on sessions. We had 157 registered participants, coming from all continents (except for Antarctica!). The course was a fully online course. All material, with videos of lectures and more is at \url{https://github.com/NuclearTalent/MachineLearningECT}. In total I gave eight one-hour lectures.
\item {\em Hackathon on Computing in Science Education}, June 3-7, 2019. Intensive workshop on Computing in Physics Education at Michigan State University. Organized together with Danny Caballero, MSU.
\item Integrating a Computational Perspective in Physics (and Science) Courses, October 23, 2019. Ole R\o mer Colloquium, Department of Physics and Astronomy, University of \AA rhus, Denmark  \url{https://phys.au.dk/en/news/item/artikel/ole-roemer-colloquium-morten-hjort-jensen-tba/}
\item  Machine Learning and Quantum Mechanics for Many Interacting Particles, UiO, March 3, 2020  \url{https://www.mn.uio.no/math/english/research/groups/statistics-data-science/events/seminars/hjorth-jensen.html}
\item Tuesday, October 28, 2019 Machine Learning and the Physical Sciences \url{https://www.meetup.com/Under-the-hood-Explaining-what-goes-on-inside-DNN-AI/events/263780932/}. 
\item {\em Workshop on Time-Dependent Many-body methods}, organizer of mini-workshop at the University of Oslo, November 11, 2019.
\item Online teaching of an intensive Master of Science course on Machine Learning applied to Nuclear Physics, GANIL and University of Basse-Normandie, Caen, France, January 21-31, 2020. I taught an intensive course for the Erasmus+ Master of Science program on Nuclear Physics, see \url{http://www.emm-nucphys.eu/}. In total there were 22 participants. I gave two online lectures every day.  In total 20 one hour lectures. 
\end{enumerate}

\subsection*{Present Graduate Students}
\begin{enumerate}
\item Benjamin Hall, Michigan State University, started summer 2018. Scott Bogner is co-supervising  since we share a common research grant.
\item Jane Kim, Michigan State University, started summer 2018. Scott Bogner is co-supervising  since we share a common research grant.
\item Julie Butler, Michigan State University, started summer 2018. Scott Bogner is co-supervising  since we share a common research grant.
\item Danny Jammoa, Michigan State University, started 2020, co-supervisor with Dean Lee. Main supervisor Betty Tsang.
\item \O yvind Sigmundsson Sch\o yen, University of Oslo, started October 2019.
\item John Mark Aiken, University of Oslo, started August 2017, defends thesis September 2020. Co-supervisor with Danny Caballero, MSU
\item Stian Bilek, University of Oslo, starts September 1 2020.
\item Zhen Li, University of Oslo, starts September 1 2020.  
\end{enumerate}
In addition I supervise and/or co-supervise ten Master of Science students at the University of Oslo
\subsection*{Students who graduated in 2019-2020}
\begin{enumerate}
\item Justin G. Lietz, PhD Michigan State University, June 2019, now post-doctoral fellow at Oak Ridge National Laboratory. Scott Bogner was co-supervising  since we share a common research grant.
\item Thirteen Master of Science students at the University of Oslo finalized their master theses during the above period. I was the main supervisor for nine of these students.
\end{enumerate}  

\subsection*{REU Students in 2019-2020}
\begin{enumerate}
\item REU student Debora Mroczek, University of Texas A\&M, Houston. Summer 2019. A talk at the APS meeting April 18-21, 2020 was presented, see \url{http://meetings.aps.org/Meeting/APR20/Session/D21.37}, {\em  Equation of State for QCD with a Critical Point: Imposing Thermodynamic Stability Using Neural Networks}, D Mroczek, M Hjorth-Jensen, C Ratti, P Parotto, J Noronha-Hostler. Now graduate student at  University of Illinois Urbana-Champaign.
Bulletin of the American Physical Society 
\item Kate A. Roberts (Kalamazoo), and Lexie Weghorn (Wisconsin), summer 2020. Work on Machine Learning applied to the analysis of experimental data.
\item Jacob Crawford and   Michael Holmquist, undergraduate students from MSU. Work on many-body theory with me and Scott Bogner during summer 2020.
\end{enumerate}  


\subsection*{Research grants and applications in 2019-2020}
\begin{enumerate}
\item {\em Quantum Many-Body Theories and Methods for Nuclear Physics}, National Science Foundation, PI Scott Bogner, I am a co-PI. This a continuation of our previous grant from the period 2017-2020. Totalt amount 600kUSD over three years.
\item  {\em From Quarks to Stars; A Quantum Computing Approach to the Nuclear Many-Body Problem}, Department of Energy, Nuclear Physics Office under the DOE Quantum Horizons: QIS Research and Innovation for Nuclear Science program. PI Morten Hjorth-Jensen, co-PIs: Alexei Bazavov (MSU), Scott Bogner, Heiko Hergert, Dean Lee, Andrea Shindler, Huey-Wen Lin (MSU), Matthew Hirn (MSU) and Patrick Coles (LANL). Total amount 750kUSD over three years, start September 2020. This is a broad collaboration and new initiative on Quantum Computing applied to the nuclear many-body problem.
\item {\em International Partnership on Computing in Science Education}, Research Council of Norway. This is a collaboration on Science education between MSU, Oregan State University, University of Colorado at Boulder and the University of Oslo. PI: Morten Hjorth-Jensen. Total amount 4.5M Norwegian kroner, roughly 500kUSD for the period 2019-2021.
\item {\em QLCI-CI: Institute for Quantum Computing and Control (IQC2) at MSU}. Center of excellence application to the National Science Foundation, total possible award 25MUSD over five years. PIs: A.K. Wilson, A.J. Christlieb, M. Dantus, M. Dykman and J. Pollanen, Michigan State University. I was a co-PI together with several people at our lab. We did not succeed in the second round.
\item {\em AI Institute: Transdisciplinary Institute for Physics-Informed Machine Learning}. Center of excellence application to the National Science Foundation on artificial intelligence, total possible award 25MUSD over five years. PIs: Brian O'Shea, Matthew Hirn and Michael Murillo, Michigan State University. I was a co-PI together with several people at our lab. We did not proceed to the first round.  
\item We have recently also submitted a large center of excellence application on Quantum Computing to DoE. Angela Wilson, Andrew Christlieb and  and Johannes Pollanen at MSU are the lead scientists again. This application involves several universities and National Labs. 
\end{enumerate}
  \subsection*{Service to the community}
\begin{itemize}

\item Scientific Board member of the European Center for Theoretical Physics and Related Areas, Trento, Italy (2017-present)
  
\item Editorial Board member of European Physical Journal Special Topics (2010-present)

\item Editorial Board member of Springer's Lecture Notes  in Physics, LNP,  (2010-present)

\item Editorial Board member of Springer's Undergraduate Lecture Notes in Physics, UNLP, (2014-present)

\item Editorial Board member of Springer's Undergraduate Texts in Physics, UTP,  (2014-present)

\item Editorial Board member of Springer's Graduate Texts in Physics, GTP,  (2017-present)

\item Editorial Board member of Springer's Unittexts in Physics (2017-present)


\item {Initiated and led the Nuclear Talent initiative from 2010 till 2015, board member till fall 2019}

\item I initiated and lead the new {Master of Science program on Computational Science at the University of Oslo}. This is a new and multi-disciplinary program across several disciplines at the College of Natural Science of the University of Oslo. It includes now six departments at the University of Oslo.
\end{itemize}

\subsection*{Courses I teach}
During the last year I have been responsible for four courses, one at MSU (spring) and three at the University of Oslo (Fall semester and Spring semester)
\begin{itemize}
\item {PHY321 Classical Mechanics}, undergraduate level, Spring semester (MSU), 61 students Spring 2020 

\item {FYS3150/4150 Computational Physics I}, Fall semester, senior undergraduate level (Oslo), 100 students Fall 2019. 

\item{FYS-STK3155/4155 Applied Data Analysis and Machine Learning}, Fall semester, graduate course (Oslo). 130 students fall 2019.
  
\item {FYS4411 Computational Physics II: Quantum mechanical systems}, graduate level, Spring semester (Oslo). Online teaching only.  14 students spring 2020.

This spring semester, due to Covid-19, required from all of us extra
efforts when we transited to online teaching. Since I had already
taught many classes remotely during the last decade, I felt the transition gave me the
possibility to explore alternative ways of organizing the lectures. I
have always preferred a project oriented way of teaching. With PHY 321
being fully online after spring break, this allowed me to transform
the final exam and the midterm exam to two one-week long
projects. With the introduction of numerical exercises as well (an a
weekly basis) plus numerical projects, this allowed the students to
explore mechanics from a more research based point of view. Although
all these efforts took a heavy toll timewise, I feel, seen the student
evaluation on SIRS, that the efforts were much appreciated. All in
all, although I have spent a lot of time on educational activities
both for the students at our lab and on regular courses, I feel this
to be an important contribution to our community. The intensive
courses on Machine Learning for Nuclear Physics have been extremely
rewarding to teach. The Nuclear Talent course we taught recently had
157 registered participants.  These courses have also opened up many interesting research avenues on Machine Learning.


\end{itemize}

\subsection*{Personal Summary}

I started in January 2012 at Michigan State University. I have a
shared position between the University of Oslo and Michigan State
University.

My main activity is on studies of nuclear physics systems, with an
emphasis on many-body methods for nuclear structure studies. In
particular, I wish to understand the stability of nuclear matter and
nuclei from first principle methods. This matches perfectly the
scientific mission of the NSCL and FRIB at MSU.

During the last three years I have also started new activities on Machine Learning applied to Nuclear Physics (experiment and theory). This has lead to several collaborations, among these
\begin{enumerate}
\item Deep Learning methods applied to regression and classification of $\beta$-decay experiments with Sean Liddick and collaborators.
\item Deep Learning methods applied to classification of active target experiments with Daniel Bazin and collaborators (see paper submitted to NIM).
\item Deep learning methods and studies of the Equation of state for dense matter. Collaboration with Betty Tsang and Dean Lee
\item On the theory started we submitted recently a large center application to the NSF. Unfortunately we did not succeed. However, we have started recently a large collaboration among several theorists (Bogner, Hergert, Lee, Nazarewicz and myself) on theoretical developments of deep learning methods appled to the nuclear many-body problem.
\end{enumerate}.

Moreover, also during the last three years, we have started a larger
collaboration at the lab on studies of Quantum Computing algorithms
applied to the nuclear many-body problem. This collaboration involves
several colleagues at the lab, Bogner, Hergert, Lee and Shindler, plus
other colleagues at MSU. We received recently funding for this from
the DoE. We have also recruited several excellent students. The
nuclear many-body problem is perhaps one of the most difficult
quantum-mechanical problems studied with powerful
supercomputers. Quantum computing has emerged as a new computational
paradigm that could circumvent the challenges and limitations
encountered on classical computers. In our new project we  propose to create a bridge
between quantum computing as it exists today and unsolved nuclear
physics problems by developing new quantum algorithms, exploring
applications of quantum error stabilzation techniques, machine
learning and quantum compiling for realizing efficient short-depth
quantum circuits. This award also focuses on creative adaptations of
quantum simulators to many-body problems in nuclear physics. 

Our studies here have also created new collaborations between us and
colleagues at other departments at MSU.  Amongst these are Johannes
Pollanen (experimental condensed matter physics), Huey-Wen Lin
(particle physics) and several colleagues at the newly established
department of Computational Mathematics, Science and Engineering.


I have also a strong educational commitment, reflected in the
initiation and partecipation in a project at the University of Oslo
called 'Computing in Science Education'. This project has changed
totally the way we teach science, with computations being introduced
at first semester of study. Many of the seminars I have given at
various US institutions deal with the integration of a computational
approach to the basic science courses. Locally at MSU I am involved in
similar activities in close collaboration with Danny Caballero and his research team.
We share a common research grant on educational activities.
I have continuous discussions with
several colleagues at the FRIB/NSCL, the Department of Physics and
Astronomy and the new Department of Computational Mathematics, Science
and Engineering on computational issues and education in computational
aspects. Much inspired by the developments at MSU, I have initiated
and chair a Master of Science program in Computational Science at the
University of Oslo, Norway. I spend the period July-December in
Norway.


Finally, my long term goal is to be able to contribute to build up a
strong activity on the nuclear many-body problem at MSU, an activity
which will match the experimental program at the FRIB/NSCL. With
Scott Bogner and Heiko Hergert we have a group of several excellent graduate students. It is a great pleasure to be able to be here and contribute to the development of our field in close contact with so many excellent researchers.


 \end{document}

