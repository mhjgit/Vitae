\chapter{Research program}

This chapter starts with a list of principal collaborators
involved in the different projects discussed in sections 2.2
and 2.3. The nuclear physics research programme, which represents
my main scientific activity, past, present and future, 
is detailed in section 2.2.

A new project on the emerging field Quantum Computing and
Quantum information theory is described in section 2.3.
This project has just been started in collaboration
with Profs.~Yuri Galperin, Jon Magne Leinaas and Leif Veseth
at the University of Oslo.  

\section{International and national collaborators}

\begin{enumerate} 

\item Oak Ridge National Laboratory, USA:
Drs.\ Cyrus Baktash, David Dean and Matej Lipoglav\v{s}ek

\item University of Barcelona: 
Profs.\ Artur Polls and Angels Ramos, Msc.\ Isaac Vida\~na

\item University of Strasbourg:
Profs.\ Andres Zuker and Drs.~Etienne Caurier and  
Frederic Nowacki

\item University of T\"ubingen: Prof.\  Herbert M\"uther 

\item GSI, Darmstadt: Prof.\  Hubert Grawe

\item CERN, Isolde: Drs.\ Juha \"Ayst\"o, Markku Oinonen and Teemu Siiskonen

\item University of Jyv\"askyl\"a: Prof.\ Jouni Suhonen

\item University of Moscow, Idaho, USA: Prof.\ Ruprecht Machleidt 

\item Institute of Physics, Belgrade: Dr.~Alexandar Belic

\item Nordita, Copenhagen: Dr.\ Henning Heiselberg 

\item Michigan State University, East Lansing,
USA: Prof.\ Alex Brown 

\item University of Oslo: Profs.\ Torgeir Engeland, Yuri Galperin,
Magne Guttormsen, Profs.~Jon Magne Leinaas,  Eivind Osnes, John Rekstad, 
and Prof.~Leif Veseth

\item University of Bergen: Profs.~Jan Vaagen

\item University College of Oslo: Drs.~Mark Burgess, H\aa rek Haugerud and
Anne Holt
\item Engineering college of Trondheim: Dr.~Lars Engvik 
\end{enumerate}



\section{Nuclear physics research: Introduction and aims}

This research programme aims at \newline
1) To continue present research in the nuclear many-body
problem with emphasis on nuclear structure problems
and further development of many-body approaches applied to nuclear
properties of current international interest
and \newline
2) to widen the scope of the above research project
in order to address 
problems of relevance for nuclear astrophysics, especially
of studies of dense nuclear matter above ordinary terrestrial 
nuclear densities and of neutrino processes in 
neutron stars and type II Supernovae. 

Both these aims fall within existing research in
theoretical nuclear physics in Norway at the Universities
of Bergen and Oslo. However, they allow also for novel  improvements
through e.g., the possibility
to  enhance 
research in Norway in the internationally rapidly 
developing field of nuclear and particle astrophysics.


Nuclear physics has many links to both particle physics
and astrophysics. 
Modern research in the latter fields is closely related
to current experimental and theoretical development
in nuclear physics, such as experiments to detect
the quark-gluon plasma, the study of the abundance of the 
lightest elements in the universe, the equation
of state of nuclear matter relevant for studies of supernovae
and neutron stars, to mention just a few open problems
shared by nuclear physics, particle physics and astrophysics.

However, in order to be able to apply 
methods from nuclear physics to astrophysics,
a firm understanding of nuclear properties is mandatory.
Nuclear physics is the science of atomic nuclei, aiming
at understanding the properties of nuclei, 
their interactions and 
their constituents. Properties of nuclei are determined
by the interplay between strong, electromagnetic and
weak interactions.

At low energies, nuclear properties are determined  
in terms of nucleons and mesons. Information on these
properties is derived from nuclear structure studies,
theoretically and experimentally. At higher energies,
the substructure of nucleons and mesons in terms of
quarks and gluons becomes visible and one of the great
challenges
of modern nuclear physics is to study how elementary particles
like quarks and gluons, described by the  underlying theory
of quantum-chromo-dynamics (QCD), build up hadrons such as mesons
and nucleons. The study of the structure and the 
dynamics of hadrons form then important topics in our basic
understanding of nuclear properties. The way hadronic 
properties change when hadrons are inserted in a nuclear
medium are key issues in understanding, both from
the point of view of QCD and nuclear physics, relativistic
heavy-ion collisions where matter can be heated and 
compressed under extreme conditions. Huge efforts are devoted to
detecting the deconfinement of quarks and gluons into a
plasma phase within such a hot and compressed nuclear medium.

The understanding of the above properties of nuclear systems,
ranging from low to high energies and/or extreme conditions,
are central ingredients in the field of nuclear astrophysics. 
To give an example which will enter this research programme,
it suffices to mention the relation between nuclear physics
and the physics of Supernovae of type II. 
When the iron core of a star undergoes a core collapse to very high
densities, the outcome may be a type II supernovae and/or black hole.
Nuclear physics aspects which strongly enter the final outcome
are the nuclear equation of state and how neutrinos interact with
matter, together with the treatment of the hydrodynamics of the
explosion. The nucleonsynthesis accompanying such an explosion
gives rise to a large fraction 
of the present day abundance of elements.
Final stages of a supernova explosion may be so-called neutron stars,
and the understanding of their structure and properties 
relies also on our understanding of the equation of state 
for nuclear matter and neutrino processes in dense matter.


As mentioned above, one of the aims of this project is to
develop our present understanding of nuclear matter
at high densities and neutrino processes in stellar environments,
in order to improve our current understanding of matter under
such extreme conditions.
The prerequisite for such a programme rests however on a firm understanding
of properties of nuclear systems.
Thus, the structure of this application starts in the next subsection
with a
description of research, present and scheduled, in nuclear structure.
Thereafter, we detail a project on the nuclear structure aspect
of hadron dynamics. The latter, through the understanding of properties
of hadrons in a nuclear medium, is important for modelling of the interior
of neutron stars, especially above nuclear matter densities where 
particles with a strangeness content, hyperons and kaons, may be present.
The nuclear astrophysics projects are then presented in the 
subsequent subsection. 



\subsection{Nuclear structure research}

Research in nuclear structure, especially under extreme conditions such 
as the study of exotic nuclei far from the 
valley of beta stability,  
presents important challenges for nuclear physics. Nuclear structure 
studies are also important  in nuclear 
astrophysics studies, e.g.,
for the understanding of the synthesis of the elements, and to understand 
weak interactions through e.g., neutrino induced reactions on nuclei.

In the field of nuclear structure 
we have mainly worked on developments of many-body techniques 
for the nuclear shell model and with shell-model studies, 
and recently
especially in connection with spectroscopy of exotic nuclei 
in the mass regions around A=56, 100 and 132.
These techniques are also applied to studies of  
neutrino interactions with nuclei and processes of relevance
for the synthesis of the elements.

We are also involved in
extensive collaborations with experimental groups at GSI, 
Argonne and Oak Ridge national laboratories and Isolde at Cern
on the physics of exotic nuclei. 



\subsubsection{Theoretical developments of many-body techniques}

We have been working on a many-body program
based on a microscopic
foundation of the nuclear shell model and other nuclear models.
Applications are made
to a wide range of nuclear systems, from finite nuclei to
theoretical studies of infinite matter with an emphasis
on the equation of state for
nuclear and neutron matter, relevant for astrophysical
studies. The effect of subnucleonic degrees of freedom and
relativistic effects on
various nuclear properties have also been examined.

The philosophy behind this  many-body approach can be
divided in the following three steps:\newline
First, one needs a realistic model for the free
interaction between the various
baryons, an interaction we assume to be well represented
by the exchange of various mesons,
e.g.,\ meson-exchange models for
the nucleon-nucleon potential. 
Secondly, the free potential is renormalized
in the nuclear medium by accounting for the Pauli principle
and other medium effects. This step is
accounted for by the introduction of the so-called
nuclear reaction matrix $G$.
Finally, this renormalized
interaction is employed in a perturbative many-body
scheme, where degrees of freedom not accounted for by the
reaction matrix, are included through higher-order terms in the
perturbation
expansion. This approach is described in Physics Reports
261 (1995) 125, a review article co-authored
with Eivind Osnes and Tom Kuo. 
It forms the basic starting point for most shell-model
studies we have performed, and last but not least, it is
the starting point for any realistic shell-model study.

At present we have also extended the above described many-body formalism.
For finite nuclei we are  now able to sum 
up the so-called Parquet diagrams, which include both
particle-particle-hole-hole and particle-hole correlations
in a non-perturbative way for finite nuclei. 
This scheme allows one to sum up a large class of two-body
diagrams never done before in studies of effective interactions
for the nuclear shell model and studies of quantum dots. 
We are  at present also working on a summation scheme for 
effective three-body diagrams, still within the context
of studies relevant to the nuclear shell model. 
Finally, within the development of many-body approaches,
we plan to study the so-called Exponential ansatz in order
to sum up large classes of many-body terms.
Applications of these new approaches to shell-models
studies are planned.
Several of these many-body techniques will be described  in 
a forthcoming book co-authored with Adelchi Fabrocini (Pisa)
and Artur Polls (Barcelona), see reference 3 below.
\newline\newline
Some recent relevant references are
\begin{enumerate}
\item
    M. Hjorth-Jensen, T.T.S.\ Kuo and E.\ Osnes,
    {\em Realistic effective interactions for nuclear systems},
    Physics Reports {\bf 261} (1995) 125-270
\item M.\ Hjorth-Jensen, {\em Effective interactions
for the nuclear shell model},
in ``Advances in Quantum Many-Body Theory'',
Vol.\ 2, eds.\ R.\ Bishop and N.R.\ Walet, 
(World Scientific, Singapore, 2001), in press, 50 pages approximately
\item Adelchi Fabrocini, Morten Hjorth-Jensen and Artur Polls, 
{\em Quantum many-body methods for finite and infinite systems},
Approved for publication by World Scientific, 
planned finished end 2001, 400 pages
approximately
\end{enumerate}


\subsubsection{Nuclear structure, from $A\sim 56$ to $A\sim 100$}
Doubly magic nuclei and their nearby neighbors are of great 
experimental and theoretical interest providing an excellent
testing ground for large-scale shell-model calculations and
many-body approaches to nuclear physics. Being the heaviest
doubly-magic nucleus, $^{100}$Sn occupies a unique place
among nuclei. However, despite numerous experimental
efforts, little is known about the excited states in this nucleus
or its immediate neighbors. Therefore, information about the 
single-particle energies and core excitation energies nearby
this nucleus have to be inferred indirectly. Recently, the 
Oak Ridge group, see Ref.~[1] below, has reported the first
observation of core-excitations across $N,Z=50$ in $^{99}$Cd and
$^{101}$In. 


In our description of these nuclei, we have employed an effective
interaction determined for a model space with protons in the orbits
$\pi 1p_{1/2}$ and $\pi 0g_{9/2}$ and neutrons in
$\nu 0g_{7/2}$, $\nu 1d_{5/2}$, $\nu 1d_{3/2}$, $\nu 2s_{1/2}$ and
$\nu 0h_{11/2}$ with $^{88}$Sr as closed shell core, see
Ref.~[2] below. This effective interaction reproduces almost
perfectly the excited states of $^{99}$Cd up to 
$J^{\pi}=23/2^+$, where our model space sets a limit for
the angular momentum value. The experiment of Ref.~[1]
reports for the first time states with $J^{\pi}> 23/2^+$,
being an evidence  of excitation across the $^{100}$Sn 
core. To describe such states, we need 
to explore degrees of freedom such as the 
inclusion of the $g_{9/2}$ orbit in our studies of
$^{99}$Cd and
$^{101}$In or the structure of the $^{100}$Sn.

This poses additional challenges to our shell-model approach
and we plan to
\begin{itemize}
\item Compute a particle-hole interaction to be used in shell-model
      studies and
\item allow for truncations in the shell-model code, since the number
      of shell-model states may easily exceed dimensions of the order
      of $\sim 10^8$. 
\end{itemize}

We are currently working on these extensions. 
\newline\newline 
Principal collaborators: T.\ Engeland$^a$, 
A.\ Holt$^a$, E.\ Osnes$^a$, C.\ Baktash$^b$, D.\ Dean$^b$, M.\ Lipoglav\v{s}ek$^b$, E.\ Caurier$^c$, F.~Nowacki$^c$, A.\ Zuker$^c$, 
and H.\ Grawe$^d$\newline
${}^a$ Department of Physics,
University of Oslo\newline
${}^b$ Oak Ridge, USA\newline
${}^c$ CRN IN2P3-CNRS/University Louis Pasteur, Strasbourg, \newline
${}^d$ Gesellschaft f\"ur Schwerionenforschung,
Darmstadt\newline\newline
Some recent relevant references are
\begin{enumerate}
\item 
M.~Lipoglav\v{s}ek, C.~Baktash,
M.~P.~Carpenter, T.~Engeland, C.~Fahlander,
M.~Hjorth-Jensen, R.~V.~F.~Janssens, A.~Likar,
E.~Osnes, S.~D.~Paul, A.~Piechaczek, D.~C.~Radford,
D.~Rudolph, D.~Seweryniak, D.~G.~Sarantites M.~Vencelj, C.~H.~Yu,
{\em First observation of excitation across the} $^{100}$Sn 
{\em core}, in preparation
for Physical Review Letters
\item
    A. Holt, T. Engeland, M. Hjorth-Jensen and E. Osnes,
    {\em Applications of realistic effective interactions to the structure
         of the Zr isotopes},
    Physical Review {\bf C61} (2000) 064318/1-11
\item 
    R.\ Grzywacz, R.\ Beraud, C.\ Borcea, A.\ Ensallem, M.\ Glogowski,  H.\ Grawe, D.\ Guillemaud-Mueller, M.\ Hjorth-Jensen, M.\ Houry,
  M.\ Lewitowicz, A.C.\ Mueller, A.\ Nowak, A.\ Plochocki, 
  M.\ Pf\"utzner, K.\ Rykaczewski, M.G.\ Saint-Laurent, J.E.\ Sauvestre, M.\ Schaefer, O.\ Sorlin, J.\ Szerypo, W.\ Trinder, S.\ Viteritti, J.\ Winfield,
  {\em New island of $\mu s$-isomers in neutron-rich nuclei around the
  Z=28 and N=40 shell closures}, 
  Physical Review Letters {\bf 81} (1998) 766-769 
\item S.M.\ Vincent, P.H.\ Regan, S.\ Mohammadi, D.\ Blumenthal,
      M.\ Carpenter, C.N.\ Davids, W.\ Gelletly, D.J.\ Henderson,
      R.V.F.\ Janssens, M.\ Hjorth-Jensen, C.J.\ Lister, C.J.\ Pearson,
      D.\ Seweryniak, J.\ Schwartz, J.\ Simpson and D.D.\ Warner,
      {\em Yrast study of the $fpg$ shell nuclei} $^{56}$Ni, $^{61}$Cu
      {\em and} $^{61}$Zn,  Physical Review {\bf C60} (1999) 064308/1-18

\item E.\ Caurier, H.\ Grawe, M.\ Hjorth-Jensen, F.\ Nowacki and G.\ 
      Martinez-Pinedo, {\em Core excitations in doubly magic} $^{56}$Ni
      {\em and}  $^{100}$Sn, submitted to Physical Review {\bf C}, rapid communication

\end{enumerate}



\subsubsection{Nuclear structure, from $A\sim 100$ to $A\sim 132$}
Nuclei far from the line of $\beta$ stability are at present in
focus of the nuclear structure physics community. Considerable
attention has been devoted to nuclei close to the doubly-magic
$^{100}$Sn and studies of proton-emitting nuclei are increasingly
popular.
The Oak Ridge group, see Ref.~[1] produced recently 
$^{105}$Sb in the reaction $^{50}$Cr($^{58}$Ni,1p2n)
with a beam energy of 225 MeV and a 2.1 mg/cm$^2$ thick target.
This experiment led to 
the first observation of excited states in $^{105}$Sb, a nucleus which has
4 neutrons and 1 proton more than $^{100}$Sn and is a known 
proton emitter. 

The excitation spectra of $^{106}$Sb and $^{107}$Sb
have also been reported recently. 
In the enclosed table we display theoretical and 
experimental excited states for these nuclei, see also 
Refs.~[2,3].
\begin{table}[hbt]
\begin{center}
\caption{ Low-lying states of $^{105,106,107}$Sb, theory and experiment.
Energies in MeV. }\footnotesize
\begin{tabular}{ccc|ccc|ccc}
\hline
\multicolumn{3}{c|}{ $^{105}$Sb} & \multicolumn{3}{c|}{ $^{106}$Sb}& \multicolumn{3}{c}{ $^{107}$Sb} \\ 
{$J^{\pi}_i$} & {Exp} & {Theory} & 
{$J^{\pi}_i$} & {Exp} & {Theory} & 
{$J^{\pi}_i$} & {Exp} & {Theory} \\
\hline 
$5/2^{+}$ & 0 & 0 & $2^{+}$ & 0 & 0 & $5/2^{+}$ & 0 & 0 \\
$9/2^{+}$ & 1.22 & 1.22 & $4^{+}$ & 0.10 & 0.25 & $7/2^{+}$ & 0.77 & 0.69 \\
$13/2^{+}$ & 1.84 & 1.94 & $5^{+}$ & 0.32 & 0.54 & $9/2^{+}$ & 1.06 & 1.08  \\
$15/2^{+}$ & 2.21 & 2.10 & $6^{+}$ & 0.44 & 0.66 & $11/2^{+}$ & 1.79 & 1.80 \\
$17/2^{+}$ & 2.50 & 2.41 & $7^{+}$ & 0.89 & 1.34 & $13/2^{+}$ & 1.90 & 1.94  \\
$19/2^{+}$ & 2.99 & 2.94 & $8^{+}$ & 1.53 & 1.78 &  $15/2^{+}$ & 2.24 & 2.38\\
$23/2^{+}$ & 3.73 & 4.09 & $10^{+}$ & 2.26 & 2.57 & $17/2^{+}$ & 2.75 & 2.83  \\\hline
\end{tabular}
\end{center}
\end{table}
As can be seen, the agreement with experiment, even for
the odd-odd nucleus $^{106}$Sb, is very good. 

The interaction used, employs $^{100}$Sn as closed shell core,
however, extensive shell-model calculations for heavy tin isotopes
using $^{132}$Sn were performed in Ref.~[4], with an overall 
excellent
agreement with data. Our future plans include
\begin{itemize} 
\item a survey of Sb isotopes,
\item a detailed study of Cd and light Sn isotopes using 
      $^{88}$Sr as closed shell core, see e.g., the previous project,
\item an investigation of effective three-body interactions 
      with an emphasis on the binding energy and 
\item studies of GT transitions near $A\sim 100$ and $A\sim 132$
      (discussed in the next project as well).
\end{itemize}     
Principal collaborators: T.\ Engeland$^a$, 
A.\ Holt$^a$, E.\ Osnes$^a$, 
C.\ Baktash$^b$, D.\ Dean$^b$, M.\ Lipoglavsek$^b$ 
and H.\ Grawe$^c$\newline
${}^a$ Department of Physics,
University of Oslo\newline
${}^b$ Oak Ridge, USA\newline
${}^c$ Gesellschaft f\"ur Schwerionenforschung,
Darmstadt\newline\newline
Some recent relevant references are
\begin{enumerate}
\item 
M.~Lipoglav\v{s}ek, C.~Baktash,
M.~P.~Carpenter, T.~Engeland, C.~Fahlander,
M.~Hjorth-Jensen, R.~V.~F.~Janssens, A.~Likar,
E.~Osnes, S.~D.~Paul, A.~Piechaczek, D.~C.~Radford,
D.~Rudolph, D.~Seweryniak, D.~G.~Sarantites M.~Vencelj, C.~H.~Yu,
{\em Excited states of the proton emitter} $^{105}$Sb, submitted
to Physical Review {\bf C}, Rapid communication



\item T.~Engeland, M.~Hjorth-Jensen and E.~Osnes,
    {\em Shell model studies of the proton drip line nucleus} $^{106}$Sb,
    Physical Review {\bf C61} (2000) 021302R/1-4
\item T.~Engeland, M.\ Hjorth-Jensen, and E.~Osnes, {\em  Effective interactions for medium heavy nuclei}
in 
      proceedings of the
      5$^{th}$ international conference on radioactive nuclear beams, 3-8
       April 2000, Divonne (France) organized by CERN and the ISOLDE Collaboration,
       Nuclear Physics {\bf A}, in press and nucl-th/0007061

 \item
    T. Engeland, M. Hjorth-Jensen, A. Holt and E. Osnes,
    {\em Shell-model calculations of heavy Sn isotopes},
    Nuclear Physics {\bf A634} (1998) 41-56

\end{enumerate}




\subsubsection{Nuclear structure, from $A\sim 132$ to $A\sim 208$}

Beta decay in the region of nuclei around $^{132}$Sn and out to the
neutron drip line is important for the understanding of the 
astrophysical r-process.
Although $^{132}$Sn is a neutron-rich nucleus quite far from the
valley of stability, the nuclei in 
this region are produced in fission and many of their properties have 
been studied experimentally. The observed properties indicate 
that $^{132}$Sn is probably the best closed-shell nucleus which exists
in nature. 

Recent advances in experimental techniques allow for the
measurements of the beta decay strengths of nuclei near $^{132}$Sn
with considerable detail, and provide a means for stringent
tests of microscopic models.
Calculations are presented in Ref.~[1] below 
for the beta decay of $^{132}$Sn and
$^{133}$Sb based upon a microscopic shell-model hamiltonian obtained
with a renormalized G matrix. They are compared with new data
for the $^{133}$Sb decay and with older data for the $^{132}$Sn decay.
The agreement between experiment and theory is excellent
over a range of four orders of magnitude in the reduced transition
probability.

Several of the interactions 
used in this shell-model study, were tested 
in Refs.~[2,3] for the $N=82$ isotones.

Our further plans involve
\begin{itemize}
\item studies of heavy tin isotopes with $A > 132$,
\item GT studies of processes relevant for the 
      astrophysical $r$-process,
\item pairing in Sn isotopes for $A > 132$ and $A < 132$
\end{itemize}
Principal collaborators: T.\ Engeland$^a$, 
A.\ Holt$^a$, E.\ Osnes$^a$, 
J.\ Suhonen$^b$,  
B.A.\ Brown$^c$\newline
${}^a$ Department of Physics,
University of Oslo\newline
${}^b$ Department of Physics,
University of Jyv\"{a}skyl\"{a}\newline
${}^c$ Department of Physics, Michigan University, East Lansing\newline 
\newline\newline
Some recent relevant references are
\begin{enumerate}
\item B.~Alex Brown, M.~Hjorth-Jensen,  M.~Sanchez-Vega, H.~Mach, 
B.~Fogelberg and A.~Lindroth, 
{\em Beta Decay of }$^{132}$Sn {\em and} $^{133}$Sb; a test of microscopic shell model hamiltonians, submitted to
Physical Review Letters

\item
    A. Holt, T.\ Engeland, M.\ Hjorth-Jensen, E.\ Osnes and J.\ Suhonen,
    {\em The structure of the N=82 isotones with realistic effective
     interactions},
    Nuclear Physics {\bf A618} (1997) 107-125
\item
    J.\ Suhonen and J.\ Toivanen, 
    T.\ Engeland, M.\ Hjorth-Jensen, A.\ Holt and E.\ Osnes,
    {\em Study of odd-mass N=82 isotones: comparison of the microscopic
    quasiparticle-phonon model and the nuclear shell model},
    Nuclear Physics {\bf A628} (1998) 41-61
\item 
    G.\ White, N.J.\ Stone, J.\ Rikovska, Y.\ Koh, J.\ Copell,
    T.\ Giles, I.S.\ Towner, B.A.\ Brown, S.\ Ohya, B.\ Fogelberg,
    L.\ Jacobsson, P.\ Rahkila and M.\ Hjorth-Jensen,
    {\em Ground state magnetic dipole moment of $^{135}$}I,
    Nuclear Physics {\bf A644} (1998) 277-288

\end{enumerate}





\subsubsection{Thermodynamical properties of nuclei}
The Oslo Cyclotron group has developed an experimental 
setup for extracting level densities 
at low spin from primary gamma rays.
From the level density, one can define the microcanonical
partition function, which is the correct partition function
for an isolated system such as a nucleus.
Using standard thermodynamical relations between the microcanonical
and the canonical ensemble, we derive, based on the 
level density as a function of excitation energy,  
the nuclear heat
capacity within the framework of the canonical ensemble. 

The density of accessible levels at low spin in the ($^3$He,$\alpha
\gamma$) reaction has been extracted for the $^{162}$Dy, $^{166}$Er and
$^{172}$Yb nuclei. 

The heat capacity in the canonical ensemble for these nuclei
exhibits an S-formed shape as a function of
      temperature, and is interpreted as a fingerprint 
of a transition from a strongly correlated to an uncorrelated phase. 
The critical
temperature for the quenching of pair correlations is found at $T_c=0.50(4)$ 
MeV. 
These topics are discussed in Refs.~[1,2].  
The paper in Ref.~[1] was given extra press attention in 
Physics News Update of August 16 1999 and in Science 
News {\bf 156} (2000) 116.  

Using the same level densities in Ref.~[3] below,
we deduce the entropy of the even-odd and even-even nuclei 
as function of excitation energy.
The entropy of one quasiparticle outside an even-even core is found to be 1.70(15) $k_B$. This
 quasiparticle picture of hot nuclei is well accounted for within a simple pairing model. The
onset of two, four and six quasiparticle excitations in the $^{162}$Dy and $^{172}$Yb nuclei is 
discussed and compared to theory. The number of quasiparticles excited per excitation energy
 is a measure for the ratio of the level energy spacing and the pairing strength.  \newline\newline 
Principal collaborators: M.\ Guttormsen, E.~Melby, J.~Rekstad, A.~Schiller, S.~Siem$^a$, 
A.\ Belic$^b$\newline 
${}^a$ Department of Physics,
University of Oslo\newline
${}^b$ Institute  of Physics, Belgrade\newline

Some recent relevant references are
\begin{enumerate}
\item  E.\ Melby, L.\ Bergholt, M.\ Guttormsen, M.\ Hjorth-Jensen,
       F.\ Ingebrigtsen, J.\ Rekstad, A.\ Schiller, S.\ Siem and
       S.W.\ \O deg\aa rd,  
       {\em Observation of thermodynamical properties in the} 
       $^{162}$Dy, $^{166}$Er, $^{172}$Yb {\em nuclei},
       Physical Review Letters {\bf 83} (1999) 3150-3153
\item  A.\ Schiller, A.\ Bjerve, M.\ Guttormsen, M.\ Hjorth-Jensen,
       F.\ Ingebrigtsen, E.\ Melby,  J.\ Rekstad,  S.\ Siem and
       S.W.\ \O deg\aa rd,  
       {\em The critical temperature for quenching of pair correlations},
       Physical Review {\bf C63} (2001) 21306(R)/1-5
\item  M.\ Guttormsen, A.\ Bjerve, M.\ Hjorth-Jensen,
       E.\ Melby,  J.\ Rekstad, A.\ Schiller, S.\ Siem, and A.~Belic,       {\em Entropy in hot} 
       $^{161,162}$Dy {\em and} $^{171,172}$Yb {\em nuclei},
       Physical Review {\bf C62} (2000) 024306/1-10
\item  M.\ Guttormsen, M.\ Hjorth-Jensen, E.\ Melby,  
       J.\ Rekstad,  A.\ Schiller and S.\ Siem,
       {\em Properties of thermal quasiparticles in nuclei},
       Physical Review {\bf C63} (2001) 044301/1-16
\end{enumerate}


\subsubsection{Muon capture in nuclei,  
neutrino scattering on nuclei}

The aim of this project is to study the role of 
various renormalizations of the effective interaction
and effective operators which enter shell-model studies
of weak processes
where a muon is captured and 
of weak processes with neutrino scattering. 

A typical example for muon capture that we study
is the reaction $^{16}$O$(\mu , \nu)$ $^{16}$N.
Muon capture has great importance in fundamental
physics theories as it can be used in the determination
of the weak-interaction coupling constants.
Compared with the more traditional and well studied
$\beta$ decay with electrons, the energy release in the muon
capture is 200 times larger than that in the electron
capture. This energy transfer from the muon makes 
it possible to excite many nuclear levels in the daughter
nucleus and due to the large momentum transfer the reaction is 
therefore sensitive to those
parts of the weak interaction hamiltonian that are 
not observed in ordinary $\beta$ decay. 


For neutrino scattering on nuclei we plan to look at 
the reaction $^{16}$O$(\nu(\overline{\nu}) ,\nu(\overline{\nu}))$ $^{16}$O 
below particle-emission threshold.
We wish to study here medium renormalizations of the isoscalar axial
coupling constant, as this may affect the predicted 
rates for the above mentioned neutrino-scattering reactions.

Moreover, neutrino scattering processes like  
$(A,Z)+\nu(\overline{\nu}) \rightarrow 
\nu(\overline{\nu}) +(A,Z)$ are also thought to be important
in the synthesis of the elements, where the so-called 
$\nu$-process appears to be very promising in explaining
the so-called $p$-process in the synthesis of the elements.
The $p$-process is important in the understanding 
of the synthesis of the heavy elements in nucleosynthesis
theories.

The $p$-process nuclides are also expected to be synthesized
from nearby $r$-process products through charged current
interactions with the electron neutrinos. For example,
a process we wish to study is the neutrino capture
$^{92}$Zr$(\nu , e^{-})$ $^{92}$Nb and 
$^{92}$Nb$(\nu , e^{-})$ $^{92}$Mo
through which a great deal of $^{92}$Mo could be produced.
\newline
\newline
Collaborators : T.\ Siiskonen$^a$ 
and J.\ Suhonen$^b$\newline
${}^a$ Cern, ISOLDE\newline
${}^b$ Department of Physics,
University of Jyv\"askyl\"a\newline\newline
Some recent relevant references are
\begin{enumerate}
\item T.\ Siiskonen, J.\ Suhonen and M.\ Hjorth-Jensen, {\em  
      Towards the solution of the $C_{P}/C_{A}$ anomaly in shell-model 
      calculations of muon capture}, Physical Review {\bf C59} (1999) 
      R1839-R1843
\item T.\ Siiskonen, J.\ Suhonen and M.\ Hjorth-Jensen, {\em  
      Shell-model effective operators for muon capture in} $^{20}$Ne,
      Journal of Physics {\bf G 25} (1999) L55-L61
\item T.\ Siiskonen, M.~Hjorth-Jensen, and J.\ Suhonen, {\em Renormalization of of the weak hadronic
current in the nuclear medium"}, 
Physical Review {\bf C63} (2001) 055501/1-12
\end{enumerate}


\subsubsection*{Superallowed  Fermi $\beta$-transitions}

Superallowed Fermi $\beta$-transitions in nuclei provide an 
excellent laboratory for precise tests of the properties of the
electroweak interaction and have been the subject of intense 
study for several decades. According to the conserved-vector-current
(CVC) hypothesis, for pure Fermi transitions the product of
the partial half-life $t$ and the statistical phase-space factor
$f$ should be nucleus independent (after correcting for the
charge-dependent effects). 
These transitions  provide thus   an excellent test for the conserved
vector current hypothesis (CVC) and for the unitary of the
Cabibbo-Kobayashi-Maskawa matrix (e.g., Phys. Rev. C52, 2455 (1995)). The
extracted results are sensitive to the level spacings in the parent and
daughter nuclei. The superallowed beta decay of $^{74}$Rb is going to be
measured at ISOLDE, CERN, within a few months.  What this experiment is
looking for is the spectrum of the low-lying collective $0^+$ states,
which, at the moment, is poorly known. Previous calculations fail in
reproducing the energy of the first excited $0^+$ state mainly because of
the small shell-model space ($fp$-shell only).  Therefore the inclusion of
the $g_{9/2}$ orbit is necessary to gain enough collectivity to push the
calculated energies down by the required amount (roughly 2 MeV). This will
increase the dimension of the Hamiltonian matrix roughly by an order of
magnitude. With such a calculation, more severe restrictions on the
validity of CVC can be obtained.
\newline
\newline
Collaborators : Juha \"Ayst\"o$^a$, M.\ Oinonen$^a$ and T.\ Siiskonen$^a$ \newline
${}^a$ Cern, ISOLDE\newline


\subsection{Hadron properties in the medium: Nuclear structure aspect}


The last decade, through measurements done at NIKHEF in
Holland, at MAMI in Mainz and now TJNAF in the USA, has been marked by a 
remarkable interplay between many-body theory and high precision 
electron scattering experiments.
The nuclear response has been measured at high momentum and 
into the continuum. The partial occupancy of mean-field orbits obtained
from these experiments is one
of the cleanest   signatures of nucleon-nucleon correlations.
To understand such correlations forms a very active field in nuclear physics
and to elucidate the short-distance structure of nuclei will be a topic
of special interest in the future. Nuclear structure studies of hyperons
provide also a new input to the nuclear many-body problem. Hyperons
are baryons with a strangeness content. Studies of nuclei with a 
hyperon content, so-called hypernuclei, allow for both a study of 
weak interaction and to gain information in order to constrain the 
strong interaction between hyperons and nucleons.            

Finally,
properties of hyperons    in     dense matter, nucleons
and the isobar $\Delta$ are also of importance for the description
of matter in the interior of a neutron star, see next section. 

\subsubsection{Self-energy of baryons}


We have investigated, together with colleagues from
Germany and Spain,  the self-energy of various baryons
in dense matter and finite nuclei.
The self-energy can be
used as an important ingredient for an evaluation of the structure
function or response function of nuclei, which are used to
analyze the excitation modes of nuclei as they are observed
e.g.\ in (e,e') experiments.

From the self-energy one can furthermore derive the single-particle
Green's function through the solution of the Dyson equation and
the spectral function depending on momentum
and energy. The integration of the spectral function with respect
to the energy yields the occupation
probabilities, which provide important information on nuclear
correlations. Several experimental studies have been made
to obtain reliable information on these quantities.


In previous works we have mainly studied the nucleon
self-energy. Recently we have also studied
the self-energy of the isobar $\Delta$ in a nuclear medium like
$^{16}$O or $^{40}$Ca.
Moreover, we have also started to study
baryons with a strangeness content,
such as the
$\Lambda$ and $\Sigma$, and their behavior in a nuclear
medium.
Of importance here is the evaluation of the self-energy for these
baryons, since the self-energy is intimately related to the
decay width.
This is especially relevant for the $\Sigma$, since the existence
of relatively long-lived $\Sigma$-bound states has still to be
understood. Such studies may also allow to discriminate
between various nucleon-hyperon potential models. Contrary to
what is the case for the nucleon-nucleon potential, where experimental
data  constrain the on-shell scattering matrix, few data exist
in the hyperon-nucleon channel.
\newline
\newline
Collaborators :
H.\ M\"{u}ther${}^a$,
A.\ Polls${}^{b}$, A.\ Ramos${}^{b}$ and I.\ Vidanya${}^{b}$\newline
${}^a$ Institut f\"{u}r Theoretische Physik,
Universit\"{a}t T\"{u}bingen,\newline
${}^{b}$ Departament d'Estructura y Constituentes de la Materia,
Universitat de Barcelona\newline\newline
Some recent relevant references are
\begin{enumerate}
\item
    M. Hjorth-Jensen, H. M\"{u}ther and A. Polls,
    {\em Width of the $\Delta$ resonance in nuclei},
    Physical Review {\bf C50} (1994) 501-505
\item M.\ Hjorth-Jensen, H.\ M\"{u}ther, A.\ Polls and A.\ Ramos,
    {\em Self-energy of $\Lambda$  in finite nuclei},
    Nuclear Physics {\bf A605} (1996) 458-474
\item I.\ Vida\~na, A.\ Polls, A.\ Ramos and  
       M.\ Hjorth-Jensen,
    {\em Hyperon properties in finite nuclei using realistic YN interactions},
    Nuclear Physics {\bf A644} (1998) 201-220
\end{enumerate}

\subsection{Nuclear astrophysics}

Our research in nuclear astrophysics has dealt mainly 
with theoretical
studies of the equation of state for dense nuclear matter,
determination of properties of neutron stars such as the total mass,
radius, phase transitions and the composition
of matter in the interior of a neutron star and the interesting topic
of superfluidity in neutron stars. 
Neutron stars have a rich structure, where the outermost layers 
are rather similar to terrestrial matter. With increasing depth
in the star, and thereby increasing density, nuclei become more and
more neutron rich until at a density of about one thousandth of nuclear
matter saturation density, nuclei reach the so-called neutron drip line.
At higher densities nuclei coexist with a neutron liquid and they 
eventually dissolve just below nuclear matter saturation density.

Important issues in neutron star studies deal with theoretical
determinations of the equation of state up to densities
several times nuclear matter saturation density.
At high densities matter consists of interacting baryons (neutron, protons
and possibly hyperons and other particles) and/or quarks in beta-equilibrium
with leptons.
In addition, bose condensates of pions or kaons may be present.
The central  problem is then 
to develop reliable techniques for calculating
properties of strongly correlated matter. This is crucial since
95\% of the matter in a neutron star is located in regions with 
densities above nuclear matter saturation density.
Other topics are
the total proton/neutron ratio and neutrino emission processes.
The latter processes are also of great importance since the loss of heat
through neutrino emissions and the measurement of surface temperatures
of a neutron star provides a way of probing neutrino processes
in the star. Since neutrino emissions are very sensitive to the composition
of dense matter,  such measurement may therefore provide information
on the interior of neutron stars. 

Necessary tools in such studies are various many-body techniques.
In the present project,  
the summation of the Parquet class of diagrams discussed in section 2.2.1 is 
currently planned extended 
to studies of infinite nuclear and neutron matter of relevance 
for nuclear astrophysics. 
The summation of such classes of diagrams is important
in order to obtain effective interactions for infinite matter.
Such an effective interaction can in turn be used
in order to define the equation of state for nuclear,
neutron matter and $\beta$-stable matter.

A further goal is also to include three-body forces, realistic
and effective, in our nuclear matter program, in addition to studies
of hyperons at higher densities.
Finally, a careful study of various neutrino emissions in neutron
stars is planned. The various projects are listed below.

\subsubsection{Equation of state for neutron stars}

The physics of compact objects like neutron stars offers
an intriguing interplay between nuclear processes  and
astrophysical observables.
Neutron stars exhibit conditions far from those encountered on earth;
typically, expected densities of a neutron star interior are of the
order $10^3$ or more than the density 
at neutron drip  ($10^{11}$ g/cm$^{3}$).
Thus, central to calculations of neutron star properties, is the
determination of an equation of state for dense matter. This determines
the mass range as well as the mass-radius relantionship for these stars.
It is also an important ingredient to the determination of the
composition of dense matter and to how
thick the crust of a neutron star is.
The latter influences neutrino generating processes and the cooling
of neutron stars.

We have studied properties of neutron
stars like
the total mass, radius, moment of inertia, and
surface gravitational redshift  for neutron
stars using equations of state (EOS) with different proton fractions.
Modern meson-exchange potential models are used to
evaluate the $G$-matrix for asymmetric nuclear matter.
We calculate both a non-relativistic and a relativistic EOS.
Of importance here is the fact that relativistic Brueckner-Hartree-Fock
calculations for symmetric nuclear matter fit the empirical data, which
are not reproduced by non-relativistic calculations.
Relativistic  effects are known to be important at high densities, 
giving an increased repulsion. This leads  to a stiffer EOS
compared to the EOS derived with a non-relativistic approach.

We have also studied
properties of neutron star crusts, where
matter is expected to consist of 
nuclei surrounded by superfluid 
neutrons and a homogeneous background of relativistic electrons. 

The aim of our present study of neutron star
properties is to apply new
developments in many-body theory, see above,
to derive an equation of state with nucleonic degrees of freedom to
$\beta$-stable matter. Furthermore, inclusion of hyperon 
degrees of freedom have recently been performed and further analyses
are planned, such as studies of hyperon superfluidity and 
neutrino emissions from electroweak processes involving
hyperons, see the next three subsections.
\newline
\newline
Principal collaborators: \O.\ Elgar\o y$^a$,
L.\ Engvik$^a$, E.\ Osnes$^a$, F.\ de Blasio$^a$,
H.\ Heiselberg$^b$, R.\ Machleidt${}^c$, A.\ Polls${}^d$,  
H.\ M\"uther${}^e$\newline
${}^a$ Department of Physics,
University of Oslo\newline
${}^b$ Nordita, Copenhagen\newline
${}^c$ Department of Physics,
University of Idaho, USA\newline
${}^{d}$ Departament d'Estructura y Constituentes de la Materia,
Universitat de Barcelona\newline
${}^e$ Institut f\"{u}r Theoretische Physik,
Universit\"{a}t T\"{u}bingen,\newline\newline
Some recent relevant references are
\begin{enumerate}
\item    
    L.\ Engvik, M.\ Hjorth-Jensen, R.\ Machleidt, H.\ M\"uther and
    A.\ Polls,
    {\em Modern nucleon-nucleon potentials and symmetry energy in infinite
         matter},
    Nuclear Physics {\bf A627} (1997) 85-100
\item 
    H.\ Heiselberg and M.\ Hjorth-Jensen,
    {\em Phase transitions in rotating  neutron stars}, 
    Physical Review Letters {\bf 80} (1998) 5485-5488
\item    
    A.\ Polls, H.\ M\"uther, R.\ Machleidt and
    M.\ Hjorth-Jensen, 
    {\em Phase-shift equivalent NN potentials and the deuteron},
    Physics Letters {\bf B432} (1998) 1-7
\item 
        H.\ Heiselberg and M.\ Hjorth-Jensen,
    {\em Phase transitions in neutron stars and maximum masses}, 
    Astrophysical Journal Letters {\bf 525} (1999) L45-L48
\item 
    H.\ Heiselberg and M.\ Hjorth-Jensen,
    {\em Phases of dense matter in neutron stars}, 
    Physics Reports {\bf 328} (2000) 237-327

\end{enumerate}


\subsubsection{Hyperons in  neutron stars}
In Ref.~[1] we have developed a formalism for microscopic Brueckner-type calculations of dense nuclear matter that includes all types of
      baryon-baryon interactions and allows to treat any asymmetry on the fractions of the different species (n, p, $\Lambda$, $\Sigma^0$,
      $\Sigma^+$, $\Sigma^-$, $\Xi^-$ and $\Xi^0$). 
This formalism has in turn been applied to 
microscopic studies of beta-stable neutron star matter with strangeness. We find that both the
      hyperon-nucleon and hyperon-hyperon interactions play a non-negligible role in determining the chemical potentials of the different
      particle species. 
In Ref.~[2] we present results from Brueckner-Hartree-Fock calculatons for beta stable neutron star matter with nucleonic and hyperonic degress
      degrees of freedom, employing the most recent parametrizations of the baryon-baryon interaction of the Nijmegen group. It is found that
      the only strange baryons emerging in beta stable matter up to total barionic densities of 1.2 fm$^{-3}$ are $\Sigma^-$ and $\Lambda$. The
      corresponding equations of state are then used to compute properties of neutron stars such as masses and radii. 

This study is the first of its kind, since microscopic many-body
calculations employing realistic baryon-baryon interactions
had never been performed before. 



Principal collaborators: 
L.\ Engvik$^a$, A.\ Polls${}^b$, A.\ Ramos${}^b$,  I.\ Vida\~na${}^b$\newline
${}^a$ Department of Physics,
University of Oslo\newline
${}^{b}$ Departament d'Estructura y Constituentes de la Materia,
Universitat de Barcelona\newline\newline
Some recent relevant references are
\begin{enumerate}
\item I.\ Vida\~na, A.\ Polls, A.\ Ramos,   
       M.\ Hjorth-Jensen and V.G.J.\ Stoks
    {\em Strange nuclear matter within the Brueckner-Hartree-Fock theory},
    Physical Review {\bf C61} (2000) 025802/1-12
\item I.\ Vida\~na, A.\ Polls, A.\ Ramos, L.\ Engvik and  
       M.\ Hjorth-Jensen,
    {\em Hyperon-hyperon interactions and properties of neutron star matter},
    Physical Review {\bf C62} (2000) 035801/1-8

\end{enumerate}



\subsubsection{Pairing in $\beta$-stable matter}

Superfluidity and superconductivity of matter in neutron stars is 
expected to have a number of consequences directly related 
to observation.
Among processes that will be affected is the
emission of neutrinos. Neutrino emission from e.g.\ 
various URCA processes  is expected to be the dominant 
cooling mechanism in neutron stars less than $10^5-10^6$ years old.
Typically,
proton superconductivity  reduces considerably the energy losses
in so-called modified URCA processes 
and may have important consequences for the
cooling of young neutron stars.
Another  possible manifestation of superfluid 
phenomena in neutron stars  
is glitches in rotational frequencies  observed in a number
of pulsars. Moreover, 
the estimation of superfluid gaps and studies of pairing 
are not only important issues
in neutron star matter, but also in the rapidly developing 
field of neutron-rich systems such as heavy nuclei
close to the neutron drip line or the study of 
light halo nuclei. 
Therefore, theoretical studies
of pairing in neutron-rich assemblies form currently a central
issue in nuclear physics and nuclear astrophysics. 

In our studies of pairing hitherto, we have
not included core-polarization or screening effects. 
These are expected to influence considerably the value of
pairing gaps in infinite matter. 
Our plan here is therefore
to  include the particle-hole terms 
in the effective interaction used to calculate the pairing gaps.
Especially, we plan to study the $^1S_0$ gaps for protons and
neutrons and the $^3P_2$ gap for neutrons.
Their influence on  neutrino emissivities will be discussed
in the next subsection.
We plan also to study hyperon superfluidity in matter, based on our
recent calculation of $\beta$-stable matter with hyperons, see the 
previous project.
\newline
\newline
Collaborators: \O.\ Elgar\o y$^a$,
L.\ Engvik$^a$, E.\ Osnes$^a$, F.\ de Blasio$^a$, M.\ Baldo$^b$
and H.-J.\ Schulze$^b$ \newline
${}^a$ Department of Physics,
University of Oslo\newline
${}^b$ INFN and Department of Physics, University of Catania\newline
\newline\newline
Some recent relevant references are
\begin{enumerate}
\item
    \O.\ Elgar\o y, L. Engvik,
    M.\ Hjorth-Jensen
    and E. Osnes,
    {\em Superfluidity in $\beta$-stable neutron
         star matter},
    Physical Review Letters {\bf 77} (1996) 1428-1431
\item
    \O.\ Elgar\o y, L. Engvik, E.\ Osnes, F.V.\ De Blasio,
    M.\ Hjorth-Jensen and G.\ Lazzari,
    {\em Superfluidity and  neutron star crust matter},
    Physical Review {\bf D54} (1996) 1848-1851
\item
    \O.\ Elgar\o y, L. Engvik,
    M.\ Hjorth-Jensen
    and E. Osnes,
    {\em Model-space approach to $^1S_0$ neutron and proton pairing 
      in neutron star matter with the Bonn meson-exchange potentials},
    Nuclear Physics {\bf A604} (1996) 466-490
\item
    \O.\ Elgar\o y, L. Engvik,
    M.\ Hjorth-Jensen
    and E. Osnes,
    {\em Triplet pairing of neutrons in $\beta$-stable neutron star
         matter},
    Nuclear Physics {\bf A607} (1996) 425-441
\item
    \O.\ Elgar\o y and
    M.\ Hjorth-Jensen,
    {\em Nucleon-nucleon phase shifts and pairing in infinite matter},
    Physical Review {\bf C57 } (1998) 1174-1177
\item
    \O.\ Elgar\o y, L.\ Engvik, 
    M.\ Hjorth-Jensen and E.\ Osnes,
    {\em Minimal relativity and $^3S_1$-$^3D_1$ pairing in symmetric nuclear matter},
    Physical Review {\bf C57} (1998) R1069-R1072 
\item
    M.\ Baldo, \O.\ Elgar\o y, L. Engvik,
    M.\ Hjorth-Jensen and H.-J.\ Schulze,
    {\em Modern nucleon-nucleon potentials and $^3P_2$-$^3F_2$ pairing 
      in neutron matter},
    Physical Review  {\bf C58} (1998)  1921-1928 
\end{enumerate}

\subsubsection{Neutrino emissivities in neutron stars}



The thermal evolution of a neutron star may provide information
about the interiors of the star, and in recent years much effort
has been devoted in measuring neutron star temperatures, especially
with the Einstein Observatory and ROSAT.
The main cooling mechanism in the early life of a neutron star is
believed to go through neutrino emissions in the core
of the neutron star.
The most powerful energy losses are expected to be given by the so-called
direct URCA mechanism
\begin{equation}
    n\rightarrow p +e +\overline{\nu}_e, \hspace{1cm} p+e \rightarrow
    n+\nu_e .
    \label{eq:directU}
\end{equation}
However, in the outer cores of massive neutron stars and in the
cores of not too massive neutron stars ($M < 1.3-1.4 M_{\odot}$), the direct
URCA process is allowed at densities
where the momentum conservation $k_F^n < k_F^p + k_F^e$ is
fulfilled. This happens
only at densities $\rho$ several times
the nuclear matter saturation density $\rho_0 =0.16$ fm$^{-3}$.


Thus, for long time the dominant processes for neutrino emission
have been the so-called modified URCA processes first discussed by
Chiu and Salpeter, in which the two reactions
\begin{equation}
    n+n\rightarrow p+n +e +\overline{\nu}_e,
    \hspace{0.5cm} p+n+e \rightarrow
    n+n+\nu_e ,
    \label{eq:ind_neutr}
\end{equation}
occur in equal numbers.
These reactions are just the usual processes of neutron
$\beta$-decay and electron capture on protons of Eq.\ (\ref{eq:directU}),
with the addition of an extra bystander neutron. They produce
neutrino-antineutrino pairs, but leave the composition of matter constant
on average. Eq.\ (\ref{eq:ind_neutr}) is referred to as the
neutron branch of the modified URCA process. Another branch is the
proton branch
\begin{equation}
    n+p\rightarrow p+p +e +\overline{\nu}_e, \hspace{0.5cm} p+p+e
    \rightarrow
    n+p+\nu_e .
    \label{eq:ind_prot}
\end{equation}
Similarly, at higher densities, if muons are present we may also
have  processes where the muon and the muon neutrinos 
($\overline{\nu}_{\mu}$ and $\nu_{\mu}$) 
replace the electron and the electron neutrinos
($\overline{\nu}_e$ and $\nu_e$) in the above equations.
In addition one also has the possibility of neutrino-pair
bremsstrahlung,
processes with baryons more massive than the nucleon
participating, such as isobars or hyperons
or neutrino emission from more exotic states like pion and kaon
condensates or quark matter.

The aim of this project is to reanalyze the various neutrino
emissivities discussed above accounting for the short-range part
in a self-consistent way, including as well the role
of pairing discussed in the previous subsection. 

\section{Quantum computing and Quantum information theory}

This is a newly established research programme, in collaboration
with Profs.~Yuri Galperin, Jon Magne Leinaas and Leif Veseth,
all at the department of Physics, University of Oslo.

\subsection{Scientific objectives}
\begin{itemize}
  \item Develop basic research within quantum mechanical aspects related
        to `Quantum information and quantum computation (QIC)':
\begin{itemize}
	\item {\sl Study various implementations of QUBITS (quantum mechanical
        representation of bits) and many-qubit systems from a
        fundamental point of view;}
	\item {\sl Develop theoretical models for loss of coherence in
quantum systems due to finite temperature and interaction with environment. }
\end{itemize}
  \item %Besides basic theoretical research on quantum mechanical
        %problems, an important part of this application is to Study
        technological implementations of QIC, both from a theoretical
        and an  experimental point of view, and trigger
        related experimental activities in Norway:
\begin{itemize}
\item {\sl Prepare a theoretical basis to start relevant technological
activities in Norway;}
\item{\sl Study alternative implementations
of QIC of relevance to
Norwegian activities in the fields of nanotechnology and microelectronics; }
\item {\sl Develop a collaboration with the Norwegian programme on
microelectronics.}
\end{itemize}

\end{itemize}

QIC will most likely pave the way for the next technological
revolution. It is therefore of strategic importance
that research within this field is started in Norway.

\subsection{Educational objectives}

%Since this is a rapidly developing field and activities, experimental and
%theoretical ones,  in Norway are essentially non-existent,
A major task of this project is to  provide an {\sl educational background}
for experimental and theoretical research activity in QIC,  which at
the present time essentially does not exist in Norway.

We foresee the establishment of {\sl course activities} directed towards
studies of QIC aspects and the engagement of Master of Science (Msc) and
doctoral students (PhD) in QIC topics.
%The aim is that these
These students will 
later
continue and broaden the field of
research in QIC. 
In this way they will 
enable Norwegian research and development to reach an
internationally competitive level.


\subsection{Introduction and Motivation}

The new field of {\em quantum information and computation},
hereafter abbreviated to QIC,  is based on an
understanding of the fundamental difference between classical and quantum
physics at the level of statistical interpretation and information
content. This difference at an early stage lead to strong opposition from
{\em Albert Einstein} and others, who thought that the (Copenhagen)
formulation of quantum theory could not be correct (or complete) due to
its deviation from basic principles concerning the relation between
physical reality and statistical uncertainty. This difference was
sharpened by {\em John Bell} who showed that it could not be due to
incompleteness of the description, but that a genuine, measurable
difference between classical and quantum physics was present at this
point.

A few years ago there was much interest to check experimentally these
differences, to investigate the {\em Bell inequalities}. In
particular an experiment by {\em Alain Aspect} clearly showed violations
of limits set by classical theories. In recent years there has been an
increased experimental activity in the field, but the focus of interest
has changed in the direction of utilizing the peculiarities of quantum
theory rather than to confirm their presence. Much of the activity
has been motivited by the theoretical possibility of creating a {\em
quantum computer} and also by theories on {\em quantum cryptography} and
ideas of using {\em quantum teleportation} in information transfer.

An interesting feature of this new
activity is that it makes use of aspects of quantum theory that are
usually not much discussed in textbooks. These aspects have to do with
the {\em interpretation of quantum mechanics}. This is not a new field of
physics, but it has often been disregarded due to the pragmatic view that
one should concentrate on the operational meaning of quantum mechanics.

Thus, recently, technological issues as to whether quantum mechanics
and QIC ideas could be applied in order to construct computers,
whose operation are governed by quantum mechanics, have gained
momentum. Such issues are intimately connected with
the  dramatic miniaturization
in computer technology seen over the last 40 years.
In few years the basic memory components
of a computer will be the size of individual atoms.
Presently, of the order of 100000 electrons are needed in order
to store one bit of information. Within some 15-20 years,
conventional micro-electronics is predicted to
hit the wall due to nanoscale size problems, the reason being
well-documented physical, technical and manufacturing
problems.

QIC is therefore re-inventing the foundations of
computer science and information theory in a way
that is consistent with quantum physics, the most accurate
model of reality that is currently known.
Nature, and its pertinent description via quantum theory,
offers a natural encoding of information, since states
of atoms or molecules can be utilized to represents
bits 0 or 1. The quantum mechanical representation of
bits are popularly labeled QUBITS.

Quantum computers are also expected to perform
certain computational tasks exponentially faster than any
conventional computer. Moreover, quantum effects
allow unprecedented tasks to be performed such as
teleporting information, generating true random numbers and
communicating with messages that betray the presence of
eavesdropping, and schemes for sending and receiving
ultra-secure messages have already been implemented.

In the US theoretical and experimental activities within QIC started
several years ago.  Major successes with ion-trap and NMR realizations
of logic gates such as XOR or Controlled NOT gates have been
achieved. In Europe, where nano-technology has had a strong basis,
there are several research programmes directed towards the realization
of a solid-state quantum computer. One of the promising ones is
labeled `Superconducting qubits' and is described in the next section
under our scientific programme.  In addition, a programme has been
created in Australia with the dedicated aim of making a solid-state
based quantum computer. Thus, QIC will most likely pave the ground for
the next technological revolution.


\subsection{Scientific programme}

Our scientific activity will concentrate on four topics
\begin{enumerate}
\item \underline{\sl Quantum theory and information.\/} This project provides
      the basis for understanding quantum mechanical issues
      related to quantum computers. Moreover, it is a research field
      per se aiming at studying the foundations and implications
      of quantum mechanics as the presently best theory of nature.
\item \underline{\sl Superconducting qubits.\/} A research project closely
      related to a recent EU programme. Presently, together with ion-trap
      and NMR approaches it represents one of the viable approaches
      towards construction of a solid-state quantum computer. In addition,
      there are several fundamental basic research problems connected to it.
\item \underline{\sl Quantum dot qubits.\/} Quantum dots are systems of
      few trapped
      electrons in two dimensions. Such systems
      involving states of quantum dot (QD) spins
      coupled to a microcavity mode have recently been launched as
      possible candidates for quantum computers.
\item \underline{\sl Atoms in ionic traps and microwave cavities.\/}
      Recent experiments have demonstrated the feasibility of
      making logic gates with trapped ions. These aspects will
      be further explored in this research project.
\end{enumerate}

Our projects are described below.


\subsubsection{Quantum theory and information}

The basic physics involved has to do with the investigation of and
application of {\em quantum correlations}. Typically one uses what is
known as {\em entangled states}. {\em Quantum measurement theory} is used
to describe the manipulations and measurements on such states. Much
of the theory on quantum correlations is well established. However, there
are still many open problems concerning the description of the coupling
between a quantum system and its environment, loss of coherence and the
corresponding correction of quantum information.

There are also basic questions concerning relationships
between quantum theory and information which are not settled. The
%connection
relationship between information and quantum {\em entropy} is of
importance in this connection. It is of interest to note that in {\em
quantum field theory} one meets related questions. This applies to the
so-called {\em quantum holographic principle} which states that at a
fundamental level all information of a quantum system is present at its
boundary, and it applies to the information paradox in {\em black hole
physics} [1]. These questions are all of interest for the
activity under the present programme, but with emphasis on the aspects
related to the application of quantum information and computation.

\subsubsection{Superconducting qubits}

Superconducting technologies have the unique potential for realizing
compact solid-state devices with  controllable macroscopic quantum
properties and long coherence time. They represent probably the most
realistic approach for a technology of quantum computers. In
superconductors, all electrons are condensed in  the same macroscopic
quantum state, separated by a gap from the many quasi-particle states.
Superconductors can be weakly coupled with Josephson tunnel junctions
(regions where only a thin oxide  separates them). The current through
a Josephson junction depends upon the phase differences between the
superconductors which act as non-commuting conjugate quantum variables
to the charges of isolated islands.  That makes it possible to construct
qubits using superconductors (SQUBIT).

So far, superconducting electronics has not been able to compete
with Si- and GaAs-technology in the  field of computers, not even for
special supercomputers. However, in the emerging field of Quantum
Computing the situation is completely different. Now ``quantum
coherence'' is the key issue and  superconductivity has great
advantages due to its built-in principle of ``macroscopic quantum
coherence''. An  important feature of superconducting junctions is a
possibility to reach long decoherence times. The main  reason for that
is a weak sensitivity of properly designed SQBITs to external electric
fields produced by charge fluctuations. A very important step has very
recently been taken by Nakamura et al. at the NEC  research laboratory
in Japan~[2],  who have demonstrated that it is
possible to manipulate qubits in a quantum coherent way for reasonably
long times in a Cooper pair
box.

The role of dissipation and it's influence on decoherence in these
circuits needs to be understood and  controlled better. Studies of
this issue will be one of the directions of our activity along the
project. They will  include description of the transfer of Cooper
pairs in gated arrays of small Josephson junctions in the  adiabatic
regime, as well as the control of Josephson currents by manipulating
Andreev levels in ballistic  junctions. The main focus will be made on
the interaction of these systems with environment which is the  source
of decoherence. Another direction is to suggestion and investigation
of new principle for superconducting qubits. In particular, a
coherent charge transfer by ``shuttling'' of  Cooper pairs will be
studied.

In this research we will use experience from previous activities on
physics of fluctuations in various mesoscopic systems: point
contacts~[3], SQUIDs~[4], Coulomb-blockaded~[5,6]
and ballistic [7] superconductor devices, and  Andreev quantum
interferometers~[8].


\subsubsection{Quantum dot qubits}

The quantum computation schemes based on Raman-coupled low-energy
states of trapped ions and nuclear spins in chemical solutions provide
methods of fast quantum manipulation of qubits. Even though these
schemes are likely to provide examples of quantum information
processing at 5-10 qubit level, they do not appear to be scalable to
larger systems containing more than 100 qubits. More suitable for
large systems seem the schemes involving states of {\sl quantum dot}
(QD) spins coupled to a microcavity mode. A Quantum dot is a confined
system of few electrons which behaves as an {\sl artificial}
atom. This fact implies that coherent states of QD can be used as
qubits. An advantage of QD systems is that they can be produced in a
controllable way by means of modern semiconductor technology. Another
advantage is that their interaction with environment can be rather weak 
and hence being able to provide long coherence times.


Several systems of QD type have recently been proposed, see
e.g.,~[9]. The advantages of the schemes of this type are (i)
possibilities to scale to $\ge 100$ couple qubits; (ii) very long spin
decoherence in III-V and II-VI semiconductors; (iii) possibilities to
organize long-distance fast interaction between qubits via microcavity
modes. The scheme proposed in the above-mentioned paper relies on the
use of a single cavity mode and laser fields to mediate coherent
interactions between distant QD spins. The doped QDs were embedded
into a microdisk structure. The doping level has been chosen to make
the valence band fully occupied and to leave a single conduction
electron state. The qubit was organized by two spin states of the
conduction electron split by external magnetic field. The important
point is the Raman coupling of these states via strong laser fields
and microcavity mode of the disk structure. It is very important that
the dots should {\sl not necessarily be identical}. The main feature
of the scheme is that only the dots which are in resonance with the
laser radiation are coupled. Consequently, the dots are addressed {\sl
individually}. The key issue regarding schemes of such type is the
possibility to provide long spin life-times. In this connection,
theoretical studies of the interactions leading to spin depolarization
are extremely important. According to recent measurements, one can
expect that the relaxation time in GaAs QDs can exceed $1 \ \mu$sec,
a rather promising feature. 
Presently the major technological
limitation for the proposed scheme is the relatively large photon loss
which, in principal, can be improved by a proper design of the system.

An important issue is increasing the number of QDs which are coherently
coupled. In this respect the so-called {\sl self-organized} QDs can be
promising since the distribution of dot parameters in this case is
rather narrow.

\subsubsection{Experiments on atoms in ionic traps and microwave cavities}

During the last five years several elegant experiments have been
performed that demonstrate entangled states of material bodies like
atomic ions in electromagnetic traps and neutral atoms in microwave
cavities. These experiments actually also verify the possibility of
making logic gates based on the quantum behavior of individual atoms,
i.e. the basic building blocks of a quantum computer.  The first of
these ground-breaking experiments were performed at NIST in Boulder by
the group of David Wineland (Phys. Rev. Lett. 18. December 1995).  In
this experiment a two-bit "controlled-NOT" quantum logic gate
(CN-gate) was realized through manipulations of a single laser-cooled
Be$^+$ atom in a Paul electromagnetic trap.  A CN-gate is a logic unit
with two input lines, one of which is the control line and the other
the target line, and with two output lines. If the input control line
carries the bit 1 the bit on the target line is reversed, whereas
nothing happens if the control input bit is 0.  A quantum CN-gate
might be realized by entangling two two-level atoms ("two-bit" atoms),
i.e. the two levels of the "target" atom are switched only if the
"control" atom is in its 1-bit state.  This is achieved in the cavity
experiment to be briefly described below.

Wineland et al. realized their CN-gate by entangling external and
internal degrees of freedom of a single Be$^+$ ion.  In the harmonic
force field of a Paul trap the Be$^+$ ion will be in one of its
vibrational states $|0\rangle$, $|0\rangle$, etc. , however, in this
case laser cooled to the lowest level $|0\rangle$.  The two internal
states are specified as spin-up and spin-down states respectively,
i.e. $|u\rangle$ and $|d\rangle$.  In the experiment the Be$^+$ ion was
exposed to a series of laser pulses that enabled switching between the
two lowest vibrational levels $|0\rangle$ and $|1\rangle$, as well as
switches between the $|u\rangle$ and $|d\rangle$ states.  The
following transitions were obtained: $|0u\rangle\rightarrow
|0u\rangle$, $|0d\rangle\rightarrow |0d\rangle$,
$|1d\rangle\rightarrow |1u\rangle$, $|1u\rangle\rightarrow
|1d\rangle$, i.e.~a CN-gate with the vibrational state as the control
bit.

In another pioneering experiment two rubidium atoms were entangled
through the exchange of a single photon in a microwave cavity.  The
experiment was performed at the Kastler Brossel Laboratory in Paris
(Phys. Rev. Lett. 7.July 1997), and is actually the first observation
of entanglement between two massive systems.  In this experiment a
Rubidium atom in the excited state $|e\rangle$ is sent through a
resonant cavity, and the time spent in the cavity is adjusted so that
there is a 50\% chance that it will emit a photon and go to the lower
state $|g\rangle$, with the photon remaining in the cavity.  Then
another Rubidium atom in the state $|g\rangle$ is sent through the
cavity, and its transit time is fixed so that it will certainly be
excited to state $|e\rangle$ if the photon is present in the cavity.
Thus, after passage of the cavity the two atoms are entangled, if the
first one is in state $|e\rangle$ the second will be in state
$|g\rangle$, and vice versa.  The probabilities should be
$P_{ge}=P_{eg}=0.5$, $P_{gg}=P_{ee}=0.0$.  Measurements gave the
results $P_{eg}=0.44$, $P_{ge}=0.27$ (loss of photon in the cavity),
$P_{gg}=0.23$, $P_{ee}=0.06$.

To make a workable quantum computer a large number af atoms have to be
entangled and kept in such an entangled or coherent state for
macroscopic times. This seems hard to achieve with present
technologies. None the less, experiments as briefly outlined above are
considered crucial steps on the road to quantum computing. Theoretical
studies are also highly required, in particular related to the loss of
coherence, i.e.  the loss of quantum information due to interaction
between the atomic system and the macroscopic environment.


\begin{enumerate}

\item John Preskill, {\em Quantum information and physics:
some future directions}; quant-ph/9904022 (1999).
\item Y. Nakamura, Yu. A. Pashkin, and  J. S. Tsai, {\sl  Coherent
control of macroscopic quantum states in
a single-Cooper-pair  box}, Nature {\bf 398}, 786 (1999).
\item Y.~M. Galperin, V.~L. Gurevich, and V.~I. Kozub, {\sl
``Disorder-induced  low-frequency   noise   in small systems: point
and tunnel contacts  in the normal and superconducting state.''},
Europhys. Lett., {\bf 10}, 753 (1989).
\item Yu. M. Galperin, {\sl `Disorder-induced flicker noise in
high-T$_c$ rf
superconducting quantum interferometers''}, Journ. Appl. Phys. {\bf
73}, 4054 (1993).
\item U. Hanke, M. Gisself\"alt, Yu. M. Galperin, M. Jonson,
R. I. Shekhter, and K. A. Chao, {\sl ``Parity-controlled
Coulomb-blockade effects identified by shot noise''}, Phys. Rev. B,
{\bf  50}, 1953 (1994).
\item U. Hanke, Y. M. Galperin and K. A. Chao, {\sl ``Optimization
of charge sensitivity of a single electron tunneling transistor with a
superconducting grain''}, J. Appl. Phys. {\bf 20}, 1245 (1996).
\item J. P. Hessling, V. S. Shumeiko, Yu. M. Galperin, and
G. Wendin,
{\sl ``Current noise in biased superconducting weak links''},
Europhys. Lett., {\bf 34}, 49 (1996).
\item N. I. Lundin, Y. M. Galperin, M. Jonson, {\sl
     ``Microwave-Activated Quantum Interferometer in an
     Environment''}, Journal of Low Temperature Physics, {\bf 118},
     579 (2000).
\item A. Imamolu, D. D. Awschalom, G. Burkard,
D. P. DiVincenzo, D. Loss, M. Sherwin, and A. Small, {\sl Quantum
Information Processing Using Quantum Dot Spins and Cavity QED},
Phys. Rev. Lett., {\bf 83},  4204 (1999).

\end{enumerate}











