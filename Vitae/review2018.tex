 \documentclass[prc,amsart,english,twocolumn,superscriptaddress,showpacs,floatfix]{revtex4}
 \usepackage[T1]{fontenc}       % DC-fonts
 \begin{document}


 \section*{Contribution from Morten Hjorth-Jensen (May 2015-May 2016}

 \subsection*{Regular Articles in Journals with a Referee System only, after April 30 2018}
\begin{enumerate}

 \end{enumerate}


 \subsection*{Books}
My first book on Computational Physics is now in press. The second book will be submitted for publication in October/November thus year and will appear in 2020.
\begin{enumerate}
\item Morten Hjorth-Jensen, \emph{Computational Physics, an introduction}, to be published by IOP in 2019. Approx 500 pages

\item Morten Hjorth-Jensen, \emph{Computational Physics, an advanced course}, to be published by IOP in 2020. Approx 400 pages

\end{enumerate}



 \subsection*{Talks at workshops, conferences and institute seminars and organization of meetings}
\begin{enumerate}


\end{enumerate}


 \subsection*{Service to the community}
\begin{itemize}


\item Editorial Board member of European Physical Journal Special Topics (2010-present)

\item Editorial Board member of Springer's Lecture Notes  in Physics, LNP,  (2010-present)

\item Editorial Board member of Springer's Undergraduate Lecture Notes in Physics, UNLP, (2014-present)

\item Editorial Board member of Springer's Undergraduate Texts in Physics, UTP,  (2014-present)

\item Editorial Board member of Springer's Graduate Texts in Physics, GTP,  (2017-present)

\item Editorial Board member of Springer's Unittexts in Physics (2017-present)


\item {Initiated and led the Nuclear Talent initiative from 2010 till 2015, now member of the Steering committee}

\item I initiated and lead the new {Master of Science program on Computational Science at the University of Oslo}. This is a new and multi-disciplinary program across several disciplines at the College of Natural Science of the University of Oslo. It includes now six departments at the University of Oslo.
\end{itemize}

\subsection*{Courses I teach}
During the last year I have been responsible for four courses, two at MSU (spring) and two at the University of Oslo (Fall semester)
\begin{itemize}
\item {FYS3150/4150 Computational Physics I}, Fall semester, senior undergraduate level (Oslo) 

\item {FYS4411 Computational Physics II: Quantum mechanical systems}, graduate level, Spring semester (Oslo) 

\item {PHYS981 Nuclear Structure}, graduate level, Spring semester (MSU) 

\item {PHY480/905 Computational Physics} (MSU), undergraduate and graduate level, Spring semester
\end{itemize}

I presently supervise 12 Master of Science students (University of Oslo) and four PhD students (MSU) either as main supervisor or co-supervisor. 

\subsection*{Lectures and organization of schools}
\begin{enumerate}

\end{enumerate}



\subsection*{Personal Summary}

I started in January 2012 at Michigan State University. I have a
shared position between the University of Oslo and Michigan State
University.

My main activity is on studies of nuclear physics systems, with an
emphasis on many-body methods for nuclear structure studies. In
particular, I wish to understand the stability of nuclear matter and
nuclei from first principle methods. This matches perfectly the
scientific mission of the NSCL and FRIB at MSU.  Several of the
articles published in the review period address these issues, amongst
these there are several articles which study
nuclei close to the lines of stability.  Moreover, our recent
optimizitation program holds great promise for the future.  We 
started this work in 2012, and the results are extremely interesting.  We
have recently optimized the nucleon-nucleon and the three-nucleon interaction from chiral
effective field theory at next-to-next- to-leading order. These results will published soon in the Physical Review C. These Hamiltonians have then been applied to many other 
nuclear systems, a recent Nature Physics publication on calcium-48 and its neutron skin.

I have also a strong educational commitment, reflected in the initiation and partecipation in a project at the University of Oslo called 'Computing in Science Education'. This project has changed totally the way we teach science, with computations being introduced at first semester of study. Many of the seminars I have given at various US institutions deal with the integration of a computational approach to the basic science courses. Locally at MSU I am involved in teaching Computational Physics and have continuous discussions with several colleagues at the NSCL, the Department of Physics and Astronomy and the new Department of Computational Mathematics, Science and Engineering on computational issues and education in computational aspects. Much inspired by the developments at MSU, I have initiated and chair a Master of Science program in Computational Science at the University of Oslo, Norway. I spend the period July-December in Norway. 

I am also involved in the so-called Nuclear Talent initiative,  for more details. Last year I organized two of the courses and taught as well at one of the courses for three weeks. Next year with Alex Brown and Alexandra Gade, we are planning a Nuclear Talent course on Nuclear Theory for Nuclear structure experiments, most likely to be held at the ECT* in Italy. Last year I gave also lectures at the National Nuclear Physics summer school at Lake Tahoe in california. I taught this year also an intensive course on Computational Physics at the university of Tunis in Tunisia, with roughly 40 students attending. Last year I taught a week at the Doctoral training program of the ECT* in Italy. 

Locally I have extensive collaborations with fellow theorists Scott Bogner, Alex Brown, Heiko Hergert, Witek Nazarewicz and partly with Filomena Nunes, as well as tight collaborations with our graduate students and post-doctoral fellows. The Theory trailer is a lively community, with an average age well below 35. I collaborate also 
with many experimentalists at the lab. It is fun to be here!

Finally, my long term goal is to be able to contribute to build up a
strong activity on the nuclear many-body problem at MSU, an activity
which will match the experimental program at the NSCL and FRIB. With
Scott Bogner and Heiko Hergert we have now a group of several excellent graduate students. 


 \end{document}

1) Nuclear Talent course on Many-body methods for nuclear physics, from Structure to Reactions at Henan Normal University, P.R. China, July 16-August 5 2018. 
Main Organizers: 
o Morten Hjorth-Jensen, MSU
o Chun-Wang Ma at Henan Normal University, Xinxiang, Henan 453007, P.R. China
o ``Furong Xu at School of Physics, Peking University, Beijing 100871, P.R. China
o Shan-Gui Zhou at  the Institute of Theoretical Physics, Chinese Academy of Sciences, Beijing 100864, P.R. China

I gave the following lectures
Monday July 16, 830-1230, 4 hours
Second quantization  and Hamiltonians
Tuesday July 17, 9-11  2 hours
Full configuration interaction theory
Wednesday July 18, 9-11, 2 hours
Full configuration interaction theory and the pairing model
Thursday July 19, 9-12, 3 hours
Full configuration interaction theory        |
Hartree-Fock theory and links to Coupled Cluster theory

In addition I run exercise sessions in the afternoons after lunch
For the full program, see https://nucleartalent.github.io/ManyBody2018/doc/pub/program/html/program.html


2) Applied Data Analysis and Machine Learning, a Course given at Ganil, Caen, France, January 19-31 2019
I gave the following lectures (see https://compphysics.github.io/MLErasmus/doc/pub/Intro2Course/html/Intro2Course.html)
(I had two TAs in the afternoons, acronyms BS and KBH here).
  Monday 21 9am-945am  Introduction and welcome
           10am-1045am Review of Python and Linear Algebra
            1115am-12pm   Getting started with Linear Regression
           2pm-6pm     Python installations and setups, anaconda and more   Exercise set 1 (BS,KBH, MHJ) 
 -------------------------------------------------------------------------- 
  Tuesday 22   9am-945am    Linear Regression
           10am-1045am   Linear Regression, Lasso and Ridge
            1115am-12pm   Linear Regression, Lasso and Ridge
           2pm-6pm     Brief intro to _scikit-learn_   Exercise set 2 (BS,KBH, MHJ)  
 -------------------------------------------------------------------------- 
  Wednesday 23   9am-945am    Summary of linear regression
           10am-1045am   Statistical analysis of data, bias and variance
            1115am-12pm   Statistical analysis of data, bias and variance
           2pm-6pm     More _scikit-learn_ functionality    Project 1 (BS,KBH, MHJ)  
 -------------------------------------------------------------------------- 
  Thursday 24   9am-945am    Statistical analysis, cross-validation and Bootstrap
           10am-1045am   Statistical analysis, cross-validation and Bootstrap
            1115am-12pm   Optimization and gradient descent
           2pm-6pm     Statistics tools   Project 1 (BS,KBH, MHJ)  
 -------------------------------------------------------------------------- 
  Friday 25   9am-945am    Optimization and Gradient descent
           10am-1045am   Logistic Regression
            1115am-12pm   Logistic regression and classification and start neural networks
           2pm-6pm     Gradient descent coding   Project 1 (BS,KBH, MHJ)  
 -------------------------------------------------------------------------- 
  Saturday 26   9am-945am    Neural Networks
           10am-1045am   Neural Networks and the back propagation algo
            1115am-12pm   Neural Networks and the Back propagation algo
           2pm-3pm     Only discussion of project 1   Project 1 (BS,KBH, MHJ)  
 -------------------------------------------------------------------------- 
  Monday 28   9am-945am    Neural networks, setting up code for back propagation
           10am-1045am   Neural networks,examples
            1115am-12pm   Neural networks, examples and convolutional NN, setting up your code
           2pm-6pm     Introduction to Tensorflow and Keras examples    Projects 1 and  2 (BS,KBH, MHJ) 
 -------------------------------------------------------------------------- 
  Tuesday 29   9am-945am    Solving differential equations with Neural Networks
           10am-1045am   Boltzmann machines and the the many-body problem
            1115am-12pm   Boltzmann machines and the many-body problem
           2pm-6pm     Tensorflow/Keras and CNNs    Projects 1 and  2 (BS,KBH, MHJ)  
 -------------------------------------------------------------------------- 
  Wednesday 30   9am-945am    Decision Trees and Bagging
           10am-1045am   Bagging, Ensembles  and Random Forests
            1115am-12pm      Support Vector Machines
           2pm-6pm     Continuation of SVM lectures and work on projects   Projects 1 and  2 (BS,KBH, MHJ)  
  Thursday 31   930am-1230am   Work on Projects 1 and 2 
           2pm-6pm     Work on projects   Projects 1 and  2 


3) FRIB-TA Summer school on Machine Learning for Nuclear Physics Experiment and Theory, Michigan State University, May 20-23, 2019 (with Matthew Hirn, MSU, Michelle Kuchera and Raghuram Ramanujan, Davidson College)

I gave the following lectures
Monday May 20
830am-930am: Introduction to Machine Learning and various Python packages
930am-1030am: Linear Regression
11am-12pm: Logistic Regression 
1pm-2pm: Optimization of functions, gradient descent and stochastic gradient descent
2pm-3pm: Decision Trees and Random Forests 
330pm-6pm: Hands-on sessions with selected Physics examples
Wednesday May 22

2pm-3pm: Boltzmann machines and many-body problems (MHJ)

1) Marcos Daniel Caballero and Morten Hjorth-Jensen, Integrating a Computational Perspective in Physics Courses, in New Trends in Physics Education Research, Editor Salvatore Magazu, Nova Publishers, New York, 2018, pages 47-76
 
2) D. A. Torres, R. Chapman, V. Kumar, B. Hadinia, A. Hodsdon, M. Labihe, X. Liang, D. O'Donnell, J. Ollier, R. Orlandi, J. F. Smith,, K.-M. Spohr, P. Wady, Z. M. Wang, L. Corradi, E. Fioretto, A. Gadea, G. de Angelis, N. M rginean, D. R. Napoli, E. Sahin, A. M. Stefanini, J. J. Valiente-Dobón, F. D. Vedova,
M. Axiotis, T. Martinez, S. Szilner, D. Bazza, S. Beghini, E. Farnea, R. Marginean, D. Mengoni, G.
Montagnoli, F. Recchia, F. Sarlassara, C. A. Ur, S. M. Lenzi, S. Lunardi, T. Kröll, F. Haas, T. Faul,
M. Hjorth-Jensen, B. G. Carlsson, S. J. Freeman, A. G. Smith, G. Jones, N. Thompson, G. Pollarolo,
and G. S. Simpson, Study of Medium-Spin States of Neutron-rich 87,89,91 Rb isotopes, European Journal of Physics A, 2019, in press
 
 
Workshops/meetings/conferences/schools
 
1) Organizer and main teacher of Nuclear Talent course on  Many-body methods for nuclear physics, from Structure to Reactions at Henan Normal University, Xinxiang, P.R. China, July 16-August 5 2018 (with Thomas Papenbrock, UTK/ORNL, Kevin Fossez (MSU) and Ragnar Stroberg, INT, Washington University), in total 15 hours of lectures and 15 hours of exercises and project work
2)  Organizer and main teacher of a course on  Data Analysis and Machine Learning, Ganil, Caen, France, January 19-31, 2019, in total 30 one hour lectures plus 30 hours of computational lab
3) Organizer and teacher of FRIB-TA summer school on Machine Learning for Nuclear Physics Experiment and Theory, Michigan State University, May 20-23, 2019 (with Matthew Hirn, MSU, Michelle Kuchera and Raghuram Ramanujan, Davidson College). In total 6 hours of lectures and 12 hours of computational lab.
4) Morten Hjorth-Jensen,  Solving Quantum Mechanical Many-body Problems with Machine Learning Algorithms, seminar at The Institute of Nuclear and Particle Physics, University of Ohio, Athens, Tuesday, April 16, 2019
5) Morten Hjorth-Jensen,  Solving Quantum Mechanical Many-body Problems with Machine Learning Algorithms, seminar at Department og Physics, University of Trento, Trento, Italy, Tuesday, March  12, 2019 
6) Morten Hjorth-Jensen, Integrating a Computational Perspective in Physics (and Science) Courses, seminar at Department og Physics, University of Trento, Trento, Italy, Thursday, March  7, 2019 
7) Morten Hjorth-Jensen, Education and research in computational science and data science; from bachelor programs to research, seminar at Oslo Metropolitan University, Oslo, Norway, October 22, 2018
8) Morten Hjorth-Jensen, Hylleraas seminar on Many-body methods, from variational ones to machine learning approaches, Department of Chemistry, University of Oslo, Oslo, Norway, December 14, 2018.
9) Morten Hjorth-Jensen, Many-Body Methods, from Variational ones to Machine Learning Approaches, talk at FRIB-TA Topical Program: Connecting bound state calculations with the scattering and reaction theory, June 11-22, 2019, Michigan State University
