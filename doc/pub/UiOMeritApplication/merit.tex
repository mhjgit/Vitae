\documentclass[aps,floatfix,preprint]{revtex4-1}


\usepackage{graphicx}
\usepackage{comment}
\usepackage{lipsum}
\usepackage[utf8]{inputenc}
\usepackage[toc,page]{appendix}
\usepackage{braket}
\usepackage{amsmath,amssymb}
\usepackage{amsthm}
\usepackage{hyperref}
\usepackage{pgfplots}
\usepackage{bbold}
\usepackage{physics}
\usepackage{csquotes}
\usepackage{commath}
\usetikzlibrary{external}
\tikzexternalize[prefix=figures/]



\usepackage[T1]{fontenc}
\usepackage{ucs}

\usepackage{lmodern}         % Latin Modern fonts derived from Computer Modern

% Hyperlinks in PDF:
\definecolor{linkcolor}{rgb}{0,0,0.4}
\usepackage{hyperref}
\hypersetup{
    breaklinks=true,
    colorlinks=true,
    linkcolor=linkcolor,
    urlcolor=linkcolor,
    citecolor=black,
    filecolor=black,
    %filecolor=blue,
    pdfmenubar=true,
    pdftoolbar=true,
    bookmarksdepth=3   % Uncomment (and tweak) for PDF bookmarks with more levels than the TOC
    }
%\hyperbaseurl{}   % hyperlinks are relative to this root

\setcounter{tocdepth}{2}  % levels in table of contents

% prevent orhpans and widows
\clubpenalty = 10000
\widowpenalty = 10000



\begin{document}

\title{Søknad om Meritteringsordning for utdanningsfaglig kompetanse ved Universitetet i Oslo}

\author{Morten Hjorth-Jensen}
\email{mhjensen@uio.no}
\affiliation{Department of Physics and Center for Computing in Science Education, University of Oslo, N-0316 Oslo, Norway}
\affiliation{Department of Physics and Astronomy and National Superconducting Cyclotron Laboratory and Facility for Rare Ion Beams, Michigan State University, East Lansing, MI 48824, USA}

\date{May 15}


\maketitle




\section*{Kort introduksjon}

Denne søknaden inneholder først en kort oversikt om undertegnede  med bakgrunn og historikk.
Deretter følger sjølve søknadsteksten og til slutt har jeg lagt ved den pedagogiske mappa. 


\section*{Bakgrunnsinformasjon}

Jeg har vært ansatt ved Fysisk Institutt ved Universitet i Oslo siden
januar 1999, først som førsteamanuensis og deretter som professor fra
mai 2001.

Fra og med januar 2012 har jeg delt tida mi mellom Michigan
State University (MSU) og UiO. Jeg har et professorat i fysikk begge
steder og tilbringer tida januar-juni i USA og juli-desember i
Norge. Jeg har en redusert stilling ved UiO. Jeg underviser på alle
nivå begge steder og vegleder laveregradsstudenter, masterstudenter,
PhD studenter og Post-docs både i Norge og i USA heile året.

Jeg er utdanna Sivilingeniør fra NTNU i Trondheim i 1988 og disputerte
for PhD-graden i desember 1993 ved UiO.

I tida januar 1994 til desember 1998 var jeg post-doc ved European
Center for Theoretical Studies in Nuclear Physics (Trento, Italia,
1994-1996) og deretter Nordita (København, Danmark, 1996-1998).

Jeg er innvalgt medlem av \textbf{Det Norske Videnskaps-Akademi} og \textbf{Det Kongelige Norske Videnskabers Selskab}.

Mitt engasjement i undervisning, undervisningsforskning og utvikling
av aktive læringsmiljø, studiemiljø og studieprogrammer har resultert
i flere utdanningspriser i Norge og USA. Mye av dette blir diskutert i sjølve søknaden også.

\subsection*{Utdanningsspriser}
\begin{enumerate}
\item \href{{http://www.uniforum.uio.no/nyheter/2000/11/det-viktigste-er-aa-inspirere.html}}{UiOs utdanningspris i 2000},  (250kNOK)

\item \href{{http://www.uniforum.uio.no/nyheter/2011/08/undervisning-for-framtidig-forsking.html}}{UiOs utdanningspris i 2011 for utvikling of Computing in Science Education prosjektet}  (250kNOK)

\item \href{{http://www.uniforum.uio.no/nyheter/2012/04/uio-tok-andreplass-i-utdanningskvalitet.html}}{NOKUTs pris for fremragende utdanning 2012} for  \textbf{Computing in Science Education} prosjektet

\item \href{{http://www.uniforum.uio.no/nyheter/2015/10/instituttet-som-lofter-fram-gode-forelesere.html}}{UiOs utdanningspris 2015} for utvikling av studie og læringsmiljøet i Computational Physics ved Fysisk Institutt (250kNOK)

\item \href{{https://pa.msu.edu/news-events/news/2016-departmental-awards/}}{Favorite Graduate Teacher Award at Department of Physics  and Astronomy, Michigan State University, 2016} 

\item \href{{https://www.ntbinfo.no/pressemelding/olav-thon-stiftelsen-annonserte-arets-priser-42-millioner-til-forskning-og-undervisning?publisherId=8983491&releaseId=16475069}}{Olav Thon Stiftelsen, Nasjonal utdanningspris 2018}

\item \href{{https://web.pa.msu.edu/alumni/awards/osgood_fac_awards.html}}{Thomas H. Osgood Faculty Teaching award at Michigan State University, 2018}
\end{enumerate}

\noindent
Jeg har siden jeg blei ansatt ved UiO i 1999 og ved Michigan State
University i 2012, vegleda og vegleder over 100 Master, PhD studenter og post-docs. Over 50\%
av masterstudentene har fortsatt med PhD-studier enten med meg som
vegleder eller andre i Norge og internasjonalt.

En fullstendig CV finnes på
\href{{http://mhjgit.github.io/info/doc/pub/cv/html/cv.html}}{\nolinkurl{http://mhjgit.github.io/info/doc/pub/cv/html/cv.html}}.





\section*{Søknaden: Beskrivelse av undervisningsarbeid, utvikling og forskning}

I de 20 åra jeg har vært ansatt ved UiO og siden 2012 ved MSU,
har jeg hatt og har et stort fokus på utdanning, utvikling
av læringsmiljøer, nye undervisningsformer, innovasjon i utdanning,
samt utvikling av utdanningsprogrammer i henhold til samfunnets
kompetansekrav og spesifikke behov i ulike fagfelt.

Jeg har utvikla flere nye kurs, samt starta nye studieprogram og
bidratt til å utvikle internasjonale kurstilbud i eget forskningsfelt,
i tett samarbeid med kolleger på tvers av disipliner nasjonalt og
internasjonalt.


\subsection*{Overordna målsettinger ved mitt utdanningsarbeid}

Det er ti overordna målsettinger ved mitt undervisningsarbeid ved UiO og MSU:

\begin{enumerate}
\item Gi studentene ei forståing av den vitenskapelige metoden så tidlig som mulig i studieløpet. Dette lar seg gjøre spesielt gjennom prosjektorientert undervisning med tett oppfølging og tilbakemelding til studentene. Her spiller mitt initiav til å starte Computing in Science Education prosjektet en sentral rolle. Dette er beskrevet nedenfor i større detalj.  

\item Gi studentene kompetanse, faglig trygghet  og innsikt som kreves for å løse naturvitenskapelige og teknologiske problem for det 21ste århundre, spesielt med tanke på digital kompetanse. Her har min rolle som programrådsleder i to store studieprogram vært, og er, svært viktig. 

\item Gi studentene en god etisk holdning til deres arbeid, samt å utvikle kritisk tenkende mennesker med djup innsikt i alle sider av den vitenskapelige prosessen.

\item Å utvikle gode utdanningsprogrammer og faglig progresjon i studieløpa, i tett samarbeid med kolleger ved Fysisk institutt (både i Norge og i USA) og  på tvers av disipliner.

\item Å sørge for at det faglige innholdet er i tråd med universitetenes samfunnsoppgave, ved å utdanne svært ettertraktede kandidater til forskning, utdanning, offentlig og privat sektor.

\item Å sørge for at det faglige innholdet har intellektuelle utfordringer og innhold i tråd med nåtidige og framtidige faglige behov og forskningsretninger. 

\item Å utvikle ei kunnskapsbasert tilnærning til læring, samt utvikle forskningsprogram om hva som gir studentene økt innsikt og faglig forståelse. 

\item Å utvikle et internasjonalt perspektiv på utdanninga vår.

\item Kunnskapen skal være fritt tilgjengelig for alle.

\item Pedagogiske tiltak skal være forskningsbaserte og/eller initiere ny forskning.
\end{enumerate}


Disse ti overordna målsettingene er beskrevet i større detalj nedenfor. Flere av disse overordna
målsettingene inngår i beskrivelsen av tiltak jeg har initiert,
samt kurs og utdanningsmiljø jeg har utvikla.


\subsection*{Faglig fornying av utdanning og arbeid med studieprogram}

For å oppnå måla om digital kompetanse, prosjektorientert utdanning,
tett oppfølging av studenter og mange flere av de ti måla ovafor var
jeg i 2003 en av de sentrale initiativtakerne til prosjektet \textbf{Computing in Science
Education} (CSE), sammen med kolleger på Fysisk Institutt (Arnt Inge Vistnes), Matematisk Institutt (John Grue) og Institutt for Informatikk (Hans Petter Langtangen og Knut Mørken). Mye av dette skjedde via
senteret for fremragende forskning (SFF) \textbf{Center for Mathematics for
Applications} og i min rolle som programrådsleder for Bachelor programmet Fysikk,
Astronomi og Meteorologi (FAM). De første midlene som blei tildelt til CSE
prosjektet var via UiO tiltaket \textbf{Fleksibel læring} i 2004. Undertegnede var
prosjektleder sammen med programrådsleder for MIT programmet John
Grue. Jeg satt også som programrådsmedlem i MIT styret og var med å koordinere innføringa av CSE prosjektet.
Senteret CMA spilte en sental rolle i begynnelsen og finasierte deler av stillinger også. Seinere kom også et anna 
SFF inn på banen, Physics of Geological Processes (PGP) og Anders Malthe-Sørenssen ved Fysisk Institutt. Anders var sentral i implemeteringen av CSE prosjektet i det første fysikkurset FYS-MEK1100.

Jeg var programrådsleder for FAM i tida 2002-2011 og var drifkraft
og overordna ansvarlig for integrering av et beregningsperspektiv i
FAM programmet. Dette har resultert i at fysikkfaget ved UiO har
utdanna studenter med de digitale kompetansene som trengs for å møte
de vitenskapelige og teknologiske utfordringene i det 21ste
århundre. CSE prosjektet har fungert som inspirasjon for alle andre
studieprogram ved MatNat-fakultetet og andre fakultet ved UiO og
universitet nasjonalt.

CSE prosjeket er nå gjenspeilt i omtrent alle studieprogram ved
MatNat-fakultetet ved UiO. I tillegg har jeg leda en komite ved fysisk
institutt på Michigan State University (2018) om innføring av CSE-liknende tiltak
ved dette universitetet. I vår underviste jeg et sentralt laveregrads kurs i
klassisk mekanikk med full CSE implementering ved Michigan State
University, til  positiv respons fra studentene, se fagevalueringa
i den pedagogiske mappa.

CSE prosjektet fikk UiO sin utdanningspris i 2011, samt NOKUT sin
utdanningspris i 2012. Prosjektet resulterte i at UiO fikk et
senter for fremragende utdanning i 2016, senteret for \textbf{Computing in
Science Education} (CCSE). Senteret har blitt en stor suksess og har
utvikla flere nye forskningsprosjekt om utdanning. Her er jeg involvert i
prosjekt om kvantitativ utdanningsforksning samt kursutvikling.

Som et ledd i å styrke UiO sin utdanning i digital kompetanse,
videreføre CSE prosjektet, samt utvikle masterprogrammer i beregninger
og databehandling, tok jeg, sammen med kolleager fra Fysisk Institutt (Anders Malthe-Sørenssen), Institutt for Biovitenskap (Marianne Fyhn), Institutt for Matematikk (Knut Mørken) og Institutt for Informatikk (Hans Petter Langtangen) i mai  2015 initiativ til å etablere det nye
Masterprogrammet \href{{https://www.uio.no/english/studies/programmes/computational-science-master/index.html}}{Computational
Science}
ved UiO. Programmet starta med det første kullet høsten 2018, og har
hittil vært en stor suksess, og et viktig satsningsfelt for
MatNat-fakultetet ved UiO. Jeg er programrådsleder for dette
programmet og har jobba intenst med å utvikle og integrere kunnskap og
kurs i beregningsorienterte fag. Jeg har sjøl utvikla et svært så
populært kurs i maskinlæring (FYS-STK4155) for dette
studieprogrammet.

Dette studieprogrammet er et samarbeid med alle institutt på
MatNat-fakultetet unntatt Farmasi. CS-programmet har 10
studieretninger fordelt på sju institutt (ITA, Biovitenskap, Fysikk,
Geovitenskap, Informatikk, Kjemi og Matematikk). Etter to års
forarbeid (høst 2015-vår 2018) klarte vi å utvikle et faglig spennende
studieprogram. Kandidatene har ofte jobbtilbud et år før de er ferdige
med sin utdanning.  Jeg har leda arbeidet på tvers av instituttgrenser og disiplinære grenser.


CS-programmet var også fra min side det første strategiske trinnet i å
utvikle et heilhetlig tilbud til våre studenter i det som på engelsk
kalles Computational Science and Data Science, på tvers av
disipliner. Det neste steget var lanseringa av et nytt institutt i
Computational Science og Data Science våren 2018, se materialet med
white paper på
\href{{https://computationalscienceuio.github.io/CCAD/doc/pub/whitepaper/html/whitepaper-bs.html}}{\nolinkurl{https://computationalscienceuio.github.io/CCAD/doc/pub/whitepaper/html/whitepaper-bs.html}}.
Mange av de opprinnelige initiativtakerne for CS-programmet var også sterkt
delaktige i å utvikle grunnlagsmaterialet som seinere (høst 2018-vår 2019) resulterte i ei innstilling
om et nytt senter i Data Science og Computational Science ved MatNat-fakultetet ved UiO.
Senteret antas å ha oppstart i 2021 og vil spille ei viktig rolle i utdanningstiltak som fokuserer på digital kompetanse for heile universitetet. I tillegg planlegges det flere etterutdanningstiltak retta mot både privat og offentlig sektor.

Om dette lykkes, vil det bety ei videreføring og videreutvikling av
CSE prosjektet til å dekke alle utdanningstrinn ved UiO. Å utvikle
ansattes og studentenes digitale kompetanse er et sentralt element i
vår faglige utvikling og vårt arbeid.




\subsection*{Utvikling av studiemiljø}

Siden 2000 har jeg aktivt jobba for å utvikle et utdanningsmiljø i
Computational Physics på mastergradsnivå.  Fra og med  2007 har dette arbeidet vært
gjort sammen med en nær kollega, Anders Malthe-Sørenssen, som også er
direktør for CCSE. I 2015 blei dette arbeidet tilkjent UiO sin
utdanningspris, se
\href{{https://www.uniforum.uio.no/nyheter/2015/10/instituttet-som-lofter-fram-gode-forelesere.html}}{\nolinkurl{https://www.uniforum.uio.no/nyheter/2015/10/instituttet-som-lofter-fram-gode-forelesere.html}}.

Vi har spesielt vektlagt
\begin{enumerate}
\item Utvikle et godt sosialt miljø hvor deling av resultat og programvare  står sterkt. Studentene definerer ofte sine egen prosjekt for masteroppgavene sine.

\item Oppbygging av lokaler som er imøtekommende og inkluderende. Bygd opp mange arenaer for sosialt samvær.

\item Studentene integreres tidlig i forskninga og studentene fungerer som læremestre for hverandre.

\item Studentene engasjeres tidlig i å utvikle læringsmateriale, spesielt som gruppelærere i kurs hvor beregninger (CSE prosjektet) er viktig.
\end{enumerate}

\noindent
Med etableringen av det nye CS-masterprogrammet har vi nå utvikla nye
tiltak for å forbedre studiekvaliteten, med blant annet tett
integrering av studentene i utforming av nye oppgaver og prosjekter til
mastergradskursene, tett kopling mellom studenter og fagmiljø, samt
mentorprogram i oppstart av masterprogrammet, med vekt på mulige fagvalg og
karriereveger. Individuell oppfølging av studentene spiller en sentral
rolle, spesielt også for studenter som sliter mentalt.  Tett
oppfølging og individuell tilrettelegging er et sentralt aspekt som
vektlegges i utviklinga av et godt studiemiljø.

Studieprogrammet har sosiale spillkvelder med faglige seminar annen
hver fredag. Her møter studentene  forskere og/eller potensielle
oppdragsgivere fra privat og offentlig sektor. Og i en del tilfeller
har det resultert i sommerjobber og kanskje fast ansettelse seinere.

Computational Physics miljøet har siden 2003 utdanna over 100 mastergradsstudenter, og veldig mange av disse (over 50\%) har fortsatt med PhD studier. Det er tett samarbeid mellom bachelor studenter, mastergrads studenter, PhD studenter og Post-docs.  


\subsection*{Utvikling av kurs med prosjektbasert innhold}

Over to tiår har jeg utvikla kurs med en prosjektbasert profil.

Da jeg underviste Kvantefysikk FYS2140 i perioden 1999-2004
introduserte jeg numeriske prosjekt som studentene jobba med. På den
tida var dette ganske nytt ved UiO og resulterte blant annet i UiO
sin utdanningspris i 2000, etter bare litt over ett år som ansatt ved
UiO. Mye av dette arbeidet la grunnlaget for ideer og tiltak rundt CSE
prosjektet i 2003. Numeriske prosjekt tillater studenter å utdjupe sin
faglige innsikt på et heilt anna vis enn gjennom tradisjonelle papir
og blyant oppgaver. En har en stor mulighet til å teste på et djupere
nivå studentenes innsikt i et fysisk fenomen og er et ypperlig
pedagogisk verktøy for å utvikle studentenes innsikt og forståelse av
den vitenskapelige metoden.  Med programmering har en også mulighet
til å bringe inn mer realistiske problemstillinger på et tidligere
stadium av utdanninga. Ofte møter ikke studentene forskninsgrelatert
utdanning før de begynner på sine masterprosjekter. Med et
prosjektbasert løp spiller tilbakemeldinger på arbeidene deres og tett
kontakt med faglærere en sentral rolle. Tilbakemeldingene fra
studentene er også avgjørende for forbedring av læringsmål og
utdanningsmateriale.


Prosjektbasert undervisning er også sentralt i kursene jeg har utvikla fra scratch, FYS3150
Computational Physics I, FYS4411 Computational Physics II og
FYS-STK4155 Applied Data Analysis and Machine Learning som jeg
underviser ved UiO. FYS3150 og FYS-STK4155 er kurs med over 100
studenter hver og studentene jobber med 3-5 prosjekter gjennom heile
semesteret. Prosjekta er lagt opp som vitenskapelige arbeider og for
det siste prosjektet kan studentene ofte definere tema sjøl. Dette gir
studentene en stor frihet i å utforske egne veger og kople
undervisninga opp til eventuell forskning.  Omfattende
tilbakemeldinger på prosjekta med tanke på evalueringa spiller ei
sentral rolle og er en tidkrevende, men en viktig del av disse
kursa. Disse kursa er faktisk noen av de få ved UiO hvor
studentene får lange tilbakemeldinger med begrunnelse for endelig
karakter.


Alt utdanningsmateriale, forelesningsnotater, programmer, kildekode og
mye mer er fritt tilgjengelig for studentene. Studentene kan feks bruke
kildekoden til notata til å lage sine egne elektroniske notatbøker
(ofte i form av en såkalla jupyter-notebook). Det styrker den
ovennevnte delingskulturen og fungerer som et godt eksempel for
studentene om deling og samarbeid om vitenskapelige resultat. I
tillegg gir moderne versjonskontroll programvare en unik mulighet til
å fokusere på utvalgte aspekt av vitenskapelig etikk, som
reproduserbarhet av vitenskapelige resultat og korrekt behandling av
kildemateriale.  Eksempler på hvordan materiale er gjort fritt
tilgjengelig kan ses på min GitHub adresse
\href{{https://github.com/mhjensen}}{\nolinkurl{https://github.com/mhjensen}}. Her ligger undervisningsmateriale
for flere kurs, og ved hjelp av versjonskontroll programvaren \textbf{git}
kan alt materiale lastes ned med enkle tastetrykk.

At alt utdanningsmateriale er fritt tilgjengelig forenkler også
læringsprosessen for studentene.

Det neste utvklingstrinnet i min undervisning er å utvikle et miljø
for ei såkalla \textbf{Flipped Classroom} tilnærming. Dette blir også gjort
som et mulig tiltak i anledning et eventuelt fullt eller delvis digitalt  undervisningsløp høsten 2020.  Dette
blir implementert til høsten 2020 for kursene FYS3150 og FYS-STK4155
og innebærer ny produksjon av visuelt materiale i form av videoer som
er kopla opp mot allerede eksisterende digitalt materiale. Studentene
foventes å gå gjennom ukentlig oppgitt materiale før de eventuelt møter i
mindre grupper for å diskutere materialet og jobbe med prosjekter og
oppgaver. Det gir oss som faglærere en mulighet til enda tettere
kopling opp mot studentene for å følge deres læring. Flipped Classroom er et spennende pedagogisk område, med mange interessante forskningsbaserte resultat, blant anna med tanke på økt læringsutbytte for studentene.  


Jeg har også utvikla en sterk prosjektbasert profil i kurset Classical
Mechanics PHY 321 som jeg underviser ved Michigan State University,
med veldig positive tilbakemeldinger fra studentene (se kursevaluering
i pedagogisk mappe).


Til slutt har jeg laga et nytt forslag til første studieår i fysikk
som åpner for en ny integrering av teori, eksperiment og
beregninger. Dette er ganske nytt og banebrytende og danner også
grunnlag for spennende forskningsprosjekter, spesielt med tanke på den overordna forståelsen av den vitenskapelige prosessen. Forslaget er beskrevet på
\href{{https://mhjensen.github.io/FirstYearPhysicsUiO/doc/pub/proposal/html/proposal-bs.html}}{\nolinkurl{https://mhjensen.github.io/FirstYearPhysicsUiO/doc/pub/proposal/html/proposal-bs.html}}.

Dette forslaget føyer seg inn i Fysisk Institutt sin strategiske
fornyelse av bachelor programmet i Fysikk og Astronomi. En svakhet i
dag er at det eksperimentelle elementet er mindre framtredende enn ved
andre universitet. Og fysikk er et eksperimentelt fag. Med moderne
programmeringsspråk, samt hardware som tillater en å gjøre mange
eksperiment (feks aksellerometer i mobiltelefon) kan en integrere
beregninger, eksperiment og teori på et heilt anna vis enn
tidligere. Studentene kan gjøre mange av eksperimenta med feks
mobiltelefonene sine, nesten hvor som helst. Dette åpner opp for en tydeligere integrering av
alle steg i et vitenskapelig studie og kan implementeres allerede fra
første semester.

\subsection*{Internasjonale tiltak}
Sammen med kolleger fra flere land starta jeg og etablerte et
internasjonalt initiativ i 2010 kalt \textbf{Nuclear TALENT (Training in
Advanced Low-Energy Nuclear Theory)} hvis mål er å styrke
en faglig bredde i kjernefysikk internasjonalt. Mange universitet har
ikke nok vitenskapelig personale til å gi studentene på master og PhD
nivå den nødvendige faglige bredde i feltet. Nuclear TALENTs mål
er å tilby denne faglige bredden i form av et titalls avanserte kurs
som undervises på et intensivt vis over tre uker ulike steder i verden
(Nord-Amerika, Europa og Asia). Siden sommeren 2012 har vi organisert
over 15 slike kurs og jeg har undervist og organisert 5 av disse
kursene og organisert tre andre. Dette tiltaket har vært en enorm
suksess med over 500 deltakere totalt siden 2012. Pga COVID-19 er alle
tre kurs i år utsatt til 2021, men kurset jeg har ansvaret for i år om maskinlæring anvendt på kjernefysikk tilbys
digitalt i juni-juli 2020, se URL:''http://www.ectstar.eu/node/4472''.

For mer informasjon om Nuclear TALENT, se \href{{https://fribtheoryalliance.org/TALENT/}}{\nolinkurl{https://fribtheoryalliance.org/TALENT/}}.

Ellers leder jeg et større INTPART prosjekt om Computing in Science
Education mellom CCSE ved UiO, Michigan State University, Oregon State
University og University of Colorado ved Boulder. Et viktig mål med
dette prosjektet er å utvikle et program i kvantitativ utdanningsforskning. Vår
første workshop om dette blei  dessvere avlyst i år pga COVID-19 situasjonen.

Jeg har også etablert flere internasjonale utvekslingsprogrammer for
studenter i Oslo, spesielt mot USA og Europa.

\subsection*{Utdanningsforskning}

CCSE senteret spiller en avgjørende rolle i forskning rundt
beregninger (Computing in Science Education) i utdanning. Jeg er
spesielt interessert i forskning rundt faglig innsikt og studentenes
innsikt om den vitenskapelige metoden. Et av målene er å kunne utvikle
et forskningsbasert program i kvantitativ utdanningsforskning.

Her spiller feks maskinlæring en viktig rolle og sammen med PhD
student John Aiken, Prof Danny Caballero fra Michigan State University
og andre kolleger hat vi nå utvikla flere prosjekt for å nå disse
måla.  Å kunne gi kvantitave og kvalitative mål på hva som virker er
sentralt for mange utdanningstiltak.


\newpage

\section*{Pedagogisk mappe}

\subsection*{Utdanningsverv og redaktøransvar for lærebøker i fysikk}
\begin{itemize}
\item 2002-2011: Programrådsleder i bachelorprogrammet Fysikk, Astronomi og Meteorologi, et samarbeid mellom tre institutt. I tillegg til jobben som programrdsleder hadde jeg det overordna ansvaret for den faglige helheten samt innføring av Computing in Science Education prosjektet.
\item 2003-2006: Styremedlem i programrådet Matematikk, Informatikk og Teknologi
\item 2003-nå:  Var med å starte Computing in Science Education prosjektet
\item 2010-nå: Initierte Nuclear Talent prosjektet med kolleger fra Nord-Amerika, Europa og Asia. Leda prosjektet fra 2010 til 2015. Styremedlem 2016-2020. Har undervist og organisert flere tre-ukers avanserte intensive kurs. Se lista lenger ned over kurs jeg har organisert og undervist.
\item 2015: Tok initiativ til, sammen med kolleger fra Fysisk Institutt, Institutt for Biovitenskap, Matematisk Institutt og Institutt for Informatikk for å etablere det nye masterprogrammet Computational Science. Er programrådsleder siden oppstart 2017 og har jobba med kolleger fra sju institutt ved MatNat-fakultetet for å skape et godt faglig program.
\end{itemize}

I tillegg har jeg viktige verv i utvikling av faglig litteratur i fysikk for Springer. Jeg er medredaktør i fem bokserier og flere kolleger ved UiO har fått publisert sine bøker via Springer.

\begin{itemize}

\item Editorial Board member of Springer's Lecture Notes in Physics, 2010-present

\item Editorial Board member of Springer's Undergraduate Lecture Notes in Physics, 2014-present

\item Editorial Board member of Springer's University Texts  in Physics, 2015-present

\item Editorial Board member of Springer's Undergraduate Texts  in Physics, 2016-present

\item Editorial Board member of Springer's Graduate Texts  in Physics, 2018-present

\end{itemize}


\subsection*{Overordna målsettinger med utdanningsaktiviteten min}

De ti overordna målsettingene med min utdanningsaktivitet og forskning er gjengitt her:

\begin{enumerate}
\item Gi studentene ei forståing av den vitenskapelige metoden så tidlig som mulig i studieløpet. Dette lar seg gjøre spesielt gjennom prosjektorientert undervisning med tett oppfølging og tilbakemelding til studentene. Her spiller mitt initiav til å starte Computing in Science Education prosjektet ei sentral rolle. Dette er beskrevet nedenfor i større detalj.  

\item Gi studentene kompetanse, faglig trygghet  og innsikt som kreves for å løse naturvitenskapelige og teknologiske problem for det 21ste århundre, spesielt med tanke på digital kompetanse. Her har min rolle som programrådsleder i to store studieprogram vært, og er, svært viktig. 

\item Gi studentene en god etisk holdning til deres arbeid, samt å utvikle kritisk tenkende mennesker med djup innsikt i alle sider av den vitenskapelige prosessen.

\item Å utvikle gode utdanningsprogrammer og faglig progresjon i studieløpa, i tett samarbeid med kolleger ved Fysisk institutt (både i Norge og i USA) og  på tvers av disipliner.

\item Å sørge for at det faglige innholdet er i tråd med universitetenes samfunnsoppgave ved å utdanne svært ettertraktede kandidater til forskning, utdanning, offentlig og privat sektor.

\item Å sørge for at det faglige innholdet har intellektuelle utfordringer og innhold i tråd med nåtidige og framtidige faglige behov og forskningsretninger. 

\item Å utvikle ei kunnskapsbasert tilnærning til læring, samt utvikle forskningsprogram om hva som gir studentene økt innsikt og faglig forståelse. 

\item Å utvikle et internasjonalt perspektiv til utdanninga vår.

\item Kunnskapen skal være fritt tilgjengelig for alle.

\item Pedagogiske tiltak skal være forskningsbaserte og/eller initiere ny forskning.
  
\end{enumerate}

De ulike aktivitetene i den pedagogiske mappa gjenspeiler 
disse ti overordna målsettingene. Jeg har også prøvd etter beste evne å flette disse overordna målsettingene inn i 
kriteriene for tildeling om a) Fokus på studentenes læring, b) En klar utvikling over tid, c) En forskende tilnærming
og d) En kollegial holdning og praksis.


\subsection*{Prosjektbasert undervisning og vegen videre}

Min tilnærming til undervisning er sterkt inspirert av en serie med
vitenskapelige artikler fra 90-tallet og seinere om prosjektbasert
læring. Studentene jobber i stor grad sjølstendig med ulike
prosjekt. Prosjekta ender ofte med å være nær endelig vitenskapelige
resultat som er publisert i forskningslitteraturen. Og i noen
tilfeller har også studentprosjekta endt opp som vitenskapelige
publikasjoner. 

Alle kursa jeg har utvikla har prosjekt og ei prosjektbasert
tilnærming som grunnfilosofi. Ved å inkludere numeriske metoder og
beregninger (engelsk Computational Science) lærer studentene å studere
vitenskapelige problem med alle mulige verktøy, fra papir og blyant
til numeriske metoder. Det gir studentene en unik mulighet til å
utforske sin forståing av den gitte disiplinen og utdjupe sin innsikt
både om utvalgte fenomen, samt å utvikle ei større forståing for den
vitenskapelige prosessen. I prosjektbasert undervisning kan alle de 10
overordna målsettingene bakes inn.

Kopla opp med numeriske metoder og beregninger, har en mulighet til å
studere system som ikke har analytiske løsninger (som det er veldig
få av) og/eller krever kompliserte matematiske triks for å finne en
eventuell løsning. Med et diskretisert matematisk problem kan en feks
fokusere på overordna fysiske aspekt som feks hva er kreftene som
virker på et system, hva er randbetingelsene og initialbetingelsene og mer.
Dette gir studentene unike muligheter til å fokusere på faglig forståelse i stedet for  ulike matematiske triks. 

Studentene lærer også å dele kode og diskutere med andre studenter, de forstår bedre betydninga av reproduserbarhet 
av vitenskapelige resultat og ulike vitenskapelige etiske aspekt.

Et prosjektbasert løp tar også bort stresset fra en standard
eksamenssistuasjon, hvor en i løpet av noen få timer skal reprodusere
store deler av pensum. Et prosjektbasert løp gir mulighet for faglig
refleksjon som et jag fra oppgave til oppgave ofte ikke gir.

Studentene utvikler ofte eierskap til prosjekta og gjør som regel mye mer enn det en hadde forventa som lærer. Studentene får dermed mulighetene til utforske sine egne veger, ofte til stor personlig tilfredsstillelse, og ny lærdom for både lærere og studenter. 

Alle disse observasjonene er grundig diskutert i forskningslitteraturen og har for min del vært ei viktig ledestjerne i mitt pedagogiske arbeid.

Alle kursene jeg har utvikla de siste 20 åra har prosjektbaserte element hvor klassiske ukentlige oppgaver, standard eksamener og midtermeksamener er erstatta av prosjekt.  Kursa er:


\begin{itemize}
  
\item FYS2140 Kvantefysikk (1999-2004). Her utvikla jeg og erstatte de tradisjonelle midterm eksamenene med numeriske prosjekt samt at de ukentlige innleveringene hadde numeriske element. Tradisjonell skriftlig eksamen blei beholdt. Mye av erfaringene her leda til mitt arbeid med Computing in Science Education prosjektet. For arbeidet med dette kurset fikk jeg UiO sin undervisningspris i 2000. Den var delt med Arnt Inge Vistnes som da underviste FYS1120, Elektromagnetisme. Jeg gjorde betydelige forandringer på kurset og innførte gruppetimer med tett kontakt mellom studenter og lærere og obligatorisk innlevering av oppgaver.
  
\item FYS3150/4150 Computational Physics I: Dette er et kurs jeg utvikla som nyansatt fra bunnen av i 1999. Det er et av Fysisk institutts mest populære kurs, med ca 100 studenter fra omtrent alle institutt ved MatNat-fakultetet. Kurset er fullstendig prosjektdrevet, med regulære forelesninger og gruppetimer. Studentene får omfattende tilbakemeldinger på prosjektene og kravet til prosjektene er at de skal se ut som vitenskapelige rapporter. Det gir også studentene en betydelig skrivetrening og start på deres masterprosjekter.  Det er ingen eksamen i faget, kun prosjekter og mange av prosjektene ender opp som små forskningsprosjekter. Studentene lærer sentrale programmeringsspråk som C++ og Python, samt sentrale numeriske algoritmer. Siden dette er et kurs som tas av studenter fra heile MatNat-fakultetet, blir ofte de siste prosjektene tilpassa deres faglige bakgrunn og interesser. Jeg har også undervist dette kurset i tre år (2016-2018) ved Michigan State University.  Kurset har som kode PHY480/PHY905 Computational Physics.

\item FYS4411/9411 Computational Physics II er en fortsettelse av FYS3150/4150 og har fokus på beregningsorientert kvantemekanikk, ofte med tette koplinger til studentenes master eller PhD prosjekter. Dette kurset er også fullstendig prosjektbasert og i en del tilfeller har prosjektene resultert i publikasjoner. Jeg starta dette kurset i 2004. Omfattende tilbakemedlinger fra undertegnede og gruppelærere er gjennomgående her også.


\item  FYS-STK3155/4155 Applied Data Analysis and Machine Learning er et heilt nytt kurs som  inngår som et obligatorisk kurs i Computational Science master programmet ved UiO. Jeg utvikla dette kurset fra bunnen av i 2017 og underviste det første gang høsten 2018. I fjor fullførte 129 studenter kurset, med studenter fra omtrent alle institutt fra MatNat-fakultetet og andre fakultet ved UiO. Vi forventer flere studenter i 2020. Styrken igjen er prosjektorientering med de samme grunnleggende prinsippene; studentene får omfattende tilbakemeldinger samt at de kan definere egne prosjekt. Flere av sluttprosjektene har resultert i vitenskapelige publikasjoner eller har vært vesentlige deler av masteroppgaver og Phd avhandlinger. Kurset spiller en viktig rolle i CS programmet da det er et av treffstedene for studentene som ellers ville ha tilbragt sin tid på sitt respektive institutt. 

\item FYS-KJM4480/9480 Quantum Mechanics for Many Particle Systems er også et kurs jeg utvikla fra bunnen med flere prosjekt. Kurset er dessverre lagt ned da undertegnede underviser allerede tre andre 10 ECTS kurs ved UiO (og har 50\% stilling). Omtrent samme opplegg som de andre kursene men med endelig skriftlig eller muntlig eksamen. Prosjektbasert ellers.

\item PHY 981 Nuclear Structure ved Michigan State University (2013-2016) og PHY 989 Nuclear Forces ved Michigan State University (høst 2017). Disse kursene hadde ukentlige innleveringsoppgaver (vanlig papir og blyant arbeid) samt to større numeriske prosjekt som la grunnlaget for avsluttende muntlig eksamen. 

\item PHY 321 Classical Mechanics ved Michigan State University, første gang i år (se emneevaluering som er vedlagt). Jeg leda en komite ved MSU i 2018 som la fram forslag om integrering av numeriske   metoder i sentrale fag i en fysikk bachelorgrad. Dette svarer til implementering av Computing in Science Education liknende tiltak. PHY 321 er det første viktige fysikkurset som studenter som planlegger en bachelor grad i fysikk må ta. Sammen med fem andre kurs er dette det første fysikkurset hvor studenter nå møter numeriske oppgaver og prosjekt.  De ukentlige oppgavene inneholdt nå numeriske oppgaver og de vanlige skriftlige midtermeksamenene var erstatta med to en-ukers lange prosjekt. Pga. COVID-19 blei universitetet stengt fra 11 mars og for min del blei den skriftlige avsluttende eksamenen erstatta med et ukeslangt prosjekt, til stor glede for studentene.
  Tilbakemeldingene fra studentene viser hvor mye friheten rundt det å jobbe med prosjekter betyr for egen læring.
  
\end{itemize}

Min erfaring etter 20 år med prosjektbasert undervisning er at dette
gir studentene mye større mulighet for faglig refleksjon, utvikle
djupe innsikter om et bestemt fag og gjøre naturvitenskap slik den
gjøres i forskning, enten det er i akademia, offentlig eller privat
sektor. Alle fordeler med gruppearbeid, omfattende tilbakemeldinger,
mulighet for studentene til å komme med tilbakemeldinger som forbedrer
kursa, tett samarbeid student og lærere, og mer, gir et mye bedre
læringsmiljø.

Hittil har alle kursa jeg underviser hatt regulære forelesninger.
Vegen videre for min del er å ta det prosjektbaserte arbeidet over
til det som kalles Flipped Classroms. Her finnes det også omfattende
forskning som viser fordelene ved dette. Mitt første steg i høst blir
å utvikle videomateriale sammen med det allerede eksisterende digitale
materiale. På grunn av usikkerheten rundt semesterstart, enten fullt
digitalt eller hybrid løsning, vil et initiatv av typen Flipped
Classroms være en ypperlig måte for studentene å komme i kontakt med
faglærere i mindre grupper.  Tanken er å videreutvikle dette til et
mer varig tiltak for kursene FYS3150/4150, FYS4411/9411 og
FYS-STK3155/4155. Studentene vil da gå gjennom materialet før de møter
til gruppeundervisning. I gruppeundervisninga gjennomgås og oppklares
uklarheter rundt det ukentlige materialet, i dialog med faglærere,
gruppelærere og Learning Assistants. Deretter starter arbeidet med de
ulike oppgavene og prosjektene. Målet er at studentene kommer
forberedt til gruppene og at gruppene utvikles til gode faglige
diskusjonsfora, med tett dialog mellom studenter og lærere.

\newpage

\subsection*{Computing in Science Education} 

Et sentralt tema i prosjektorientert utdanning har
vært innføringa av numeriske prosjekt og oppgaver.  Mitt arbeid med
kurset FYS2140 i tida 1999-2004 og FYS3150/4150 la mye av grunnlaget og forståelsen
for visjonen om å integrere beregninger i vanlige fysikkurs. Ei
systematisk og helheltig tenkning rundt digital kompetanse har lagt
grunnlaget for at numeriske beregninger oppfattes av studentene som en
naturlig del av en naturviter sin verktøykasse.  Et prosjektorientert
studieløp krever også ei tettere oppfølging av studentene og legger grunnlaget for et mentorprogram og en-til-en kopling mellom lærere og studenter. Det gir mulighet for å utforme et mer individualisert studieløp.

For å oppnå måla om digital kompetanse, prosjektorientert utdanning,
tett oppfølging av studenter og mange flere av de ti måla ovafor var
jeg i 2003 en av de sentrale initiativtakerne til prosjektet
\textbf{Computing in Science Education} (CSE), sammen med kolleger på
Fysisk Institutt (Arnt Inge Vistnes), Matematisk Institutt (John Grue)
og Institutt for Informatikk (Hans Petter Langtangen og Knut
Mørken). Mye av dette skjedde via senteret for fremragende forskning
(SFF) \textbf{Center for Mathematics for Applications} og i min rolle
som programrådsleder for Bachelor programmet Fysikk, Astronomi og
Meteorologi (FAM). De første midlene som blei tildelt til CSE
prosjektet var via UiO tiltaket \textbf{Fleksibel læring} i
2004. Undertegnede var prosjektleder sammen med programrådsleder for
MIT programmet John Grue. Jeg satt også som programrådsmedlem i MIT
styret og var med å koordinere innføringa av CSE prosjektet.  Senteret
CMA spilte en sental rolle i begynnelsen og finasierte deler av
stillinger. Seinere kom også et anna SFF inn på banen, Physics of
Geological Processes (PGP) og Anders Malthe-Sørenssen ved Fysisk
Institutt. Anders var sentral i implemeteringen av CSE prosjektet i
det første fysikkurset FYS-MEK1100.

Jeg var programrådsleder for FAM i tida 2002-2011 og var drifkraft
og overordna ansvarlig for integrering av et beregningsperspektiv i
FAM programmet. Dette har resultert i at fysikkfaget ved UiO har
utdanna studenter med de digitale kompetansene som trengs for å møte
de vitenskapelige og teknologiske utfordringene i det 21ste
århundre. CSE prosjektet har fungert som inspirasjon for alle andre
studieprogram ved MatNat-fakultetet og andre fakultet ved UiO og
universitet nasjonalt.

CSE prosjeket er nå gjenspeilt i omtrent alle studieprogram ved
MatNat-fakultetet ved UiO. I tillegg har jeg leda en komite ved fysisk
institutt på Michigan State University (2018) om innføring av
CSE-liknende tiltak ved dette universitetet. I vår underviste jeg et
sentralt laveregrads kurs i klassisk mekanikk med full CSE
implementering ved Michigan State University, til positiv respons fra
studentene, se fagevalueringa i den pedagogiske mappa.

CSE prosjektet fikk UiO sin utdanningspris i 2011, samt NOKUT sin
utdanningspris i 2012. Prosjektet resulterte i at UiO fikk et senter
for fremragende utdanning i 2016, senteret for \textbf{Computing in
  Science Education} (CCSE). Senteret har blitt en stor suksess og har
utvikla flere nye forskningsprosjekt om utdanning.  Jeg er medlem av
CCSE og jobber både med kursinnhold med tanke på innføring av
beregninger samt nye tiltak rundt første studieår for Fysikk og
Astronomi programmet. I tillegg jobber jeg med å utvikle
forskningsaktivitet i kvantitativ utdanningsforskning.

\subsection*{Utvikling av studiemiljø}

Siden 2000 har jeg aktivt jobba for å utvikle et utdanningsmiljø i
Computational Physics på mastergradsnivå.  Fra og med  2007 har dette arbeidet vært
gjort sammen med en nær kollega, Anders Malthe-Sørenssen, som også er
direktør for CCSE. I 2015 blei dette arbeidet tilkjent UiO sin
utdanningspris, se
\href{{https://www.uniforum.uio.no/nyheter/2015/10/instituttet-som-lofter-fram-gode-forelesere.html}}{\nolinkurl{https://www.uniforum.uio.no/nyheter/2015/10/instituttet-som-lofter-fram-gode-forelesere.html}}.

Vi har spesielt vektlagt
\begin{enumerate}
\item Utvikle et godt sosialt miljø hvor deling av resultat og programvare står sterkt. Studentene definerer ofte sine egne prosjekt for masteroppgavene.

\item Oppbygging av lokaler som er imøtekommende og inkluderende. Bygd opp mange arenaer for sosialt samvær.

\item Studentene integreres tidlig i forskninga og studentene fungerer som læremestre for hverandre.

\item Studentene engasjeres tidlig i å utvikle læringsmateriale, spesielt som gruppelærere i kurs hvor beregninger (CSE prosjektet) er viktig.
\end{enumerate}

\noindent
Med etableringen av det nye CS-masterprogrammet har vi nå utvikla nye
tiltak for å forbedre studiekvaliteten, med blant anna tett
integrering av studentene i utforminga av nye oppgaver og prosjekt til
mastergradskursene, tett kopling mellom studenter og fagmiljø, samt
mentorprogram i oppstart av masterprogrammet, med vekt på mulige fagvalg og
karriereveger. Individuell oppfølging av studentene spiller en sentral
rolle, spesielt også for studenter som sliter mentalt.  Tett
oppfølging og individuell tilrettelegging er et sentralt aspekt som
vektlegges i utviklinga av et godt studiemiljø.

Studieprogrammet har sosiale spillkvelder med faglige seminar annen
hver fredag. Her møter studentene  forskere og/eller potensielle
oppdragsgivere fra privat og offentlig sektor. Og i en del tilfeller
har det resultert i sommerjobber og kanskje fast ansettelse seinere.

Computational Physics miljøet har siden 2003 utdanna over 100
mastergradsstudenter, og veldig mange av disse (over 50\%) har
fortsatt med PhD studier. Det er tett samarbeid mellom bachelor
studenter, mastergrads studenter, PhD studenter og Post-docs.

Studentene våre deltar også i å videreutvikle det gode læringsmiljøet
og spiller en avgjørende rolle som gruppelærere.


\subsection*{Nye tiltak}

Studentene kan med dagens teknologi (feks via applikasjoner på
smarttelefoner) utføre ulike eksperiment hjemme, samle inn data og
analysere og diskutere dataene som er samla inn. Et slikt løp
innebærer også en tettere kontakt mellom student og lærer, med større oppfølging av
den enkelte students læring.



Nylig lanserte jeg, sammen med kolleger på Fysisk Institutt, se  \href{{https://mhjensen.github.io/FirstYearPhysicsUiO/doc/pub/proposal/html/proposal-bs.html}}{\nolinkurl{https://mhjensen.github.io/FirstYearPhysicsUiO/doc/pub/proposal/html/proposal-bs.html}} for mer detaljer,  et forslag til nytt første studieår i Fysikk og Astronomi
hvor målet er å utvikle eksempler på eksperiment som kan integreres i ulike
fysikk kurs,samt hvordan en kan integrere disse eksperimentene med
numeriske metoder og programmerings kunnskapene til studentene. Det
vil innebære både en revisjon av kurs samt utvikling av nytt
læringsmateriale. Denne aktiviteten vil også integreres tett opp mot
pågående og ny forskning ved Center for Computing in Science
Education. Kvantitativ utdanningsforskning rundt temaer om hva som gir
økt innsikt for studentene om fysiske prosesser er sentrale og nye
temaer i utdanningsforskning. Hvordan en definerer kvantitativ
utdanningsforskning er et åpent tema hvor et slikt prosjekt kan være
med å bringe fram verdifull innsikt om studentenes læring. 

Det siste argumentet bringer oss  over til neste punkt.


\subsection*{Ei forskningsbasert tilnærming til utdanning}


Alle kurs jeg har utvikla i min tid ved UiO og MSU har hatt et
prosjektbasert perspektiv. All erfaring tyder på at studentene setter
pris på denne måten å tilegne seg ny kunnskap og innsikt om et
fagfelt. I tillegg gir det studentene stor frihet i å definere egne
prosjekter og utvikle eierskap til egen læring. Jeg ser også tydelig
at med klare læringsmål er det lett å oppnå mange av de ønska
effektene.

Kvalitativt ser vi tydelig at studentene via ei prosjektbasert tilnærming  har en bedre mulighet til å utvikle
djupere innsikter og forståelser om et fagfelt og den vitenskapelige
prosessen.  Vi ser også fra våre daglige vekselvirkninger med studentene at et studieløp hvor numeriske oppgaver inkluderes fra dag en, gir store muligheter for å utvikle videre studentenes innsikter og bringe resultat fra aktuell forskning inn i et tidligere stadium i studieløpet.

Men hvordan vi kan kvantisere denne økte innsikten gjenstår å se. Et
av de faglige måla til CCSE senteret er nettopp å utvikle et slikt
forskningsbasert program for kvantitativ utdanningsforskning. Dette er
et tema i forskningsfronten for feks. forskning rundt fysikk utdanning
på universitetsnivå.


Her spiller feks. maskinlæring en viktig rolle og sammen med PhD
student John Aiken ved CCSE, Prof Danny Caballero fra Michigan State University og CCSE
og andre kolleger hat vi nå utvikla flere prosjekt for å nå disse
måla.  Å kunne gi kvantitave og kvalitative mål på hva som virker er
sentralt for mange utdanningstiltak.

De to vedlagte artiklene ved slutten av dette dokumentet, og inkludert
her, viser hvordan vi kan bruke feks kvantitative metoder for å kunne
si noe om hva som virker eller ikke ved ulike utdanningstiltak. Den
andre artikkelen her setter opp den generelle ramma for arbeidet
rundt et kvantitativt forskningsprogram mens den første artikkelen
viser hvordan maskinlæring kan brukes til å forstå studieprogresjon
for studenter. Vårt mål er å videreutvikle ei slik kvantitativ
tilnærming til å kunne si noe om studentene faktisk opplever økt
faglig innsikt. Det er en lang veg å gå for å oppnå dette, men vi har
starta.


\begin{enumerate}
\item John M. Aiken, Riccardo De Bin, Morten Hjorth-Jensen, Marcos D. Caballero, \emph{Predicting time to graduation at a large enrollment American university}, arXiv:2005.05104 

\item Marcos Daniel Caballero, Morten Hjorth-Jensen, \emph{Integrating a Computational Perspective in Physics Courses}, arXiv:1802.08871

\end{enumerate}


I tillegg har jeg skrevet flere bøker med fokus på
beregningsorienterte metoder i fysikk. Her følger ei liste over
aktuelle bøker for den pedagogiske mappa.

\paragraph*{Bøker:}
\begin{enumerate}
\item Morten Hjorth-Jensen, \emph{Computational Physics, an introduction}, to be published by IOP in 2020, 500 pages.

\item Morten Hjorth-Jensen, \emph{Computational Physics, an advanced course}, to be published by IOP in 2020, 400 pages

\item Morten Hjorth-Jensen, \emph{Nuclear many-body physics, a computational perspective}, in preparation for Lecture Notes in Physics by Springer.

\item \href{{http://www.springer.com/us/book/9783319533353}}{Morten Hjorth-Jensen, M.P. Lombardo and U. van Kolck}, \emph{Computational Nuclear Physics-Bridging the scales, from quarks to neutron stars}, Lectures Notes in Physics by Springer, Volume \textbf{936} (2017).
\end{enumerate}




\subsection*{Studenter og Post-Docs}

Jeg har vegleda og vegleder over 100 studenter på alle nivå. Flere av
PhD studentene har endt opp i vitenskapelige stillinger i inn og
utland. Simen Kvaal (nå ansatt ved Kjemisk Institutt UiO) fikk et ERC
stipend og Gaute Hagen (ansatt ved Oak Ridge National Laboratory) fikk
det prestisjefylte Young Investigator Award fra Department of
Energy. Omtrent 50\% as masterstudentene har fortsatt med PhD studier enten med meg eller andre som vegledere.

\paragraph*{Nåværende PhD studenter.}
\begin{enumerate}
\item Benjamin Hall, Michigan State University, started 2018.

\item Jane Kim, Michigan State University, started 2018.

\item Julie Butler, Michigan State University, started 2018.

\item Omokuyani C. Udiani , Michigan State University, started 2017, co-supervisor

\item Danny Jammoa, Michigan State University, started 2020, co-supervisor

\item Øyvind Sigmundsson Schøyen, University of Oslo, started 2019

\item John Mark Aiken, University of Oslo, started 2017, defends thesis September 2020
\end{enumerate}

\noindent
\paragraph*{Nåværende masterstudenter.}
\begin{enumerate}
\item Eina Jørgensen, University of Oslo, (2019-2021), co-supervisor

\item Morten Hemmingsen, University of Oslo, (2019-2021), co-supervisor

\item Huying Zhu, University of Oslo, (2019-2021), co-supervisor

\item Jens Due Bratten, University of Oslo, (2019-2021), co-supervisor

\item Gabriel Cabrera, University of Oslo, (2019-2021), co-supervisor

\item Kristian Wold, University of Oslo, (2019-2021)

\item Martin Krokan Hovden, University of Oslo, (2019-2021)

\item Oliver Hebnes, University of Oslo, (2019-2021), co-supervisor

\item Mohamad Ismail, University of Oslo, (2019-2021), co-supervisor

\item Heine Aabø, University of Oslo, (2018-2020)

\item Stian Bilek, University of Oslo, (2018-2020)

\item Thomas Sjåstad, University of Oslo, (2018-2020)

\item Halvard Sutterud, University of Oslo, (2018-2020)

\item Stian Isachsen, University of Oslo, (2018-2020), co-supervisor

\item Marius Holm, University of Oslo, (2018-2020), co-supervisor

\item Halvard Sutterud, University of Oslo, (2018-2020)

\item Geir Utvik, University of Oslo, (2018-2020)

\item Markus Asprusten, University of Oslo, (2018-2020), co-supervisor
\end{enumerate}

\noindent
\paragraph*{Tidligere PhD studenter og deres nåværende stillinger.}
\begin{enumerate}
\item Justin Lietz (PhD MSU 2019), now post-doctoral fellow at Oak Ridge National Laboratory

\item Samuel Novario (PhD MSU 2018), now post-doctoral fellow at Oak Ridge National Laboratory

\item Fei Yuan (PhD MSU 2018), employed at Google

\item \href{{https://fi.linkedin.com/in/gustav-baardsen-831a5162}}{Gustav Baardsen} (PhD UiO 2014), Research Scientist at Varian Medical Systems, Helsinki, Finland

\item \href{{http://www.mn.uio.no/kjemi/english/people/aca/simenkv/index.html}}{Simen Kvaal} (PhD UiO 2009), now associate professor of chemistry, Department of Chemistry, University of Oslo. Recipient of an ERC starting grant

\item \href{{https://www.ornl.gov/staff-profile/gustav-r-jansen}}{Gustav Jansen} (PhD UiO 2012), now permanent position as scientist at the Computational Science Division of Oak Ridge National Laboratory  

\item \href{{http://www.mn.uio.no/math/english/people/aca/tmac/}}{Torquil MacDonald Sørensen} (PhD UiO 2012), post-doctoral fellow at the Department of Mathematics, UiO

\item \href{{http://www.usit.uio.no/english/about/organisation/bps/rc/ris/staff/jonkni/}}{Jon Kerr Nilsen} (PhD UiO 2010), senior engineer at the University of Oslo center for information technologies (co-supervisor)

\item \href{{https://www.hioa.no/tilsatt/marlys}}{Marius Lysebo} (PhD UiO 2010), now Associate Professor at Oslo University College, (co-supervisor)

\item \href{{http://www.aas.vgs.no/om-oss/organisasjon/alle-ansatte/}}{Elise Bergli} (PhD UiO 2010), teacher Ås high school, Norway

\item \href{{https://www.hbv.no/om-hbv-kontakt-oss-ansatte/eirik-ovrum-article125026-6688.html}}{Eirik Ovrum} (PhD UiO 2007), now Associate Professor at the University College of Southeast of Norway

\item \href{{https://www.ornl.gov/staff-profile/gaute-hagen}}{Gaute Hagen} (PhD UiB and UiO 2005), now permanent position as scientist at the Physics Division of Oak Ridge National Laboratory. Recipient of the Department of Energy Early career award

\item Maxim Kartamyshev (PhD UiO), now at the Bank of Norway as senior analyst

\item Øystein Elgarøy (PhD UiO 1999), now professor of Theoretical Astrophysics at the University of Oslo, Norway (co-supervisor)

\item Lars Engvik (PhD UiO 1999), now Associate Professor at Sør-Trøndelag University College, Trondheim, Norway, (co-supervisor)
\end{enumerate}

\noindent
\paragraph*{Post-docs og deres nåværende stillinger.}
\begin{enumerate}
\item \href{{https://www.chalmers.se/en/Staff/Pages/Andreas-Ekstrom.aspx}}{Andreas Ekstrøm} (UiO and MSU 2010-2014), now Associate Professor  at Chalmers Technological University in Gothenburg, Sweden

\item Øyvind Jensen (UiO 2011), now researcher at the \href{{https://www.ife.no/en}}{Institute for Energy Technology}

\item \href{{http://www.mn.uio.no/kjemi/english/people/aca/simenkv/index.html}}{Simen Kvaal} (UiO 2008-2012), now associate professor of chemistry, Department of Chemistry, University of Oslo. Recipient of an ERC starting grant

\item Elise Bergli (UiO 2010-2011), now teacher at Ås high school, Norway

\item Sølve Selstø (UiO 2008-2010), now  Professor at Oslo Metropolitan University

\item Nicolas Michel (MSU 2013), now senior researcher at Langzhou Nuclear Physics Laboratory, China
\end{enumerate}

\noindent
\paragraph*{Masterstudenter som har avlagt eksamen.}
\begin{enumerate}
\item Vebjørn Gilberg, University of Oslo, (2017-2020), co-supervisor

\item Kari Eriksen, University of Oslo, (2017-2020)

\item Robert Solli, University of Oslo, (2017-2019)

\item Andreas Lefdalsnes, University of Oslo, (2017-2019)

\item Joseph Knutson, University of Oslo, (2017-2019)

\item Bendik Samseth, University of Oslo, (2017-2019)

\item Even Nordhagen, University of Oslo, (2017-2019), (now PhD student)

\item Øyvind Schøyen Sigmundson, University of Oslo, (2017-2019), (now PhD student)

\item Sebastian Gregorius Winther-Larsen, University of Oslo, (2017-2019),  (now PhD student)

\item Giovanni Pederiva, University of Oslo, (2016-2018), co-supervisor,  (now PhD student)

\item Anna Gribovskaya, University of Oslo, (2016-2018)

\item Andrei Kucharenka, University of Oslo, (2016-2018)

\item Vilde Moe Flugsrud, University of Oslo, (2016-2018)

\item Alfred Alocias Mariadason, University of Oslo, (2016-2018)

\item Marius Jonsson, University of Oslo, (2016-2018),  (now PhD student)

\item Hans Mathias Vege Mamen, University of Oslo, (2016-2019), co-supervisor

\item Alexander Fleischer, University of Oslo, (2015-2017)

\item Håkon Emil Kristiansen, University of Oslo, (2015-2017),  (now PhD student)

\item Morten Ledum, University of Oslo, (2015-2017),  (now PhD student)

\item Håkon Treider Vikør, University of Oslo, (2015-2017), co-supervisor

\item Jon-Andreas Stende, University of Oslo, (2015-2017), co-supervisor

\item Sean Bruce Sangholt Miller, University of Oslo, (2015-2017),  (now PhD student)

\item Christian Fleischer, University of Oslo, (2015-2017)

\item John Bower, Michigan State University, (2014-2017)

\item Wilhelm Holmen, University of Oslo (2014-2016)

\item Roger Kjøde, University of Oslo, (2014-2016)

\item Håkon Sebatian Mørk, University of Oslo, (2014-2016)

\item Jonas van den Brink, University of Oslo, (2014-2016), co-supervisor,  (now PhD student)

\item Marte Julie Sætra, University of Oslo, (2014-2016), co-supervisor,  (now PhD student)

\item Audun Skau Hansen, University of Oslo, (2013-2015),  (now PhD student)

\item Henrik Eiding, University of Oslo, (2012-2014)

\item Svenn-Arne Dragly, University of Oslo, (2012-2014), defended PhD

\item Milad Hobbi Mobarhan, University of Oslo, (2012-2014), defended PhD

\item Ole Tobias Norli, University of Oslo, (2012-2014)

\item Filip Sand, University of Oslo, (2012-2014), co-supervisor

\item Emilie Fjørner, University of Oslo, (2012-2014), co-supervisor

\item Jørgen Høgberget, University of Oslo, (2011-2013), , defended PhD

\item Sarah Reimann, University of Oslo, (2011-2013), defended PhD

\item Karl Leikganger, University of Oslo, (2011-2013), defended PhD

\item Sigve Bøe Skattum, University of Oslo, (2011-2013), defended PhD

\item Veronica Berglyd Hansen, University of Oslo, (2010-2012), defended PhD

\item Camilla Nestande Kirkemo, University of Oslo, (2010-2012), co-supervisor

\item Christoffer Hirth, University of Oslo, (2009-2011)

\item Marte Hoel Jørgensen, University of Oslo, (2009-2011)

\item Yang Min Wang, University of Oslo, (2009-2011), defended PhD

\item Ivar Nikolaisen, University of Oslo, (2009-2011), began on PhD

\item Vegard Amundsen, University of Oslo, (2008-2010)

\item Håvard Sandsdalen, University of Oslo, (2008-2010)

\item Lars Eivind Lervåg, University of Oslo, (2008-2010)

\item Magnus Lohne Pedersen, University of Oslo, (2008-2010)

\item Simen Sørby, University of Oslo, (2008-2010), co-supervisor

\item Sigurd Wenner, University of Oslo, (2008-2010), co-supervisor, defended PhD

\item Lene Norderhaug Drøsdal, University of Oslo, (2007-2009), defended PhD

\item \href{{https://www.nilu.no/OmNILU/Kontaktoss/Ansatte/tabid/70/ctl/EmployeeDetails/mid/972/employeeid/5822/tabmoduleid/2333/language/en-GB/Default.aspx}}{Islen Vallejo, University of Oslo, (2007-2009)}, works at the Norwegian Institute for Air Research

\item Jacob Kryvi, Norwegian University of Science and Technology, (2007-2009), co-supervisor, defended PhD

\item Rune Albrigtsen, University of Oslo, (2007-2009)

\item Johannes Rekkedal, University of Oslo, (2007-2009), began PhD

\item Patrick Merlot, University of Oslo, (2007-2009), began PhD

\item Gustav Jansen, University of Oslo, (2006-2008), defended PhD

\item Ole Petter Harbitz, University of Oslo, (2006-2008)

\item Sutharsan Amurgian, University of Oslo, (2005-2007)

\item Jon Thonstad, University of Oslo, (2005-2007)

\item Espen Flage-Larsen, University of Oslo, (2003-2005), defended PhD

\item Joachim Berdahl Haga, University of Oslo, (2004-2006), defended PhD

\item Jon Kerr Nilsen, University of Oslo, (2002-2004), defended PhD

\item Simen Kvaal, University of Oslo, (2002-2004), defended PhD

\item Simen Reine Sommerfelt, University of Oslo, (2002-2004), defended PhD

\item Mateuz Marek Røstad, University of Oslo, (2002-2004)

\item Victoria Popsueva, University of Oslo, (2002-2004), defended PhD

\item Eivind Brodal, University of Oslo, (2001-2003), defended PhD

\item Eirik Ovrum, University of Oslo, (2001-2003), defended PhD

\item Ronny Kjelsberg, Norwegian University of Science and Technology, (2001-2003)
\end{enumerate}




\subsection*{Medlem av PhD komiteer ved MSU}

\begin{enumerate}
\item Justin Lietz, chair, defended thesis June 2019.

\item Fei Yuan, chair.  Defended thesis January 24 2018.

\item Sam Novario, chair. Defends thesis February 7 2018.

\item John Bower, chair together with Scott Bogner. Master of Science thesis May 2017.  

\item Adam Jones, committee member. Master of Science thesis July 2017.  

\item Chris Sullivan, committee member. Defended thesis January 2018.

\item Thomas Redpath, committee member. Defended thesis October 2019.

\item Sean Sweany, committee member

\item Rachel Taverner, committee member. Defended thesis May 2019.

\item Nathan Parzuchowski, committee member. Defended thesis April 2017.

\item Titus Morris, committee member. Defended thesis May 2016

\item Kenneth Whitmore, committee member. Defended thesis June 2016

\item Alex Dombos, committee member. Defended thesis May 2018.

\item Josh Bradt, committee member, Defended thesis July 2017

\item Charles Loelius, committee member, Defended thesis May 2017

\item Safwan Shanab, committee member. Defended thesis January 2020.

\item Hao Lin, committee member

\item Mao Xingze, committee member. Defended thesis May 2020.

\item Amy Lovell, committee member. Defended thesis January 24 2018.

\item Debra Richman, committee member

\item Roy Ready, committee member

\item Nathan Watwood, committee member 

\item Ben Hall, chair

\item Udiani Omokuyani, committee member

\item Jane Kim, chair
\end{enumerate}


\subsection*{Internasjonale utdanningstiltak}

Sammen med kolleger fra flere land starta jeg og etablerte et
internasjonalt initiativ i 2010 kalt \textbf{Nuclear TALENT (Training in
Advanced Low-Energy Nuclear Theory)} hvis mål er å styrke
en faglig bredde i kjernefysikk internasjonalt. Mange universitet har
ikke nok vitenskapelig personale til å gi studentene på master og PhD
nivå den nødvendige faglige bredde i feltet. Nuclear TALENTs mål
er å tilby denne faglige bredden i form av et titalls avanserte kurs
som undervises på et intensivt vis over tre uker ulike steder i verden
(Nord-Amerika, Europa og Asia). Siden sommeren 2012 har vi organisert
over 15 slike kurs og jeg har undervist og organisert 5 av disse
kursene og organisert tre andre. Dette tiltaket har vært en enorm
suksess med over 500 deltakere totalt siden 2012. Pga COVID-19 er alle
tre kurs i år utsatt til 2021, men kurset jeg har ansvaret for i år om maskinlæring anvendt på kjernefysikk tilbys
digitalt i juni-juli 2020, se \href{{http://www.ectstar.eu/node/4472}}{\nolinkurl{http://www.ectstar.eu/node/4472}}.



For mer informasjon om Nuclear TALENT, se \href{{https://fribtheoryalliance.org/TALENT/}}{\nolinkurl{https://fribtheoryalliance.org/TALENT/}}.

Ellers leder jeg et større INTPART prosjekt om Computing in Science
Education mellom CCSE ved UiO, Michigan State University, Oregon State
University og University of Colorado ved Boulder. Et viktig mål med
dette prosjektet er å utvikle et program i kvantitativ utdanningsforskning. Vår
første workshop om dette blei  dessvere avlyst i år pga COVID-19 situasjonen.

Jeg har også etablert flere internasjonale utvekslingsprogrammer for
studenter i Oslo, spesielt mot USA og Europa.

Her følger en liste over skoler jeg har organisert.

\subsection*{Organisering av skoler og foredrag ved skoler}

\begin{enumerate}
  
\item Morten Hjorth-Jensen, Nuclear Talent Course on Machine Learning in Nuclear Physics for the Erasmus+ program \href{{http://www.emm-nucphys.eu/}}{European Master in Nuclear Physics}, University of Basse-Normandie and GANIL, January 20-31, 2020. 45 lectures and 45 exercise sessions. Main teacher

\item Morten Hjorth-Jensen, Matthew Hirn, Michelle Kuchera, and R. Ramanujan, \href{{https://indico.frib.msu.edu/event/16/}}{FRIB TA Summer School - Machine Learning Applied to Nuclear Physics}, Facility for Rare Isotope Beams (FRIB) on the Michigan State University campus in East Lansing, MI from May 20 to 23, 2019. Main organizer and teacher.

\item Morten Hjorth-Jensen, Nuclear Talent Course on Machine Learning in Nuclear Physics for the Erasmus+ program \href{{http://www.emm-nucphys.eu/}}{European Master in Nuclear Physics}, University of Basse-Normandie and GANIL, January 21-February 1, 2019. 45 lectures and 45 exercise sessions. Main teacher

\item Nuclear Talent course on Many-body methods for nuclear physics, from Structure to Reactions at Henan Normal University, P.R. China, July 16-August 5 2018. Teachers: Kevin Fossez, Morten Hjorth-Jensen, Thomas Papenbrock, and Ragnar Stroberg. 
  
\item Alex Brown, Alexandra Gade, Morten Hjorth-Jensen, Gustav Jansen, Robert Grzywacz, Nuclear Talent course on Nucleartheory for Nuclear Structure Experiments, July 3-21 2017. \href{{https://github.com/NuclearTalent/NuclearStructure}}{Main organizer and teacher with in total fifteen hours of lectures}. 

\item Hjorth-Jensen, Morten, \href{{https://icer-acres.msu.edu/summer-2017/schedule/}}{High performance computing in Nuclear Physics}, Lecture at the \emph{Advanced Computational Research Experience} at Michigan State University, East Lansing, Michigan, June 1, 2017.

\item Hjorth-Jensen, Morten, \href{{https://icer-acres.msu.edu/summer-2017/schedule/}}{How to write good code}, Lecture at the \emph{Advanced Computational Research Experience} at Michigan State University, East Lansing, Michigan, May 24, 2017.

\item Hjorth-Jensen, Morten, \href{{http://rafael.ujf.cas.cz/school}}{Computational Nuclear Physics and Post Hartree-Fock Methods. Configuration Interaction Theory, Many-Body Perturbation Theory and Coupled Cluster Theory}, five lectures at 28th Indian-Summer School on Ab Initio Methods in Nuclear Physics, Prague, Czech Republic, August 29 - September 2, 2016.

\item Hjorth-Jensen, Morten, \href{{http://compphysics.github.io/CompPhysUTunis/doc/web/course.html}}{Computational Physics and Quantum Mechanical Systems}, one week course on Computational Physics at the University of Tunis El Manar, Tunis, Tunisia, May 16-20, 2016. In total 15 hours of lectures and 15 hours of computer lab and exercises. 

\item Co-organizer with Giuseppina Orlandini and Alejandro Kievsky of Nuclear Talent course \href{{https://groups.nscl.msu.edu/jina/talent/wiki/Course_3}}{Few-body methods and nuclear reactions}, ECT*, Trento, Italy, July 20-August 7 2015

\item Carlo Barbieri, Wim Dickhoff, Gaute Hagen, Morten Hjorth-Jensen, and Artur Polls, Nuclear Talent course on Many-body methods for nuclear physics, GANIL, Caen, France, July 5-25 2015. \href{{http://nucleartalent.github.io/Course2ManyBodyMethods/doc/web/course.html}}{Main organizer and teacher with in total five hours of lectures}. 

\item Hjorth-Jensen, Morten, ECT* \href{{http://www.ectstar.eu/node/1287}}{Doctoral Training Program 2015 on Computational Nuclear Physics}, April 13- May 22, ECT*, Trento, Italy. I taught the last week of the lecture series. In total I have ten one hour lectures. 

\item Hjorth-Jensen, Morten, Nuclear Talent School in Nuclear Astrophysics, co-organizer with Richard Cyburt and Hendrik Schatz of the Nuclear Talent course on Nuclear Astrophysics,  Michigan State University, May 26 - June 13, 2014. 

\item Hjorth-Jensen, Morten, Nuclear Talent course on Density Functional theories, co-organizer with Scott Bogner, Nicolas Schunck, Dario Vretenar and Peter Ring, European Center for Theoretical Nuclear Physics and Related Areas, Trento, Italy, July 13 -August 1 2014.

\item Hjorth-Jensen, Morten, Nuclear Talent Course  Introduction on High-performance computing and computational tools for nuclear physics; ECT*, Trento, Italy, June 24 - July 13 2012. Main organizer and teacher together with Francesco Pederiva, Kevin Schmidt and Calvin Johnson. 

\item Hjorth-Jensen, Morten. Computational environment for Nuclear Structure, five lectures in Nuclear Physics at Universidad Complutense Madrid; 2011-01-17 - 2011-02-09

\item Hjorth-Jensen, Morten, organizer with David Dean, Thomas Papenprock and Gaute Hagen. Third MSU-UT/ORNL-UiO winter school in nuclear physics; Oak Ridge National Lab, Tennessee, January 2012

\item Hjorth-Jensen, Morten, organizer with Alex Brown and teaching five lectures. Second MSU-UT/ORNL-UiO winter school in nuclear physics, East Lansing, Michigan, USA; 2011-01-03 - 2011-01-07

\item Hjorth-Jensen, Morten, organizer, First MSU-UT/ORNL-UiO winter school in nuclear physics, Wadahl, Norway, January 4-10 2010

\item Hjorth-Jensen, Morten.  Five lectures on Theory of shell-model studies for nuclei. CERN/Isolde course on nuclear structure theory; 2010-03-01 - 2010-03-04

\item Hjorth-Jensen, Morten.  Six lectures on Nuclear interactions and the Shell Model. 8th CNS-EFES International Summer School, Riken, Tokyo, Japan, 2009-08-26 - 2009-09-01

\item Hjorth-Jensen, Morten.  Five lectures on nuclear theory at the  20th Chris Engelbrecht Summer School in Theoretical Physics, Stellenbosch, South Africa,  2009-01-19 - 2009-01-28

\item Hjorth-Jensen, Morten.  Nuclear many-body theory, five lectures at the  UK Postgraduate Nuclear Physics Summer School, Leicester, UK,  2009-09-12 - 2009-09-23

\item Hjorth-Jensen, Morten.  Nuclear many-body methods. Lectures series at Lund University; 2008-05-04 - 2008-05-07

\item Hjorth-Jensen, Morten.  Trends in Nuclear Structure Theory. Workshop at the University of Lund; 2008-05-07 - 2008-05-07

\item Hjorth-Jensen, Morten.  Trends in Nuclear Structure Theory. Physics Division Seminar; 2008-04-17 - 2008-04-17

\item Hjorth-Jensen, Morten.  Trends in nuclear structure theory. Lecture series at the University of Padova and Legnaro National Laboratory, Padova Italy; 2008-07-16 - 2008-07-19

\item Hjorth-Jensen, Morten.  Five lectures on  Monte Carlo methods and applications in the physical sciences. eScience Winther School 2007; Geilo, Norway 2007-01-28 - 2007-02-02

\item Hjorth-Jensen, Morten.  Five lectures at the ISOLDE Spring School in Nuclear Theory; CERN, Switzerland, 2007-05-21 - 2007-05-26

\item Hjorth-Jensen, Morten.  Ten lecures at  ECT* Doctoral Training Programme 2007; Trento, Italy, April 16-20

\item Hjorth-Jensen, Morten.  From the nucleon-nucleon interaction to a renormalized interaction for nuclear systems. Lecture series at Michigan State University; April 2005

\item Hjorth-Jensen, Morten. CENS: A computational Environment for Nuclear Structure. Isolde Lecture series; 2004-11-11 - 2005-11-25
\end{enumerate}


\subsection*{Undervisningsrelevante foredrag}

\begin{enumerate}
\item \href{{https://phys.au.dk/en/news/item/artikel/ole-roemer-colloquium-morten-hjort-jensen-tba/}}{Hjorth-Jensen, Morten, Århus University, Denmark, workshop and  Ole Rømer Colloquium: Integrating a Computational Perspective in Physics (and Science) Courses, October 23, 2019}

\item Hjorth-Jensen, Morten, Computing in Science Education, seminar at the Department of Physics, University of Trento, Trento, Italy, March 5, 2019.

\item Hjorth-Jensen, Morten, "Integrating Computations in Physics Courses, Workshop on New Horizons in Teaching Science: 18th-19th, June 2018, University of Messina, Italy"

\item Hjorth-Jensen, Morten, \href{{http://www.nucleartheory.net/NPG/recent_seminars.htm}}{Computing in Science Education; how to integrate computing in Science courses across disciplines, seminar at the University of Surrey, UK, November 28 2017}

\item Hjorth-Jensen, Morten, \href{{https://www.sif.it/attivita/congresso/103}}{Computing in Physics Education, Invited talk at the 103rd National congress of the Italian Physical Society}, Trento, September 11-15, 2017, Italy

\item Hjorth-Jensen, Morten, Integrating a Computational Perspective in the Basic Science Education, Special Lectures and Events, Notre Dame University, South Bend, Indiana, March 30 2015.

\item Hjorth-Jensen, Morten, Computing in Science Education.  Integrating a Computational Perspective in the Basic Science Education, Physics Colloquium, Central Michigan University, Mt Pleasant, March 19 2015.

\item Hjorth-Jensen, Morten, Computing in Science Education.  Integrating a Computational Perspective in the Basic Science Education, condensed matter seminar, Ohio University, Athens, Ohio,  February 26 2015.

\item Hjorth-Jensen, Morten, Computing in Science education, how to introduce a computational perspective in the basic science education, special colloquium Department of Physics, Lousiana State University, Baton Rouge, Lousiana, April 4 2014.

\item Hjorth-Jensen, Morten.  Educating the next generation of nuclear scientists; how can a center like the ECT* aid in developing modern nuclear physics educational programs?. ECT* 20th anniversary colloquium; 2013-09-14 - 2013-09-14

\item Hjorth-Jensen, Morten.  Computing in Science Education. Seminar at college of engineering; 2012-03-15 - 2012-03-15

\item Hjorth-Jensen, Morten.  Computing in Science Education, a new way to teach science?. Institute seminar The Ohio State University; 2012-02-28 - 2012-02-28

\item Hjorth-Jensen, Morten.  Computers in Science Education; a new way to teach Science?. Institute seminar; 2011-03-21 - 2011-03-21

\item Hjorth-Jensen, Morten.  Computers in Science Education; a new way to teach Science?. Seminar at Universidad Complutense Madrid; 2011-01-24 - 2011-01-24

\item Hjorth-Jensen, Morten.  Computers in Science Education. Institute seminar at the university of Trento, Italy; 2010-05-05 - 2010-05-05

\item Hjorth-Jensen, Morten.  Datamaskiner i realfagsopplæringen, en ny måte å undervise realfag på?. Institutt kollokvium; 2009-02-13 - 2009-02-13

\item Hjorth-Jensen, Morten.  Computers in Science Education. Guest lecture at Michigan State University; 2008-03-30 - 2008-03-30

\item Hjorth-Jensen, Morten.  Computers in Science Education. Forelesning ved UniK, Kjeller; 2008-10-23 - 2008-10-23

\item Hjorth-Jensen, Morten.  Computers in Science education, a new way to teach science?. eNORIA: Workshop on eScience in Higher Education; 2008-10-07 - 2008-10-07

\item Hjorth-Jensen, Morten; Langtangen, Hans Petter; Malthe-Sørenssen, Anders; Mørken, Knut Martin; Vistnes, Arnt Inge.  Computers in Science Education, a new way to teach physics and mathematics?. April Meeting of the American Physical Society; 2008-04-11 - 2008-04-15

\item Hjorth-Jensen, Morten; Mørken, Knut Martin.  Computers in Science Education A New Way to Teach Science?. ”I POSE OG SEKK” - Kvalitet i både forskning og utdanning. Er det mulig?; 2008-11-12 - 2008-11-13

\item Hjorth-Jensen, Morten; Mørken, Knut Martin.  Computers in Science Education A New Way to Teach Science?. Møte i Nasjonalt råd for teknologisk utdanning; 2008-11-11 - 2008-11-11

\item Hjorth-Jensen, Morten.  Computeres in Science Education, a new way to teach science?. Institute seminar; 2007-05-15 - 2007-05-15

\item Hjorth-Jensen, Morten.  Computers in Science Education, a new way to teach science?. EUPEN's 9th General Forum - EGF2007; 2007-09-06 - 2007-09-08

\item Hjorth-Jensen, Morten.  Computers in Science Education: realfagsundervisning på en ny måte?. Pedagogisk modul for MN-fak; 2007-04-11 - 2007-04-11

\item Hjorth-Jensen, Morten.  How to Integrate Parallel Computing in Science Education?. High-Performance and Parallel Computing; 2007-10-24 - 2007-10-24

\item Hjorth-Jensen, Morten; Mørken, Knut Martin. Computers in Science Education, realfag på en ny måte?. Realfag – nøkkelen til fremtidens kunnskapssamfunn; 2007-03-23 - 2007-03-23

\item Hjorth-Jensen, Morten; Mørken, Knut Martin.  Computers in Science Education: Realfagsundervisning på en ny måte?. Presentasjon for Abelia og NHO; 2007-08-14 - 2007-08-14

\item Hjorth-Jensen, Morten.  Computers in Science Education. CMA workshop on 'Computers, computations and science education'; 2005-09-30 - 2005-09-30

\item Hjorth-Jensen, Morten.  Kvalitetsreformen, nye Muligheter for Samarbeid mellom Universitet og Næringsliv. Industridag, rom for muligheter, Universitetet i Oslo; 2005-09-16 - 2005-09-16

\item Hjorth-Jensen, Morten. Økt innsikt og læring ved hjelp av IKT i Fysikk. Det Umuliges kunst? IKT i utdanning - kvalitetetsreformen i praksis; 2004-04-28 - 2004-04-28

\item Vistnes, Arnt Inge; Hjorth-Jensen, Morten. Numerical methods as an integrated part of physics education. 9th Workshop on Multimedia in Physics Teaching and Learning; 2004-09-09 - 2004-09-11

\item Hjorth-Jensen, Morten. Bruk av numeriske verktøy i undervisningen. Pedagogisk modul i 'Undervisning i matematiske og naturvitenskapelige fag', UNiversitetet i Oslo; 2003-05-23 - 2003-05-23
\end{enumerate}


\end{document}

