%%
%% Automatically generated file from DocOnce source
%% (https://github.com/hplgit/doconce/)
%%
%%
% #ifdef PTEX2TEX_EXPLANATION
%%
%% The file follows the ptex2tex extended LaTeX format, see
%% ptex2tex: http://code.google.com/p/ptex2tex/
%%
%% Run
%%      ptex2tex myfile
%% or
%%      doconce ptex2tex myfile
%%
%% to turn myfile.p.tex into an ordinary LaTeX file myfile.tex.
%% (The ptex2tex program: http://code.google.com/p/ptex2tex)
%% Many preprocess options can be added to ptex2tex or doconce ptex2tex
%%
%%      ptex2tex -DMINTED myfile
%%      doconce ptex2tex myfile envir=minted
%%
%% ptex2tex will typeset code environments according to a global or local
%% .ptex2tex.cfg configure file. doconce ptex2tex will typeset code
%% according to options on the command line (just type doconce ptex2tex to
%% see examples). If doconce ptex2tex has envir=minted, it enables the
%% minted style without needing -DMINTED.
% #endif

% #define PREAMBLE

% #ifdef PREAMBLE
%-------------------- begin preamble ----------------------

\documentclass[%
oneside,                 % oneside: electronic viewing, twoside: printing
final,                   % draft: marks overfull hboxes, figures with paths
10pt]{article}

\listfiles               %  print all files needed to compile this document

\usepackage{relsize,makeidx,color,setspace,amsmath,amsfonts,amssymb}
\usepackage[table]{xcolor}
\usepackage{bm,ltablex,microtype}

\usepackage[pdftex]{graphicx}

\usepackage[T1]{fontenc}
%\usepackage[latin1]{inputenc}
\usepackage{ucs}
\usepackage[utf8x]{inputenc}

\usepackage{lmodern}         % Latin Modern fonts derived from Computer Modern

% Hyperlinks in PDF:
\definecolor{linkcolor}{rgb}{0,0,0.4}
\usepackage{hyperref}
\hypersetup{
    breaklinks=true,
    colorlinks=true,
    linkcolor=linkcolor,
    urlcolor=linkcolor,
    citecolor=black,
    filecolor=black,
    %filecolor=blue,
    pdfmenubar=true,
    pdftoolbar=true,
    bookmarksdepth=3   % Uncomment (and tweak) for PDF bookmarks with more levels than the TOC
    }
%\hyperbaseurl{}   % hyperlinks are relative to this root

\setcounter{tocdepth}{2}  % levels in table of contents

% prevent orhpans and widows
\clubpenalty = 10000
\widowpenalty = 10000

% --- end of standard preamble for documents ---


% insert custom LaTeX commands...

\raggedbottom
\makeindex
\usepackage[totoc]{idxlayout}   % for index in the toc
\usepackage[nottoc]{tocbibind}  % for references/bibliography in the toc

%-------------------- end preamble ----------------------

\begin{document}

% matching end for #ifdef PREAMBLE
% #endif

\newcommand{\exercisesection}[1]{\subsection*{#1}}


% ------------------- main content ----------------------



% ----------------- title -------------------------

\thispagestyle{empty}

\begin{center}
{\LARGE\bf
\begin{spacing}{1.25}
Short description of many-body program
\end{spacing}
}
\end{center}

% ----------------- author(s) -------------------------

\begin{center}
{\bf Morten Hjorth-Jensen${}^{1, 2}$} \\ [0mm]
\end{center}

\begin{center}
% List of all institutions:
\centerline{{\small ${}^1$Department of Physics, University of Oslo, Norway}}
\centerline{{\small ${}^2$Department of Physics and Astronomy and National Superconducting Cyclotron Laboratory, Michigan State University, USA}}
\end{center}
    
% ----------------- end author(s) -------------------------

% --- begin date ---
\begin{center}
June 2016
\end{center}
% --- end date ---

\vspace{1cm}


Together with colleagues at Michigan State University and Oak Ridge National Laboratory and the University of Tennessee (Knoxville), I have during the last two decades been working on and developing several many-body methods tailored to the nuclear many-body problem. Many of these methods are widely used in 
condensed matter physics, atomic and molecular physics and quantum chemistry. We have also applied these methods to condensed matter systems like quantum dots
and the homogeneous electron gas in two and three dimensions. 
The main focus has however been on applications to nuclear systems like nuclei and infinite and dense nuclear matter. The reasons for this is that 
to understand why matter is stable, and thereby shed light on the limits of nuclear stability, is one of the overarching aims and intellectual challenges of basic research in nuclear physics and science. To relate the stability of matter to the underlying fundamental forces and particles of nature as manifested in nuclear matter is central to present and planned rare isotope facilities.
Till now we have focused mainly on structure properties and less on dynamical properties and non-equilibrium physics.

\paragraph{The many-body methods we have focused on are.}
\begin{itemize}
\item Coupled cluster theory, widely used as one of the standard many-body methods in quantum chemistry.

\item Full configuration interaction (FCI) theory.

\item Many-body perturbation theory.

\item In-medium Similarity renormalization group (SRG) methods.

\item We have also developed diffusion and variational Monte Carlo codes for condensed matter systems and atomic and molecular physics systems.
\end{itemize}

\noindent
\paragraph{Strengths of the approaches.}
\begin{itemize}
\item If feasible (the limit is the exponential growth of basis states), full configuration interaction theory allows for numerically exact results and can be extended to time-dependent studies using approaches like multi-configuration time-dependent Hartree-Fock.

\item FCI provides benchmarks for approximative many-body methods like coupled cluster theory or in-medium SRG.

\item Coupled cluster (CC) theory and the in-medium SRG approches allow for systematic inclusions of more complicated correlations like three-particle-three-hole excitations.

\item CC theory and the in-medium SRG provide results close to full FCI benchmarks at a much lower numerical expenditure than FCI theory. 

\item The latter two methods can be extended to studies of time-dependencies and temperature dependence and the calculation of effective operators in a many-body medium in a systematic way. We are defining projects along these directions. 
\end{itemize}

\noindent
\paragraph{Challenges and collaborative overlaps.}
\begin{itemize}
\item In the nuclear many-body problem, effective three-body interactions (due to the effective representation of quark degrees of freedom in terms of nucleons and pions) play an important role and need to be included in the input Hamiltonian. This complicates the many-body problem. These correlations need to be included and represent considerable computational challenges t the above methods.

\item A proper understanding  of the truncation errors made both in terms of the truncation in the single-particle basis and the many-body excitations that are included is needed. There is much ongoing work on understanding how errors extrapolated to an infinite basis. Most many-body methods however cannot provide proper \emph{a priori} estimates of the errors made in truncating the type of many-body correlations to be handled. Most CC applications include only in a perturbative way (due to computational limitations) three-particle-three-hole excitations. 

\item We are presently developing our formalism in order to perform dense matter studies, of relevance for both neutron star physics and the physics of supernovae explosions. This requires studies of effective operators in dense matter and eventually the inclusion of temperature. This has important consequences for our understanding of the equation of state for dense matter as well as for the interaction of neutrinos with matter. A proper estimate of neutrino spectra in dense matter as has wide ranging consequences, from our understanding on how a neutron star cools to neutrino oscillations. 

\item The dense matter studies will be performed with a main emphasis on CC theory, in-medium SRG and Monte Carlo methods. The formalism we aim at developing can easily be used for studies of electronic systems like the homogeneous electron gas and quantum dots. 

\item These methods can also provide the basis for better constraints of DFT functionals.
\end{itemize}

\noindent
\paragraph{Selected References.}
\begin{enumerate}
\item G. Hagen et al, \emph{Charge, neutron, and weak size of the atomic nucleus}, Nature Physics, 12:186–190 (2016).

\item A. Ekstrom, G. R. Jansen, K. A. Wendt, G. Hagen, T. Papenbrock, B. D. Carlsson, C. Forssen, M. Hjorth-Jensen, P. Navratil, W. Nazarewicz, \emph{Accurate nuclear radii and binding energies from a chiral interaction}, Physical Review C, 91, 051301(R) (2015).

\item G. Hagen, T. Papenbrock, A. Ekstrom, G. Baardsen, S. Gandolfi, K. A. Wendt, M. Hjorth-Jensen, and C. Horowitz, \emph{Coupled-cluster calculations of nucleonic matter}, Physical Review C, 89:014319 (2014).

\item T. Papenbrock, G. Hagen, M. Hjorth-Jensen, and D. J. Dean, \emph{Coupled-cluster computations of atomic nuclei}, Reports on Progress in Physics, 77:096302 (2014).

\item G. Baardsen, A. Ekstrom, G. Hagen, and M. Hjorth-Jensen, \emph{Coupled-cluster studies of infinite nuclear matter}, Physical Review C, 88:054312 (2013).

\item M. Pedersen Lohne, G. Hagen, M. Hjorth-Jensen, S. Kvaal, and F. Pederiva, \emph{Ab initio calculations of Circular quantum dots}. Physical Review B, 84:032501, 2011.

\item E. Bergli and M. Hjorth-Jensen, \emph{Summation of Parquet diagrams as an ab initio method in nuclear structure calculations}, Annals of Physics, 326:1125, 2011.
\end{enumerate}

\noindent

% ------------------- end of main content ---------------

% #ifdef PREAMBLE
\end{document}
% #endif

