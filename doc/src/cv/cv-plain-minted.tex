%%
%% Automatically generated file from DocOnce source
%% (https://github.com/doconce/doconce/)
%% doconce format latex cv.do.txt --minted_latex_style=trac --latex_admon=paragraph --no_mako
%%


%-------------------- begin preamble ----------------------

\documentclass[%
oneside,                 % oneside: electronic viewing, twoside: printing
final,                   % draft: marks overfull hboxes, figures with paths
10pt]{article}

\listfiles               %  print all files needed to compile this document

\usepackage{relsize,makeidx,color,setspace,amsmath,amsfonts,amssymb}
\usepackage[table]{xcolor}
\usepackage{bm,ltablex,microtype}

\usepackage[pdftex]{graphicx}

\usepackage[T1]{fontenc}
%\usepackage[latin1]{inputenc}
\usepackage{ucs}
\usepackage[utf8x]{inputenc}

\usepackage{lmodern}         % Latin Modern fonts derived from Computer Modern

% Hyperlinks in PDF:
\definecolor{linkcolor}{rgb}{0,0,0.4}
\usepackage{hyperref}
\hypersetup{
    breaklinks=true,
    colorlinks=true,
    linkcolor=linkcolor,
    urlcolor=linkcolor,
    citecolor=black,
    filecolor=black,
    %filecolor=blue,
    pdfmenubar=true,
    pdftoolbar=true,
    bookmarksdepth=3   % Uncomment (and tweak) for PDF bookmarks with more levels than the TOC
    }
%\hyperbaseurl{}   % hyperlinks are relative to this root

\setcounter{tocdepth}{2}  % levels in table of contents

% --- fancyhdr package for fancy headers ---
\usepackage{fancyhdr}
\fancyhf{} % sets both header and footer to nothing
\renewcommand{\headrulewidth}{0pt}
\fancyfoot[LE,RO]{\thepage}
% Ensure copyright on titlepage (article style) and chapter pages (book style)
\fancypagestyle{plain}{
  \fancyhf{}
  \fancyfoot[C]{{\footnotesize \copyright\ 1999-2022, Morten Hjorth-Jensen. Released under CC Attribution-NonCommercial 4.0 license}}
%  \renewcommand{\footrulewidth}{0mm}
  \renewcommand{\headrulewidth}{0mm}
}
% Ensure copyright on titlepages with \thispagestyle{empty}
\fancypagestyle{empty}{
  \fancyhf{}
  \fancyfoot[C]{{\footnotesize \copyright\ 1999-2022, Morten Hjorth-Jensen. Released under CC Attribution-NonCommercial 4.0 license}}
  \renewcommand{\footrulewidth}{0mm}
  \renewcommand{\headrulewidth}{0mm}
}

\pagestyle{fancy}


% prevent orhpans and widows
\clubpenalty = 10000
\widowpenalty = 10000

% --- end of standard preamble for documents ---


% insert custom LaTeX commands...

\raggedbottom
\makeindex
\usepackage[totoc]{idxlayout}   % for index in the toc
\usepackage[nottoc]{tocbibind}  % for references/bibliography in the toc

%-------------------- end preamble ----------------------

\begin{document}

% matching end for #ifdef PREAMBLE

\newcommand{\exercisesection}[1]{\subsection*{#1}}


% ------------------- main content ----------------------



% ----------------- title -------------------------

\thispagestyle{empty}

\begin{center}
{\LARGE\bf
\begin{spacing}{1.25}
Biographical  information
\end{spacing}
}
\end{center}

% ----------------- author(s) -------------------------

\begin{center}
{\bf Morten Hjorth-Jensen${}^{1, 2}$} \\ [0mm]
\end{center}

\begin{center}
% List of all institutions:
\centerline{{\small ${}^1$Department of Physics, University of Oslo, Norway}}
\centerline{{\small ${}^2$Department of Physics and Astronomy and Facility for Rare Ion Beams/National Superconducting Cyclotron Laboratory, Michigan State University, USA}}
\end{center}
    
% ----------------- end author(s) -------------------------

% --- begin date ---
\begin{center}
April 2021
\end{center}
% --- end date ---

\vspace{1cm}


\subsection*{Professional preparation, education  and personal data:}

\begin{itemize}
\item Professor of Physics at Michigan State University, USA and the University of Oslo, Norway

\item Norwegian citizen, born in Haugesund, Norway, permanent resident of the USA

\item Norwegian University of Science and Technology, Trondheim, Norway,  Siv.Ing. in Theoretical Physics (Master of Science equivalent),  1988 

\item University of Oslo, Norway,  Ph.D in Theoretical Nuclear Physics, 1993

\item ECT*, Trento, Italy,  Postdoctoral Researcher in Theoretical Nuclear Physics,  1994-1996

\item Nordita, Copenhagen, Denmark, Postdoctoral Researcher in Theoretical Nuclear Physics, 1996-1998
\end{itemize}

\noindent
\subsection*{Appointments:}

\begin{enumerate}
\item Associate Professor of Physics, University of Oslo, 1999-2001;

\item Professor of Physics, University of Oslo,2001-present;

\item Adjunct Professor of Physics, Michigan State University, 2003-2011;

\item Professor of Physics, Michigan State University,2012-present;
\end{enumerate}

\noindent
\subsection*{Brief research overview}

I am a theoretical physicist with a strong interest in 
computational physics, computational science and many-body theory in general, and 
the nuclear many-body problem and nuclear structure problems in particular. 
This means that I study various methods for solving either Schrödinger's equation or 
Dirac's equation for many interacting particles, spanning from 
algorithmic aspects to the mathematical properties of such methods, including machine learning and quantum computing. 

\subsection*{Awards:}

\begin{enumerate}
\item \href{{http://www.uniforum.uio.no/nyheter/2000/11/det-viktigste-er-aa-inspirere.html}}{University of Oslo award for excellence in teaching}, 2000 (250kNOK)

\item Fellow of the American Physical Society, 2007

\item Oak Ridge National Laboratory excellence in research award, 2008

\item Outstanding referee award of the American Physical Society, 2008

\item \href{{http://www.uniforum.uio.no/nyheter/2011/08/undervisning-for-framtidig-forsking.html}}{University of Oslo award for excellence in teaching} for the \textbf{Computing in Science Education} project, 2011 (250kNOK)

\item NOKUT (Norwegian entity of quality assessment in higher education) \href{{http://www.uniforum.uio.no/nyheter/2012/04/uio-tok-andreplass-i-utdanningskvalitet.html}}{award for excellence in teaching} for the \textbf{Computing in Science Education} project, 2012

\item Elected member of the Norwegian Academy of Sciences and Letters, 2013

\item Elected member of the Royal Norwegian Society of Sciences and Letters, 2015 

\item \href{{http://www.uniforum.uio.no/nyheter/2015/10/instituttet-som-lofter-fram-gode-forelesere.html}}{University of Oslo award for excellence in teaching} for developing the Computational Physics group, 2015 (250kNOK)

\item Favorite graduate teacher at the Department of Physics and Astronomy at Michigan State University, 2016 

\item \textbf{Olav Thon Foundation} \href{{https://www.ntbinfo.no/pressemelding/olav-thon-stiftelsen-annonserte-arets-priser-42-millioner-til-forskning-og-undervisning?publisherId=8983491&releaseId=16475069}}{National prize for excellence in teaching award} (National, all Norwegian higher education institutions, 500kNOK), 2018

\item \href{{https://web.pa.msu.edu/alumni/awards/osgood_fac_awards.html}}{Thomas H. Osgood Faculty Teaching award at Michigan State University, 2018}

\item \href{{https://titan.uio.no/utdanning/2020/to-fysikere-belonnes-for%20undervisningen-sin}}{University of Oslo merited teacher award 2020}

\item \href{{https://natsci.msu.edu/faculty-staff/awards/#eta}}{College of Natural Science Norman L and Olga K. Fritz Excellence in Teaching Award, Michigan State University, 2021}
\end{enumerate}

\noindent
\subsection*{Citation metrics, highly cited articles, and additional research highlights:}

\begin{enumerate}
\item \href{{https://publons.com/researcher/1751939/morten-hjorth-jensen/}}{Web of Science} and \href{{https://scholar.google.com/citations?user=nuiyEmwAAAAJ&hl=no}}{Google Scholar}.

\item \textbf{Realistic effective interactions for nuclear systems}, M Hjorth-Jensen, TTS Kuo, E Osnes, \href{{http://www.sciencedirect.com/science/article/pii/0370157395000126}}{Physics Reports 261, 125-270 (1995)}.

\item \textbf{Phases of dense matter in neutron stars}, H Heiselberg, M Hjorth-Jensen, \href{{http://www.sciencedirect.com/science/article/pii/S0370157399001106}}{Physics Reports 328, 237-327 (2000)}.

\item \textbf{Pairing in nuclear systems: from neutron stars to finite nuclei}, DJ Dean, M Hjorth-Jensen, \href{{http://journals.aps.org/rmp/abstract/10.1103/RevModPhys.75.607}}{Reviews of Modern Physics 75, 607  (2003)}.

\item A total of more than 150  peer reviewed articles and three books.

\item Authored and co-authored 23 Physical Review Letters articles, 17 Rapid communications in Physical Review C, seven Physics Letters B articles, one Astrophysical Journal Letters article and one Nature Physics article

\item Written two Physics viewpoints and been highlighted in one other.

\item Taught and developed several courses in Computational Physics and many-body physics, courses in nuclears structure and quantum physics and mechanics and statistical mechanics.

\item More than two hundred invited talks, seminars, colloquia and lectures given worldwide.

\item Organized more than 30 conferences, workshops and schools and advanced courses.

\item Supervised and co-supervised more than 100 graduate students (Master of Science and PhD) and post-doctoral fellows

\item Presently supervising and co-supervising  eleven Master of Science students (University of Oslo) and five PhD students at MSU and three PhD students at the University of Oslo.
\end{enumerate}

\noindent
\subsection*{Synergistic Activities and service through the years:}

\begin{itemize}
\item Since 1999 I have   established an activity in computational physics  at the  Department of Physics at the University of Oslo. I have also started from scratch and developed several  courses on computational physics and computational science and many-body physics. This activity was recognized with the Excellence in Teaching award from the University of Oslo in 2015. During the last 20 years I have guided  more than 100 gradute students (at the Master of Science and PhD levels) and post-doctoral fellows at Michigan State University and the University of Oslo.

\item With colleagues at the University of Oslo, I have   been strongly involved in the development of a totally new teaching philosophy which merges computations with the traditional science and mathematics curriculum . This project is called \href{{http://www.mn.uio.no/english/about/collaboration/cse/}}{Computing in Science Education} and has received considerable support from the University of Oslo and the Norwegian Ministry of research and education.  It received the University of Oslo award for excellence in teaching  in 2011 and the NOKUT award in 2012. 

\item With colleagues from the USA and other European countries, we started in 2010 the \href{{http://www.nucleartalent.org}}{Nuclear Talent initiative},  where we aim  at providing an advanced and comprehensive training to graduate students and young researchers in low-energy nuclear theory.  The network aims at developing a broad curriculum that will provide the platform for a cutting-edge theory for understanding nuclei and nuclear reactions.  

\item I initiated and lead the new \href{{http://www.uio.no/english/studies/programmes/computational-science-master/}}{Master of Science program on Computational Science at the University of Oslo}. This is a new and multi-disciplinary program across several disciplines at the College of Natural Science of the University of Oslo. It includes now seven  departments at the faculty of Mathematics and Natural Sciences of  the University of Oslo.
\end{itemize}

\noindent
\paragraph{Editorial boards and committees.}
\begin{itemize}
\item \href{{http://ectstar.eu/}}{Board member of the European Center for Theoretical Studies in Nuclear Physics and Related Areas, Trento, Italy, 2017-present}

\item Member of the Physics Advisory Comittee at the National Superconducting Cyclotron Laboratory, Michigan State University, East Lansing, USA, 2003-2008

\item Member of the Canadian research council's evaluation board on subatomic physics 2012-2015.

\item Member of the Swedish research council's evaluation board on subatomic physics 2007-2008.

\item Editorial Board member of Physical Review C, 2014-2016

\item Editorial Board member of European Physical Journal A, 2010-2016

\item Editorial Board member of European Physical Journal Special Topics, 2010-present

\item Editorial Board member of Springer's Lecture Notes in Physics, 2010-present

\item Editorial Board member of Springer's Undergraduate Lecture Notes in Physics, 2014-present

\item Editorial Board member of Springer's University Texts  in Physics, 2015-present

\item Editorial Board member of Springer's Undergraduate Texts  in Physics, 2016-present

\item Editorial Board member of Springer's Graduate Texts  in Physics, 2018-present

\item Editorial Board member of Computers in Science and Discovery journal, a journal by IOP, UK, 2008-2014

\item \href{{http://fribtheoryalliance.org/}}{Steering Committee member of the FRIB theory alliance at Michigan State University 2013-2016}

\item \href{{http://www.nucleartalent.org/}}{Initiated and led the Nuclear Talent initiative from 2010 till 2015, member of the Steering committee} till end of 2019.

\item Member of the Board of Usit at UiO (Center for information technology at the University of Oslo), 2002-2004

\item Project leader for High-performance computing courses at UiO, 2000-2003

\item Board member of the Bachelor program Mathematics, Information theory and Technology at the University of Oslo, 2002-2008

\item Leader of the Bachelor program Physics, Astronomy and Meteorology at the University of Oslo, 2002-2011

\item Together with colleagues from the Department of Physics, Department of Mathematics and Department of Informatics at the University of Oslo, we started   the Computers in Science Education project in 2004. This project, which we conceived back in 2003,  has changed totally changed the way Science is taught.

\item Member of the OECD working group on nuclear physics 2006-2008

\item January 2009-December 2011, leader of the Nuclear Physics group at the University of Oslo

\item Leader the new \href{{http://www.uio.no/english/studies/programmes/computational-science-master/}}{Master of Science program on Computational Science at the University of Oslo}. This is a new and multi-disciplinary program across several disciplines at the College of Natural Science of the University of Oslo. 
\end{itemize}

\noindent
\paragraph{Referee for International Journals.}
\begin{itemize}
\item Referee for Reviews of Modern Physics

\item Referee for Physical Review Letters

\item Referee for Nature

\item Referee for Physical Review \textbf{C}

\item Referee for Physical Review \textbf{D}

\item Referee for Nuclear Physics \textbf{A}

\item Referee for Physics Letters \textbf{B}

\item Referee for Astrophysical Journal

\item Referee for Journal of Chemical Physics

\item Referee for Journal of Physics \textbf{A}: Mathematical Physics

\item Referee for Journal of Physics \textbf{G}: Nuclear and Particle Physics

\item Referee for European Journal of Physics \textbf{A}

\item Referee for European Physics Letters

\item Referee for Few Body Systems

\item Referee for Modern Journal of Physics E

\item Referee for Physica Scripta

\item Referee for Annals of Physics

\item Referee for SIAM

\item Referee for Computer Physics Communications

\item Referee for Computers in Science and Discovery

\item Referee for Journal of Mathematics Physics

\item Referee for Progress in Theoretical Physics

\item Referee for Polish Journal of Physics
\end{itemize}

\noindent
\paragraph{Other Referee Activities.}
\begin{itemize}
\item Referee for the Canadian Research Council

\item Referee for the Israeli Research Council

\item Referee for the South African Research Council

\item Referee for the British Research Council

\item Referee for the German Research Council

\item Referee for the American Department of Energy (DOE)

\item Referee for the American National Science Foundation (NSF)

\item Referee for INFN, Istituto Nazionale di Fisica Nucleare, Italy

\item Referee for ESF, European Science Foundation

\item Referee for Vetenskapsrådet, the Swedish Research Council

\item Referee for the Danish Resource Council

\item Referee for the Serbian Research Ministry

\item Referee for the Russian Research Council

\item Referee for the Research Council of Luxembourg

\item Opponent at several PhD dissertations.

\item Member of several  PhD guidance committees at Michigan State University

\item Several expert evaluations on promotion applications.

\item Member or leader of several job assessment committees in Norway and the USA
\end{itemize}

\noindent
\paragraph{Member of International Advisory committees.}
\begin{enumerate}
\item 22nd International Few-Body Conference, member of IAC 2018

\item International Nuclear Physics Conference, member of IAC since 2008

\item Nuclear Structure 2010 and 2014, member of IAC

\item Program Advisor Committee for Recent Progress in Many-Body Theories, member since 2007

\item Scientific advisory committee for Nuclear Theory in the Supercomputing Era 

\item International Advisory committee of International Conference on Mathematical Modeling in Physical Sciences

\item International Advisory committee for XI Latin American Symposium on Nuclear Physics and Applications 

\item International Advisory Board for Conference on Computational Physics

\item International Advisory committee for EURORIB15 and EURORIB18

\item International Advisory committee for SIAM conference on Computational Science and Engineering in Boston, 2013
\end{enumerate}

\noindent
\paragraph{Member of Graduate Advisory Committees at Michigan State University.}
I am (have been) member or chair person of the following graduate  student committees at Michigan State University:
\begin{enumerate}
\item Justin Lietz, chair, defended thesis June 2019.

\item Fei Yuan, chair.  Defended thesis January 24 2018.

\item Sam Novario, chair. Defends thesis February 7 2018.

\item John Bower, chair together with Scott Bogner. Master of Science thesis May 2017.  

\item Adam Jones, committee member. Master of Science thesis July 2017.  

\item Chris Sullivan, committee member. Defended thesis January 2018.

\item Thomas Redpath, committee member. Defended thesis October 2019.

\item Sean Sweany, committee member, defends thesis fall 2020.

\item Rachel Taverner, committee member. Defended thesis May 2019.

\item Nathan Parzuchowski, committee member. Defended thesis April 2017.

\item Titus Morris, committee member. Defended thesis May 2016

\item Kenneth Whitmore, committee member. Defended thesis June 2016

\item Alex Dombos, committee member. Defended thesis May 2018.

\item Josh Bradt, committee member, Defended thesis July 2017.

\item Charles Loelius, committee member, Defended thesis May 2017.

\item Safwan Shanab, committee member. Defended thesis January 2020.

\item Hao Lin, committee member. Defended thesis July 2020.

\item Mao Xingze, committee member. Defended thesis July 2020.

\item Amy Lovell, committee member. Defended thesis January 24 2018.

\item Debra Richman, committee member, defends thesis December 2020.

\item Roy Ready, committee member. Defended thesis May 2021.

\item Nathan Watwood, committee member. Defended thesis February 2021. 

\item Ben Hall, chair, thesis defense planned 2022

\item Udiani Omokuyani, committee member

\item Jane Kim, chair, thesis defense planned 2023

\item Khan Zhu, committee member. 

\item Byeon Heejun, committee member. Defended thesis December 2020.

\item Golubev Timofey, committee member. Defended thesis December 2020.

\item Hermansen, Kirby, committee member

\item Watkins Jacob, , committee member

\item Hill Matthew Steven, committee member
\end{enumerate}

\noindent
\subsection*{Courses, study programs and educational initiatives}

I am strongly involved in teaching at all levels. I have been heading
the bachelor program Physics, Astronomy and Meteorology ( FAM ) in the
period 2002-2011. I am also strongly involved in the project Computing
in Science Education. Furthermore, with European and American
colleagues, we have established the recent successful Nuclear Talent
initiative.

Since 1999 I have established an activity in computational physics at
the Department of Physics at the University of Oslo. I have also
started from scratch and developed several courses on computational
physics, machine learning and many-body physics. This activity was
recognized with the Excellence in Teaching award from the University
of Oslo in 2015. During the last twenty years I have guided more than
100 graduate students (Master of Science and PhD levels) and post-doctoral
fellows.

With colleagues at the University of Oslo, I have been strongly
involved in the development of a totally new teaching philosophy which
merges computation with the traditional science amd mathematics
curriculum . This project is called \href{{http://www.mn.uio.no/english/about/collaboration/cse/}}{Computing in Science
Education} and
has received considerable support from the University of Oslo and the
Norwegian Ministry of research and education.  It received the
University of Oslo award for excellence in teaching in 2011 and the
NOKUT award in 2012.

With colleagues from the USA and other European countries, we 
started in 2010 the Nuclear Talent initiate":"http://www.nucleartalent.org",
where we aim at providing an advanced and comprehensive training to
graduate students and young researchers in low-energy nuclear theory.
The network aims at developing a broad curriculum that will provide
the platform for a cutting-edge theory for understanding nuclei and
nuclear reactions. Over the years I have taught and organized several such intensive courses (see list below). 

I initiated in 2015 and chair the new \href{{http://www.uio.no/english/studies/programmes/computational-science-master/}}{Master of Science program on Computational Science at the University of Oslo}. This is a new and multi-disciplinary program across several disciplines at the College of Natural Science of the University of Oslo. 

I teach or have taught recently  the following courses at the University of Oslo and Michigan State University:

\begin{itemize}
\item \href{{http://www.uio.no/studier/emner/matnat/fys/FYS3150/}}{FYS3150/4150 Computational Physics I}, Fall semester, senior undergraduate level (Oslo) 

\item \href{{http://www.uio.no/studier/emner/matnat/fys/FYS4411/}}{FYS4411 Computational Physics II: Quantum mechanical systems}, graduate level, Spring semester (Oslo) 

\item \href{{http://www.uio.no/studier/emner/matnat/fys/FYS-kJM4480/}}{FYS-KJM4480 Quantum mechanics for many-particle systems}, graduate level, Fall semester (Oslo) 

\item \href{{https://github.com/NuclearStructure/PHY981}}{PHY981 Nuclear Structure}, graduate level, Spring semester (MSU) 

\item \href{{https://github.com/CompPhysics/ComputationalPhysicsMSU}}{PHY480/905 Computational Physics} (MSU), undergraduate and graduate level, Spring semester
\end{itemize}

\noindent
From the fall of 2018 I have developed and teach the new course on \textbf{Applied Data analysis and Machine Learning} at the University of Oslo. This course is a compulsory course that is part of the new interdisciplinary Master of Science program \href{{http://www.uio.no/english/studies/programmes/computational-science-master/index.html}}{Computational Science}. The link to the course is
\begin{itemize}
\item \href{{http://www.uio.no/studier/emner/matnat/fys/FYS-MAT4155/}}{FYS-MAT3155/4155 Data Analysis and Machine Learning}, senior undergraduate and graduate level, Fall semester (Oslo) 

\item \href{{https://github.com/mhjensen/Physics321}}{PHY321 Classical Mechanics, MSU}, undergraduate course, spring semester. First time spring 2020. 
\end{itemize}

\noindent
I have also taught introductory quantum physics, FYS2140, 2000-2004, Statistical Mechanics, FYS4130, 1990-1994 and I have developed an advanced course on \href{{http://www.uio.no/studier/emner/matnat/fys/FYS-kjm4480/}}{FYS-KJM4480 Quantum mechanics for many-particle systems}, 2009-2014, all at the at the University of Oslo, Norway. At Michigan I have also taught an advanced course in Nuclear Structure Physics PHYS981 Nuclear Structure, graduate level, Spring semester, 2013-2016. In addition, with Scott Bogner at Michigan State University, we taught a specialized course on Nuclear Force, PHY989, during the fall semester of 2018.

\paragraph{Teaching Awards:}
\begin{enumerate}
\item \href{{http://www.uniforum.uio.no/nyheter/2000/11/det-viktigste-er-aa-inspirere.html}}{University of Oslo award for excellence in teaching}, 2000 (250kNOK)

\item \href{{http://www.uniforum.uio.no/nyheter/2011/08/undervisning-for-framtidig-forsking.html}}{University of Oslo award for excellence in teaching} for the \textbf{Computing in Science Education} project, 2011 (250kNOK)

\item NOKUT (Norwegian entity of quality assessment in higher education) \href{{http://www.uniforum.uio.no/nyheter/2012/04/uio-tok-andreplass-i-utdanningskvalitet.html}}{award for excellence in teaching} for the \textbf{Computing in Science Education} project, 2012

\item \href{{http://www.uniforum.uio.no/nyheter/2015/10/instituttet-som-lofter-fram-gode-forelesere.html}}{University of Oslo award for excellence in teaching} for developing the Computational Physics group, 2015 (250kNOK)

\item Favorite graduate teacher at the Department of Physics and Astronomy at Michigan State University, 2016 

\item \textbf{Olav Thon Foundation} \href{{https://www.ntbinfo.no/pressemelding/olav-thon-stiftelsen-annonserte-arets-priser-42-millioner-til-forskning-og-undervisning?publisherId=8983491&releaseId=16475069}}{National prize for excellence in teaching award} (National, all Norwegian higher education institutions, 500kNOK), 2018

\item \href{{https://web.pa.msu.edu/alumni/awards/osgood_fac_awards.html}}{Thomas H. Osgood Faculty Teaching award at Michigan State University, 2018}

\item \href{{https://titan.uio.no/utdanning/2020/to-fysikere-belonnes-for%20undervisningen-sin}}{University of Oslo merited teacher award 2020}
\end{enumerate}

\noindent
\paragraph{Present PhD students.}
\begin{enumerate}
\item Benjamin Hall, Michigan State University, started 2018. Research topic: Quantum Computing  and the Nuclear Many-body problem

\item Jane Kim, Michigan State University, started 2018. Research topic: Machine Learning and the Nuclear Many-body problem

\item Julie Butler, Michigan State University, started 2018. Research topic: Machine Learning and the Nuclear Many-body problem

\item Øyvind Sigmundsson Schøyen, University of Oslo, started 2019. Research topic: Time-dependent many-body theory and quantum computing

\item Stian Bilek, University of Oslo, started 2020, defends thesis September 2024. Quantum Computing  and Machine Learning

\item Jonas Boym Flaten, University of Oslo, started 2020, defends thesis December 2024. Quantum Many-Body theories

\item Omokuyani C. Udiani , Michigan State University, started 2017, co-supervisor. Research topic: Nuclear Many-body theory

\item Danny Jammoa, Michigan State University, started 2020, co-supervisor. Research topic: Quantum Computing and Machine Learning

\item Paulina Souza Tedesco, University of Oslo, started 2020, defends thesis fall 2023, Machine Learning and Meteorology, co-supervisor

\item Katarzyna Michałowska, University of Oslo, started 2020, defends thesis fall 2023, Machine Learning, co-supervisor

\item Einar Aurbakken, University of Oslo, started 2020, defends thesis fall 2024, Quantum Chemistry and Many-body Physics, co-supervisor
\end{enumerate}

\noindent
\paragraph{Present Master of Science Students.}
\begin{enumerate}
\item Eina Jørgensen, University of Oslo, (2019-2021), co-supervisor

\item Morten Hemmingsen, University of Oslo, (2019-2021), co-supervisor

\item Huying Zhang, University of Oslo, (2019-2021), co-supervisor

\item Jens Due Bratten, University of Oslo, (2019-2021), co-supervisor

\item Gabriel Cabrera, University of Oslo, (2019-2021), co-supervisor

\item Kristian Wold, University of Oslo, (2019-2021)

\item Martin Krokan Hovden, University of Oslo, (2019-2021)

\item Johan Nereng, University of Oslo, (2019-2021)

\item Oliver Hebnes, University of Oslo, (2019-2021), co-supervisor

\item Mohamad Ismail, University of Oslo, (2019-2021), co-supervisor

\item Kristoffer Langstad, University of Oslo, (2019-2021), co-supervisor
\end{enumerate}

\noindent
\paragraph{Former PhD students and their present positions.}
\begin{enumerate}
\item John Mark Aiken, University of Oslo, started 2017, defended thesis September 2020, co-supervisor. Research Topic: Machine Learning applied to Physics Education Research. Now post-doctoral fellow at the University of Minnesota, Minneapolis.

\item Justin Lietz (PhD MSU 2019), now post-doctoral fellow at Oak Ridge National Laboratory, Computational Science Division

\item Samuel Novario (PhD MSU 2018), post-doctoral fellow at Oak Ridge National Laboratory, Physics Division, 2018-2020, now post-doctoral fellow at Los Alamos National Laboratoty

\item Fei Yuan (PhD MSU 2018), employed at Google as computational scientist

\item \href{{http://www.ctcc.no/people/postdocs/gba/}}{Gustav Baardsen} (PhD UiO 2014). From 2015 to 2018, ost-doctoral researcher at the Center for Theoretical and Computational Chemistry (CTCC), University of Oslo. Now employed by Varian Medical Systems, Helsinki, Finland.

\item \href{{http://www.mn.uio.no/kjemi/english/people/aca/simenkv/index.html}}{Simen Kvaal} (PhD UiO 2009), researcher, Department of Chemistry, University of Oslo. Recipient of an ERC starting grant

\item \href{{https://www.ornl.gov/staff-profile/gustav-r-jansen}}{Gustav Jansen} (PhD UiO 2012), now permanent position as scientist at the Computational Science Division of Oak Ridge National Laboratory  

\item \href{{http://www.mn.uio.no/math/english/people/aca/tmac/}}{Torquil MacDonald Sørensen} (PhD UiO 2012), post-doctoral fellow at the Department of Mathematics, UiO

\item \href{{http://www.usit.uio.no/english/about/organisation/bps/rc/ris/staff/jonkni/}}{Jon Kerr Nilsen} (PhD UiO 2010), senior engineer at the University of Oslo center for information technologies (co-supervisor)

\item \href{{https://www.hioa.no/tilsatt/marlys}}{Marius Lysebo} (PhD UiO 2010), now Associate Professor at Oslo University College, (co-supervisor)

\item \href{{http://www.aas.vgs.no/om-oss/organisasjon/alle-ansatte/}}{Elise Bergli} (PhD UiO 2010), teacher Ås high school, Norway and Assistant Professor at the Norwegian University of Life Sciences.

\item \href{{https://www.hbv.no/om-hbv-kontakt-oss-ansatte/eirik-ovrum-article125026-6688.html}}{Eirik Ovrum} (PhD UiO 2007), now Associate Professor at the University College of Southeast of Norway

\item \href{{https://www.ornl.gov/staff-profile/gaute-hagen}}{Gaute Hagen} (PhD UiB and UiO 2005), now permanent position as scientist at the Physics Division of Oak Ridge National Laboratory. Recipient of the Department of Energy Early career award

\item Øystein Elgarøy (PhD UiO 1999), now professor of Theoretical Astrophysics at the University of Oslo, Norway (co-supervisor)

\item Lars Engvik (PhD UiO 1999), now Associate Professor at Sør-Trøndelag University College, Trondheim, Norway, (co-supervisor)
\end{enumerate}

\noindent
\paragraph{Post-doctoral fellows and their present positions.}
\begin{enumerate}
\item \href{{https://www.chalmers.se/en/Staff/Pages/Andreas-Ekstrom.aspx}}{Andreas Ekstrøm} (UiO and MSU 2010-2014), now Associate Professor  at Chalmers Technological University in Gothenburg, Sweden

\item Øyvind Jensen (UiO 2011), now researcher at the \href{{https://www.ife.no/en}}{Institute for Energy Technology}

\item \href{{http://www.mn.uio.no/kjemi/english/people/aca/simenkv/index.html}}{Simen Kvaal} (UiO 2008-2012), researcher,  Department of Chemistry, University of Oslo. Recipient of an ERC starting grant

\item Elise Bergli (UiO 2010-2011), now teacher at Ås high school, Norway

\item Sølve Selstø (UiO 2008-2010), now  Professor at Oslo Metropolitan University

\item Nicolas Michel (MSU 2013), now senior researcher at Langzhou Nuclear Physics Laboratory, China
\end{enumerate}

\noindent
\paragraph{Former Master of Science Students(links to their thesis will be added).}
\begin{enumerate}
\item Heine Aabø, University of Oslo, (2018-2020)

\item Stian Bilek, University of Oslo, (2018-2020)

\item Thomas Sjåstad, University of Oslo, (2018-2020), co-supervisor

\item Eirik Thorsrud, University of Oslo, (2018-2020), co-supervisor

\item Halvard Sutterud, University of Oslo, (2018-2020)

\item Marius Holm, University of Oslo, (2018-2020), co-supervisor

\item Geir Utvik, University of Oslo, (2018-2020)

\item Markus Aspurusten, University of Oslo, (2018-2020), co-supervisor

\item Vebjørn Gilberg, University of Oslo, (2017-2020), co-supervisor

\item Kari Eriksen, University of Oslo, (2017-2020)

\item Robert Solli, University of Oslo, (2017-2019)

\item Andreas Lefdalsnes, University of Oslo, (2017-2019)

\item Joseph Knutson, University of Oslo, (2017-2019)

\item Bendik Samseth, University of Oslo, (2017-2019)

\item Even Nordhagen, University of Oslo, (2017-2019)

\item Øyvind Schøyen Sigmundson, University of Oslo, (2017-2019)

\item Sebastian Gregorius Winther-Larsen, University of Oslo, (2017-2019)

\item Giovanni Pederiva, University of Oslo, (2016-2018), co-supervisor

\item Anna Gribovskaya, University of Oslo, (2016-2018)

\item Andrei Kucharenka, University of Oslo, (2016-2018)

\item Vilde Moe Flugsrud, University of Oslo, (2016-2018)

\item Alfred Alocias Mariadason, University of Oslo, (2016-2018)

\item Marius Jonsson, University of Oslo, (2016-2018)

\item Hans Mathias Vege Mamen, University of Oslo, (2016-2019), co-supervisor

\item Alexander Fleischer, University of Oslo, (2015-2017)

\item Håkon Emil Kristiansen, University of Oslo, (2015-2017)

\item Morten Ledum, University of Oslo, (2015-2017)

\item Håkon Treider Vikør, University of Oslo, (2015-2017), co-supervisor

\item Jon-Andreas Stende, University of Oslo, (2015-2017), co-supervisor

\item Sean Bruce Sangholt Miller, University of Oslo, (2015-2017)

\item Christian Fleischer, University of Oslo, (2015-2017)

\item John Bower, Michigan State University, (2014-2017)

\item Wilhelm Holmen, University of Oslo (2014-2016)

\item Roger Kjøde, University of Oslo, (2014-2016)

\item Håkon Sebatian Mørk, University of Oslo, (2014-2016)

\item Jonas van den Brink, University of Oslo, (2014-2016), co-supervisor

\item Marte Julie Sætra, University of Oslo, (2014-2016), co-supervisor

\item Audun Skau Hansen, University of Oslo, (2013-2015)

\item Henrik Eiding, University of Oslo, (2012-2014)

\item Svenn-Arne Dragly, University of Oslo, (2012-2014)

\item Milad Hobbi Mobarhan, University of Oslo, (2012-2014)

\item Ole Tobias Norli, University of Oslo, (2012-2014)

\item Filip Sand, University of Oslo, (2012-2014), co-supervisor

\item Emilie Fjørner, University of Oslo, (2012-2014), co-supervisor

\item Jørgen Høgberget, University of Oslo, (2011-2013)

\item Sarah Reimann, University of Oslo, (2011-2013)

\item Karl Leikganger, University of Oslo, (2011-2013)

\item Sigve Bøe Skattum, University of Oslo, (2011-2013)

\item Veronica Berglyd Hansen, University of Oslo, (2010-2012)

\item Camilla Nestande Kirkemo, University of Oslo, (2010-2012), co-supervisor

\item Christoffer Hirth, University of Oslo, (2009-2011)

\item Marte Hoel Jørgensen, University of Oslo, (2009-2011)

\item Yang Min Wang, University of Oslo, (2009-2011)

\item Ivar Nikolaisen, University of Oslo, (2009-2011)

\item Vegard Amundsen, University of Oslo, (2008-2010)

\item Håvard Sandsdalen, University of Oslo, (2008-2010)

\item Lars Eivind Lervåg, University of Oslo, (2008-2010)

\item Magnus Lohne Pedersen, University of Oslo, (2008-2010)

\item Simen Sørby, University of Oslo, (2008-2010), co-supervisor

\item Sigurd Wenner, University of Oslo, (2008-2010), co-supervisor

\item Lene Norderhaug Drøsdal, University of Oslo, (2007-2009)

\item \href{{https://www.nilu.no/OmNILU/Kontaktoss/Ansatte/tabid/70/ctl/EmployeeDetails/mid/972/employeeid/5822/tabmoduleid/2333/language/en-GB/Default.aspx}}{Islen Vallejo, University of Oslo, (2007-2009)}, works at the Norwegian Institute for Air Research

\item Jacob Kryvi, Norwegian University of Science and Technology, (2007-2009), co-supervisor

\item Rune Albrigtsen, University of Oslo, (2007-2009)

\item Johannes Rekkedal, University of Oslo, (2007-2009)

\item Patrick Merlot, University of Oslo, (2007-2009)

\item Gustav Jansen, University of Oslo, (2006-2008)

\item Ole Petter Harbitz, University of Oslo, (2006-2008)

\item Sutharsan Amurgian, University of Oslo, (2005-2007)

\item Jon Thonstad, University of Oslo, (2005-2007)

\item Espen Flage-Larsen, University of Oslo, (2003-2005)

\item Joachim Berdahl Haga, University of Oslo, (2004-2006)

\item Jon Kerr Nilsen, University of Oslo, (2002-2004)

\item Simen Kvaal, University of Oslo, (2002-2004)

\item Simen Reine Sommerfelt, University of Oslo, (2002-2004)

\item Mateuz Marek Røstad, University of Oslo, (2002-2004)

\item Victoria Popsueva, University of Oslo, (2002-2004)

\item Eivind Brodal, University of Oslo, (2001-2003)

\item Eirik Ovrum, University of Oslo, (2001-2003)

\item Ronny Kjelsberg, Norwegian University of Science and Technology, (2001-2003)
\end{enumerate}

\noindent
\paragraph{Lectures and organization of schools:}
\begin{enumerate}
\item Morten Hjorth-Jensen, Daniel Bazin, Sean Liddick, , Michelle Kuchera, and R. Ramanujan, Online Nuclear Talent course on  Machine Learning Applied to Nuclear Physics, European Center for Theoretical Nuclear Physics and Related Areas, Trento, Italy, July 19 to July 30, 2021. Main organizer and teacher.

\item Morten Hjorth-Jensen, \textbf{2021 CHPC Introductory Programming Summer School}, South Africa, February 1-28, 2021, \href{{https://compphysics.github.io/MLSummerSchool/doc/web/course.html}}{five lectures on Machine Learning}. 

\item Morten Hjorth-Jensen, Nuclear Talent Course on Machine Learning in Nuclear Physics for the Erasmus+ program \href{{http://www.emm-nucphys.eu/}}{\nolinkurl{http://www.emm-nucphys.eu/}}, European Master in Nuclear Physics, University of Basse-Normandie and GANIL, January 18-29, 2021. 30 lectures and 30 exercise sessions. Main teacher

\item Morten Hjorth-Jensen, Daniel Bazin, Sean Liddick, , Michelle Kuchera, and R. Ramanujan, Online Nuclear Talent course on  Machine Learning Applied to Nuclear Physics, European Center for Theoretical Nuclear Physics and Related Areas, Trento, Italy, June 22 to July 3, 2020. Main organizer and teacher.

\item Online lectures on \textbf{Machine Learning weeks at MSU-FRIB/NSCL}, May 2020. I lectured to undergraduate, graduate and post-docs at FRIB/MSU from May 18 till May 29 on Machine Learning applied to Nuclear Physics. Two lectures per day and one hour of hands-on sessions. On average between 25-30 particpants per day. All material is available at \href{{https://github.com/mhjensen/MachineLearningMSU-FRIB2020}}{\nolinkurl{https://github.com/mhjensen/MachineLearningMSU-FRIB2020}}. In total I gave 20 one-hour lectures.

\item Morten Hjorth-Jensen, Nuclear Talent Course on Machine Learning in Nuclear Physics for the Erasmus+ program \href{{http://www.emm-nucphys.eu/}}{\nolinkurl{http://www.emm-nucphys.eu/}}, European Master in Nuclear Physics, University of Basse-Normandie and GANIL, January 20-31, 2020. 45 lectures and 45 exercise sessions. Main teacher

\item Morten Hjorth-Jensen, Matthew Hirn, Michelle Kuchera, and R. Ramanujan, \href{{https://indico.frib.msu.edu/event/16/}}{\nolinkurl{https://indico.frib.msu.edu/event/16/}}, FRIB TA Summer School - Machine Learning Applied to Nuclear Physics, Facility for Rare Isotope Beams (FRIB) on the Michigan State University campus in East Lansing, MI from May 20 to 23, 2019. Main organizer and teacher.

\item \textbf{Hackathon on Computing in Science Education}, June 3-7, 2019, Michigan State University, East Lansing, USA. . Intensive workshop on Computing in Physics Education at Michigan State University. Organized together with Danny Caballero, MSU.

\item Morten Hjorth-Jensen, Nuclear Talent Course on Machine Learning in Nuclear Physics for the Erasmus+ program \href{{http://www.emm-nucphys.eu/}}{\nolinkurl{http://www.emm-nucphys.eu/}}, European Master in Nuclear Physics, University of Basse-Normandie and GANIL, January 21-February 1, 2019. 45 lectures and 45 exercise sessions. Main teacher

\item Nuclear Talent course on Many-body methods for nuclear physics, from Structure to Reactions at Henan Normal University, P.R. China, July 16-August 5 2018. Teachers: Kevin Fossez, Morten Hjorth-Jensen, Thomas Papenbrock, and Ragnar Stroberg. 

\item Alex Brown, Alexandra Gade, Morten Hjorth-Jensen, Gustav Jansen, Robert Grzywacz, Nuclear Talent course on Nucleartheory for Nuclear Structure Experiments, July 3-21 2017. \href{{https://github.com/NuclearTalent/NuclearStructure}}{Main organizer and teacher with in total fifteen hours of lectures}. 

\item Hjorth-Jensen, Morten, \href{{https://icer-acres.msu.edu/summer-2017/schedule/}}{High performance computing in Nuclear Physics}, Lecture at the \emph{Advanced Computational Research Experience} at Michigan State University, East Lansing, Michigan, June 1, 2017.

\item Hjorth-Jensen, Morten, \href{{https://icer-acres.msu.edu/summer-2017/schedule/}}{How to write good code}, Lecture at the \emph{Advanced Computational Research Experience} at Michigan State University, East Lansing, Michigan, May 24, 2017.

\item Hjorth-Jensen, Morten, \href{{http://rafael.ujf.cas.cz/school}}{Computational Nuclear Physics and Post Hartree-Fock Methods. Configuration Interaction Theory, Many-Body Perturbation Theory and Coupled Cluster Theory}, five lectures at 28th Indian-Summer School on Ab Initio Methods in Nuclear Physics, Prague, Czech Republic, August 29 - September 2, 2016.

\item Hjorth-Jensen, Morten, \href{{http://compphysics.github.io/CompPhysUTunis/doc/web/course.html}}{Computational Physics and Quantum Mechanical Systems}, one week course on Computational Physics at the University of Tunis El Manar, Tunis, Tunisia, May 16-20, 2016. In total 15 hours of lectures and 15 hours of computer lab and exercises. 

\item Co-organizer with Giuseppina Orlandini and Alejandro Kievsky of Nuclear Talent course \href{{https://groups.nscl.msu.edu/jina/talent/wiki/Course_3}}{Few-body methods and nuclear reactions}, ECT*, Trento, Italy, July 20-August 7 2015

\item Carlo Barbieri, Wim Dickhoff, Gaute Hagen, Morten Hjorth-Jensen, and Artur Polls, Nuclear Talent course on Many-body methods for nuclear physics, GANIL, Caen, France, July 5-25 2015. \href{{http://nucleartalent.github.io/Course2ManyBodyMethods/doc/web/course.html}}{Main organizer and teacher with in total five hours of lectures}. 

\item Hjorth-Jensen, Morten, ECT* \href{{http://www.ectstar.eu/node/1287}}{Doctoral Training Program 2015 on Computational Nuclear Physics}, April 13- May 22, ECT*, Trento, Italy. I taught the last week of the lecture series. In total I have ten one hour lectures. 

\item Hjorth-Jensen, Morten, Nuclear Talent School in Nuclear Astrophysics, co-organizer with Richard Cyburt and Hendrik Schatz of the Nuclear Talent course on Nuclear Astrophysics,  Michigan State University, May 26 - June 13, 2014. 

\item Hjorth-Jensen, Morten, Nuclear Talent course on Density Functional theories, co-organizer with Scott Bogner, Nicolas Schunck, Dario Vretenar and Peter Ring, European Center for Theoretical Nuclear Physics and Related Areas, Trento, Italy, July 13 -August 1 2014.

\item Hjorth-Jensen, Morten, Nuclear Talent Course  Introduction on High-performance computing and computational tools for nuclear physics; ECT*, Trento, Italy, June 24 - July 13 2012. Main organizer and teacher together with Francesco Pederiva, Kevin Schmidt and Calvin Johnson. 

\item Hjorth-Jensen, Morten. Computational environment for Nuclear Structure, five lectures in Nuclear Physics at Universidad Complutense Madrid; 2011-01-17 - 2011-02-09

\item Hjorth-Jensen, Morten, organizer with David Dean, Thomas Papenprock and Gaute Hagen. Third MSU-UT/ORNL-UiO winter school in nuclear physics; Oak Ridge National Lab, Tennessee, January 2012

\item Hjorth-Jensen, Morten, organizer with Alex Brown and teaching five lectures. Second MSU-UT/ORNL-UiO winter school in nuclear physics, East Lansing, Michigan, USA; 2011-01-03 - 2011-01-07

\item Hjorth-Jensen, Morten, organizer, First MSU-UT/ORNL-UiO winter school in nuclear physics, Wadahl, Norway, January 4-10 2010

\item Hjorth-Jensen, Morten.  Five lectures on Theory of shell-model studies for nuclei. CERN/Isolde course on nuclear structure theory; 2010-03-01 - 2010-03-04

\item Hjorth-Jensen, Morten.  Six lectures on Nuclear interactions and the Shell Model. 8th CNS-EFES International Summer School, Riken, Tokyo, Japan, 2009-08-26 - 2009-09-01

\item Hjorth-Jensen, Morten.  Five lectures on nuclear theory at the  20th Chris Engelbrecht Summer School in Theoretical Physics, Stellenbosch, South Africa,  2009-01-19 - 2009-01-28

\item Hjorth-Jensen, Morten.  Nuclear many-body theory, five lectures at the  UK Postgraduate Nuclear Physics Summer School, Leicester, UK,  2009-09-12 - 2009-09-23

\item Hjorth-Jensen, Morten.  Nuclear many-body methods. Lectures series at Lund University; 2008-05-04 - 2008-05-07

\item Hjorth-Jensen, Morten.  Trends in Nuclear Structure Theory. Workshop at the University of Lund; 2008-05-07 - 2008-05-07

\item Hjorth-Jensen, Morten.  Trends in Nuclear Structure Theory. Physics Division Seminar; 2008-04-17 - 2008-04-17

\item Hjorth-Jensen, Morten.  Trends in nuclear structure theory. Lecture series at the University of Padova and Legnaro National Laboratory, Padova Italy; 2008-07-16 - 2008-07-19

\item Hjorth-Jensen, Morten.  Five lectures on  Monte Carlo methods and applications in the physical sciences. eScience Winther School 2007; Geilo, Norway 2007-01-28 - 2007-02-02

\item Hjorth-Jensen, Morten.  Five lectures at the ISOLDE Spring School in Nuclear Theory; CERN, Switzerland, 2007-05-21 - 2007-05-26

\item Hjorth-Jensen, Morten.  Ten lecures at  ECT* Doctoral Training Programme 2007; Trento, Italy, April 16-20

\item Hjorth-Jensen, Morten.  From the nucleon-nucleon interaction to a renormalized interaction for nuclear systems. Lecture series at Michigan State University; April 2005

\item Hjorth-Jensen, Morten. CENS: A computational Environment for Nuclear Structure. Isolde Lecture series; 2004-11-11 - 2005-11-25
\end{enumerate}

\noindent
\subsection*{Research, Publications, books, refereed scientific articles, talks and research grants}

\paragraph{Books:}
\begin{enumerate}
\item Morten Hjorth-Jensen, \emph{Computational Physics, an introduction}, to be published by IOP in 2021.

\item Morten Hjorth-Jensen, \emph{Computational Physics, an advanced course}, to be published by IOP in 2021.

\item \href{{http://www.springer.com/us/book/9783319533353}}{Morten Hjorth-Jensen, M.P. Lombardo and U. van Kolck}, \emph{Computational Nuclear Physics-Bridging the scales, from quarks to neutron stars}, Lectures Notes in Physics by Springer, Volume \textbf{936} (2017).
\end{enumerate}

\noindent
\paragraph{Publications in journals with a referee system:}
\begin{enumerate}
\item Amber Boehnlein, Markus Diefenthaler, Cristiano Fanelli, Morten Hjorth-Jensen, Tanja Horn, Michelle P. Kuchera, Dean Lee, Witold Nazarewicz, Kostas Orginos, Peter Ostroumov, Long-Gang Pang, Alan Poon, Nobuo Sato, Malachi Schram, Alexander Scheinker, Michael S. Smith, Xin-Nian Wang, Veronique Ziegler, \href{{https://arxiv.org/abs/2112.02309}}{Artificial Intelligence and Machine Learning in Nuclear Physics}, to be submitted to Reviews of Modern Physics

\item D. Rhodes, B. A. Brown, J. Henderson, A. Gade, J. Ash, P. C. Bender, R. Elder, B. Elman, M. Grinder, M. Hjorth-Jensen, H. Iwasaki, B. Longfellow, T. Mijatovic, M. Spieker, D. Weisshaar, and C. Y. Wu, \textbf{Exploring the role of high-j configurations in collective observables through the Coulomb excitation of 106Cd}, \href{{https://journals.aps.org/prc/abstract/10.1103/PhysRevC.103.L051301}}{Physical Review  C \textbf{103}, L051301 (2021)}

\item Dean Lee, Scott Bogner, B. Alex Brown, Serdar Elhatisari, Evgeny Epelbaum,  Heiko Hergert, Morten Hjorth-Jensen, Hermann Krebs, Ning Li, Bing-Nan Lu,  Ulf-G. Meissner, Robert B. Wiringa, \textbf{Hidden spin-isospin exchange symmetry}, \href{{https://journals.aps.org/prl/abstract/10.1103/PhysRevLett.127.062501}}{Physical Review Letters \textbf{127}, 062501 (2021)}

\item Aynom T. Teweldebrhan, Thomas Schuler, John Burkhart, and Morten Hjorth-Jensen, \emph{Coupled machine learning and the limits of acceptability approach applied in parameter identification for a distributed hydrological model}, \href{{https://hess.copernicus.org/articles/24/4641/2020/}}{Hydrology and Earth System Sciences 24, (2020), 4641}

\item Robert Solli, Daniel Bazin, Michelle P. Kuchera, Ryan R. Strauss, Morten Hjorth-Jensen, \emph{Unsupervised Learning for Identifying Events in Active Target Experiments}, \href{{https://www.sciencedirect.com/science/article/abs/pii/S0168900221004460}}{Nuclear Instruments and Methods in Physics Research Section A \textbf{1010}, 165461, (2020)}

\item John M. Aiken, Riccardo De Bin, Morten Hjorth-Jensen, Marcos D. Caballero, Predicting time to graduation at a large enrollment American university, \href{{https://doi.org/10.1371/journal.pone.0242334}}{PLoS ONE 15, e0242334 (2020)}

\item Calvin W. Johnson, Kristina D. Launey, Naftali Auerbach, Sonia Bacca, Bruce R. Barrett, Carl Brune, Mark A. Caprio, Pierre Descouvemont, W. H. Dickhoff, Charlotte Elster, Patrick J. Fasano, Kevin Fossez, Heiko Hergert, Morten Hjorth-Jensen, Linda Hlophe, Baishan Hu, Rodolfo M. Id Betan, Andrea Idini, Sebastian König, Konstantinos Kravvaris, Dean Lee, Jin Lei, Pieter Maris, Alexis Mercenne, Kosho Minomo, Rodrigo Navarro Perez, Witold Nazarewicz, F. M. Nunes, Marek Ploszajczak, Sofia Quaglioni, Jimmy Rotureau, Gautam Rupak, Andrey M. Shirokov, Ian Thompson, James P. Vary, Alexander Volya, Furong Xu, Remco G.T. Zegers, Vladimir Zelevinsky, Xilin Zhang, \emph{From Bound States to the Continuum}, \href{{https://urldefense.com/v3/__https://iopscience.iop.org/article/10.1088/1361-6471/abb129__;!!HXCxUKc!nOuho6Yd_SYxj9PUxWEODhuHbyyRAD_27R1bipWQvrj6Jjnwzdi7Z4FRJSF1-w$}}{Journal of Physics G Phys. 47, 123001  (2020)}

\item D. A. Torres, R. Chapman, V. Kumar, B. Hadinia, A. Hodsdon, M. Labiche, X. Liang, D. O’Donnell, J. Ollier, R. Orlandi, J. F. Smith, K. -M. Spohr, P. Wady, Z. M. Wang, L. Corradi, E. Fioretto, A. Gadea, G. de Angelis, N. Mărginean, D. R. Napoli, E. Sahin, A. M. Stefanini, J. J. Valiente-Dobón, F. D. Vedova, M. Axiotis, T. Martinez, S. Szilner, D. Bazzacco, S. Beghini, E. Farnea, R. Mărginean, D. Mengoni, G. Montagnoli, F. Recchia, F. Scarlassara, C. A. Ur, S. M. Lenzi, S. Lunardi, T. Kröll, F. Haas, T. Faul, M. Hjorth-Jensen, B. G. Carlsson, S. J. Freeman, A. G. Smith, G. Jones, N. Thompson, G. Pollarolo, G. S. Simpson, \emph{Study of medium-spin states of neutron-rich 87, 89, 91Rb isotopes}, \href{{https://epja.epj.org/articles/epja/abs/2019/09/10050_2019_Article_12839/10050_2019_Article_12839.html}}{European Physical Journal A 55 (2019) p.158}

\item Marcos Daniel Caballero, Morten Hjorth-Jensen, Integrating a Computational Perspective in Physics Courses, arXiv:1802.08871, \href{{https://novapublishers.com/shop/new-trends-in-physics-education-research/}}{Nova Publishers, New Trends in Physics Education Research (2018)}

\item \href{{https://journals.aps.org/prc/abstract/10.1103/PhysRevC.96.024323}}{Erich W. Ormand, Alex B. Brown and Morten Hjorth-Jensen}, \emph{First-principles calculations for c-coefficients of the isobaric mass multiplet equation in the 1p0f shell}, \emph{Physical Review C} Rapids, 96:024323 (2017). 

\item \href{{http://www.springer.com/us/book/9783319533353}}{Morten Hjorth-Jensen, M.P. Lombardo and U. van Kolck}, \emph{Motivation and Overarching Aims}, \emph{Lecture Notes in Physics}, Editors M. Hjorth-Jensen, M.P. Lombardo and U. van Kolck, Volume \textbf{936} pages 1-4 (2017).

\item \href{{http://www.springer.com/us/book/9783319533353}}{Justin Lietz, Sam Novario, Gustav, Jansen, Gaute Hagen, and Morten Hjorth-Jensen}, \emph{High-performance computing and infinite nuclear matter}, \emph{Lecture Notes in Physics}, Editors M. Hjorth-Jensen, M.P. Lombardo and U. van Kolck, Volume \textbf{936} pages 293-399 (2017).

\item \href{{http://aip.scitation.org/doi/abs/10.1063/1.4995615}}{Fei Yuan, Sam Novario, Nathan Parzuchowski, Sarah Reimann, Scott K. Bogner and Morten Hjorth-Jensen}.,   \emph{First principle calculations of quantum dot systems}, Journal of Chemical Physics, 147:164109 (2017).

\item \href{{https://physics.aps.org/articles/v10/72}}{Morten Hjorth-Jensen}, \emph{Scattering Experiments Tease Out the Strong Force}, \emph{Physics}, 10:72 (2017).

\item \href{{https://journals.aps.org/prc/abstract/10.1103/PhysRevC.95.021304}}{Naofumi Tsunoda, Takaharu Otsuka, Noritaka Shimizu, Morten Hjorth-Jensen, Kazuo Takayanagi, Toshio Suzuki}, \emph{Exotic neutron-rich medium-mass nuclei with realistic nuclear forces}, \emph{Physical Review C} Rapids, 95:021304(R) (2017).

\item G. Hagen, M. Hjorth-Jensen, G. R. Jansen, T. Papenbrock, \emph{Emergent properties of nuclei from ab initio coupled-cluster calculations}, \emph{Physica Scripta}, 91:063006 (2016).

\item G. Hagen, A. Ekstrom, C. Forssen , G. R. Jansen, W. Nazarewicz, T. Papenbrock, K. A. Wendt, S. Bacca, N. Barnea, B. Carlsson, C. Drischler, K. Hebeler, M. Hjorth-Jensen, M. Miorelli, G. Orlandini, A. Schwenk, and J. Simonis,  \emph{Charge, neutron, and weak size of the atomic nucleus},  \emph{Nature Physics}, 12:186–190 (2016).

\item A. Ekstrom, G. R. Jansen, K. A. Wendt, G. Hagen, T. Papenbrock, B. D. Carlsson, C. Forssen, M. Hjorth-Jensen, P. Navratil, W. Nazarewicz,   \emph{Accurate nuclear radii and binding energies from a chiral interaction}, \emph{Physical Review C}, 91, 051301(R) (2015).

\item A. Ekstrom, B. D. Carlsson, K. A. Wendt, C. Forssén, M. Hjorth-Jensen, R. Machleidt, S. M. Wild,  \emph{Statistical uncertainties of a chiral interaction at next-to-next-to leading order},   \emph{Journal of Physics G}, 42:034003 (2015).

\item A. B. Balantekin, J. Carlson, D. J. Dean, G. M. Fuller, R. J. Furnstahl, M. Hjorth-Jensen, R. V. F. Janssens, Bao-An Li, W. Nazarewicz, F. M. Nunes, W. E. Ormand, S. Reddy, B. M. Sherrill ,  \emph{Nuclear Theory and Science of the Facility for Rare Isotope Beams},   \emph{Modern Physics Letters A}, 29:1430010 (2014).

\item Zs. Vajta, M. Stanoiu, D. Sohler, G. R. Jansen, F. Azaiez, Zs. Dombrádi, O. Sorlin, B. A. Brown, M. Belleguic, C. Borcea, C. Bourgeois, Z. Dlouhy, Z. Elekes, Zs. Fülöp, S. Grévy, D. Guillemaud-Mueller, G. Hagen, M. Hjorth-Jensen, F. Ibrahim, A. Kerek, A. Krasznahorkay, M. Lewitowicz, S. M. Lukyanov, S. Mandal, P. Mayet, J. Mrázek, F. Negoita, Yu.-E. Penionzhkevich, Zs. Podolyák, P. Roussel-Chomaz, M. G. Saint-Laurent, H. Savajols, G. Sletten, J. Timár, C. Timis, and A. Yamamoto,   *Excited states in the neutron-rich nucleus 25F,   \emph{Physical Review C}, 89:054323 (2014).

\item A. Sanetullaev, M.B. Tsang, W.G. Lynch, Jenny Lee, D. Bazin, K.P. Chan, D. Coupland, V. Henzl, D. Henzlova, M. Kilburn, A.M. Rogers, Z.Y. Sun, M. Youngs, R.J. Charity, L.G. Sobotka, M. Famiano, S. Hudan, D. Shapira, W.A. Peters, C. Barbieri, M. Hjorth-Jensen, M. Horoi, T. Otsuka, T. Suzuki, Y. Utsuno  \emph{Neutron spectroscopic factors of 55Ni hole-states from (p,d) transfer reactions},   \emph{Physics Letters B}, 736:137 (2014).

\item G. Hagen, T. Papenbrock,   A. Ekstrom, G. Baardsen, S. Gandolfi, K. A. Wendt, M. Hjorth-Jensen, and C. Horowitz,  \emph{Coupled-cluster calculations of nucleonic matter},   \emph{Physical Review C},  89:014319 (2014).

\item T. Papenbrock, G. Hagen, M. Hjorth-Jensen, and  D. J. Dean,   \emph{Coupled-cluster computations of atomic nuclei},   \emph{Reports on Progress in Physics}, 77:096302 (2014).

\item N. Tsunoda, K. Takayanagi, M. Hjorth-Jensen and T. Otsuka,  \emph{Multi-shell effective interactions},   \emph{Physical Review C},    89:024313 (2014).

\item G. Baardsen, A. Ekstrom, G. Hagen, and M. Hjorth-Jensen,  \emph{Coupled-cluster studies of infinite  nuclear matter},   \emph{Physical Review C}, 88:054312 (2013).  

\item V. M. Bader, A. Gade, D. Weisshaar, T. Baugher, D. Bazin, J. S. Berryman, B. A. Brown, A. Ekstrom, M. Hjorth-Jensen, S. R. Stroberg, W. B. Walters, K. Wimmer, and R. Winkler,   \emph{Quadrupole collectivity in neutron-deficient Sn nuclei: 104Sn and the role of proton excitations},   \emph{Physical Review C},  88:051301(R) (2013). 

\item A. Ekstrom, G. Baardsen, C. Forss'en, G. Hagen, M. Hjorth-Jensen, G. R. Jansen, R. Machleidt, W. Nazarewicz, T. Papenbrock, J. Sarich, and S. M. Wild,   \emph{An optimal chiral interaction at next-to-next-to leading order},   \emph{Physical Review Letters},  110:192502 (2013).  

\item Lepailleur, A. and Sorlin, O. and Caceres, L. and Bastin, B. and Borcea, C. and Borcea, R. and Brown, B. A. and Gaudefroy, L. and Gr'evy, S. and Grinyer, G. F. and Hagen, G. and Hjorth-Jensen, M. and Jansen, G. R. and Llidoo, O. and Negoita, F. and de Oliveira, F. and Porquet, M.-G. and Rotaru, F. and Saint-Laurent, M.-G. and Sohler, D. and Stanoiu, M. and Thomas, J. C.,  \emph{Spectroscopy of 26F to Probe Proton-Neutron Forces Close to the Drip Line},   \emph{Physical Review Letters},  110:082502 (2013). 

\item D. D. DiJulio, J. Cederkall, C. Fahlander, A. Ekstrom, M. Hjorth-Jensen, M. Albers, V. Bildstein, A. Blazhev, I. Darby, T. Davinson, H. De Witte, J. Diriken, Ch.~Fransen, K. Geibel, R. Gernhäuser, A. Görgen, H. Hess, K. Heyde, J. Iwanicki, R. Lutter, P. Reiter, M. Scheck, M. Seidlitz, S. Siem, J. Taprogge, G. M. Tveten, J. Van de Walle, D. Voulot, N. Warr, F. Wenander, and K. Wimmer  \emph{Coulomb excitation of 107In},   \emph{Physical Review C},  87:017301 (2013).   

\item C. Forssen, G. Hagen, M. Hjorth-Jensen, W. Nazarewicz, and J. Rotureau,  \emph{Living on the edge of stability, the limits of the nuclear landscape},   \emph{Physica Scripta},  T152:014022 (2013). 

\item Liddick, S. N. and Abromeit, B. and Ayres, A. and Bey, A. and Bingham, C. R. and Brown, B. A. and Cartegni, L. and Crawford, H. L. and Darby, I. G. and Grzywacz, R. and Ilyushkin, S. and Hjorth-Jensen, M. and Larson, N. and Madurga, M. and Miller, D. and Padgett, S. and Paulauskas, S. V. and Rajabali, M. M. and Rykaczewski, K. and Suchyta, S.,  * Low-energy level schemes of 66,68Fe and inferred proton and neutron excitations across $Z=28$ and $N=40$*,   \emph{Physical Review C},  87:014325, 2013.  

\item D. D. DiJulio, J. Cederkall, C. Fahlander, A. Ekstrom, M. Hjorth-Jensen, M. Albers, V. Bildstein, A. Blazhev, I. Darby, T. Davinson, H. De Witte, J. Diriken, Ch.~Fransen, K. Geibel, R. Gernhauser, A. Gorgen, H. Hess, J. Iwanicki, R. Lutter, P. Reiter, M. Scheck, M. Seidlitz, S. Siem, J. Taprogge, G.M. Tveten, J. Van de Walle, D. Voulot, N. Warr, F. Wenander, and K. Wimmer,  \emph{Excitation strengths in 109Sn: Single-neutron and collective excitations near 100Sn},   \emph{Physical Review C},  86:031302(R), 2012. 

\item D. D. DiJulio, J. Cederkall, C. Fahlander, A. Ekstrom, M. Hjorth-Jensen, M. Albers, V. Bildstein, A. Blazhev, I. Darby, T. Davinson, H. De Witte, J. Diriken, Ch.~Fransen, K. Geibel, R. Gernhauser, A. Gorgen, H. Hess, J. Iwanicki, R. Lutter, P. Reiter, M. Scheck, M. Seidlitz, S. Siem, J. Taprogge, G.M. Tveten, J. Van de Walle, D. Voulot, N. Warr, F. Wenander, and K. Wimmer,  \emph{Coulomb excitation of 107Sn},   \emph{European Journal of Physics A}, 48:105,  2012.  

\item Gaute Hagen, Morten Hjorth-Jensen, Gustav Ragnar Jansen, Ruprecht Machleidt, and Thomas Papenbrock, \emph{Evolution of shell structure in neutron-rich calcium isotopes},   \emph{Physical Review Letters}, 109:032502, 2012. 

\item Gaute Hagen, Morten Hjorth-Jensen, Gustav Ragnar Jansen, Ruprecht Machleidt, and Thomas Papenbrock, \emph{Continuum effects and three-nucleon forces in neutron-rich  oxygen isotopes},   \emph{Physical Review Letters}, 108:242501, 2012.  

\item Torres, D. A. and Kumbartzki, G. J. and Sharon, Y. Y. and Zamick, 	L. and Manning, B. and Benczer-Koller, N. and Speidel, K.-H. and 	Ahn, T. and Anagnostatou, V. and Elvers, M. and Goddard, P. and Heinz, 	A. and Ilie, G. and Radeck, D. and Savran, D. and Werner, V. and 	Gurdal, G. and Taylor, M. J. and Maier-Komor, P. and Hjorth-Jensen, 	M. and Robinson, S. J. Q.  \emph{Measurement of the 96Ru g-factor and its nuclear structure interpretation}.  \emph{Physical Review C}, 85:017305, 2012.  

\item Torres, D. A. and Kumbartzki, G. J. and Sharon, Y. Y. and Zamick, 	L. and Manning, B. and Benczer-Koller, N. and Gurdal, G. and Speidel, 	K.-H. and Hjorth-Jensen, M. and Maier-Komor, P. and Robinson, S. J. Q. and Ahn, T. and Anagnostatou, V. and Elvers, M. and Goddard, 	P. and Heinz, A. and Ilie, G. and Radeck, D. and Savran, D. and Werner, V.  \emph{First g-factor measurements of the 2+ and the 4+ states of radioactive 100Pd}.  \emph{Physical Review C}, 84:044327, 2011.  

\item Naofumi Tsunoda, Takaharu Otsuka, Koshiroh Tsukiyama, and Morten Hjorth-Jensen  \emph{Renormalization persistency of the tensor force in nuclei}.  \emph{Physical Review C}, 84:044322, 2011.  

\item O. Jensen, Gaute Hagen, Morten Hjorth-Jensen, Alex Boyd Brown, and Alexandra Gade  \emph{Quenching of spectroscopic factors for proton removal in oxygen isotopes},   \emph{Physical Review Letters}, 107:032501, 2011. 

\item Magnus Pedersen Lohne, Gaute Hagen, Morten Hjorth-Jensen, Simen Kvaal, and Francesco Pederiva,  \emph{Ab initio calculations of Circular quantum dots}.  \emph{Physical Review B}, 84:032501, 2011.    

\item Elise Bergli and Morten Hjorth-Jensen,  *Summation of Parquet diagrams as an \emph{ab initio} method in nuclear structure calculations*,  \emph{Annals of Physics}, 326:1125, 2011.   

\item Gustav Ragnar Jansen, Morten Hjorth-Jensen, Gaute Hagen, and Thomas Papenbrock,  \emph{Toward open-shell nuclei with coupled-cluster theory}.  \emph{Physical Review C}, 83:054306, 2011. 

\item Morten Hjorth-Jensen,  \emph{The Carbon Challenge},  \emph{Physics}, 4:38, 2011.  

\item O. Jensen, G. Hagen, M. Hjorth-Jensen, and J. S. Vaagen,  \emph{Closed-shell properties of 24O with ab initio coupled-cluster theory},  \emph{Physical Review C}, 83:021305, 2011. 

\item Angelo Signoracci, B. Alex Brown, and Morten Hjorth-Jensen,  \emph{Renormalized interactions with a realistic single-particle   basis},  \emph{Physical Review C}, 83:024315, 2011.   

\item Boyd Alexander Brown, Angelo Signoracci, and Morten Hjorth-Jensen,  \emph{Configuration interactions constrained by energy density  functionals},  \emph{Physics Letters B}, 695:507, 2011.  

\item G. Hagen, T. Papenbrock, D. J. Dean, and M. Hjorth-Jensen,  *Ab initio coupled-cluster approach to nuclear structure with modern   nucleon-nucleon interactions,  \emph{Phys. Rev. C}, 82(3):034330, 2010.  

\item L. Atanasova, Dimiter Balabanski, S. K. Chamoli, M. Hass, G. S. Simpson,   D. Bazzacco, F. Becker, P. Bednarczyk, G. Benzoni, N. Blasi, A. Blazhev,   A. Bracco, C. Brandau, L. Caceres, F. Camera, F. C. L. Crespi, P. Detistov,   P. Doornenbal, C. Fahlander, E. Farnea, G. Georgiev, J. Gerl, K. A.   Gladnishki, M. Gorska, J. Grebosz, R. Hoischen, G. Ilie, M. Ionescu-Bujor,   A. Iordachescu, A. Jungclaus, G. Bianco, M. Kmiecik, I. Kojouharov, N. Kurz,   S. Lakshmi, R. Lozeva, A. Maj, D. Montanari, G. Neyens, M. Pfuetzner,   S. Pietri, Z. Podolyak, W. Prokopowicz, D. Rudolph, G. Rusev, T. Saito,   A. Saltarelli, H. Schaffner, R. Schwengner, S. Tashenov, J. J.   Valiente-Dobon, N. Vermeulen, J. Walker, E. Werner-Malento, O. Wieland, H. J.   Wollersheim, H. Grawe, and Morten Hjorth-Jensen.  \emph{g-factor measurements at RISING: The cases of 127Sn and   128Sn}.  \emph{Europhysics letters}, 91:42001, 2010. 

\item I. Darby, R. Grzywacz, J. C. Batchelder, C. R. Bingham, L. Cartegni, C. J.   Gross, Morten Hjorth-Jensen, D. T. Joss, S. N. Liddick, W. Nazarewicz,   S. Padgett, R. D. Page, Thomas Papenbrock, M. M. Rajabali, J. Rotureau, and   K. P. Rykaczewski, \emph{Orbital Dependent Nucleonic Pairing in the Lightest Known   Isotopes of Tin}.  \emph{Physical Review Letters}, 105:162502, 2010. 

\item A. Ekstrom, Joakim Cederkall, Claes Fahlander, Morten Hjorth-Jensen, Torgeir   Engeland, Peter Butler, P. A. Butler, T. Davinson, J. Eberth, F. Finke,   Andreas Gorgen, M. Gorska, A. M. Hurst, O. Ivanov, J. Iwanicki, U. Koster,   B. A. Marsh, J. Mierzejewski, P. Reiter, Sunniva Siem, G. Sletten,   I. Stefanescu, Gry Merete Tveten, J. Van de Walle, D. Voulot, N. Warr,   D. Weisshaar, F. Wenander, and M. Zielinska, \emph{Coulomb excitation of the odd-odd isotopes 106In and 108In},  \emph{European Physical Journal A}, 44:355, 2010. 

\item Gaute Hagen, Thomas Papenbrock, and Morten Hjorth-Jensen,  \emph{Ab Initio Computation of the 17F Proton Halo State and Resonances in A=17 Nuclei},  \emph{Physical Review Letters}, 104:182501, 2010. 

\item Morten Hjorth-Jensen, David Jarvis Dean, G. Hagen, and Simen Kvaal,  \emph{Many-body interactions and nuclear structure},  \emph{Journal of Physics G: Nuclear and Particle Physics}, 37:064035,   2010. 

\item N. Hoteling, C. Chiara, R. Broda, W. B. Walters, R. V. F. Janssens, Morten   Hjorth-Jensen, M. B. Carpenter, B. Fornal, A. A. Hecht, W. Krolas,   T. Lauritsen, T. Pawlat, D. Seweryniak, X. Wang, A. Wohr, J. Wrzesinski, and   S. Zhu.  \emph{Structure of 60,62Fe and the onset of $nu g(9/2)$ occupancy},  \emph{Physical Review C}, 82:044305, 2010. 

\item Takahuro Otsuka, Toshio Suzuki, Micho Honma, Yutaka Utsuno, Naofumi Tsunoda,   Koshiroh Tsukiyama, and Morten Hjorth-Jensen,  \emph{Novel Features of Nuclear Forces and Shell Evolution in Exotic Nuclei},  \emph{Physical Review Letters}, 104:012501, 2010. 

\item C. Barbieri and Morten Hjorth-Jensen,  \emph{Quasiparticle and quasihole states of nuclei around 56Ni},  \emph{Physical Review C}, 79:064313, 2009. 

\item A. Ekstrom, J. Cederkall, D. D. DiJulio, C. Fahlander, Morten Hjorth-Jensen,   A. Blazhev, B. Bruyneel, P. A. Butler, T. Davinson, J. Eberth, C. Fransen,   K. Geibel, H. Hess, O. Ivanov, J. Iwanicki, O. Kester, J. Kownacki,   U. Koster, B. A. Marsh, P. reiter, M. Scheck, B. Siebeck, Sunniva Siem,   I. Stefanescu, Heidi Kristine Toft, Gry Merete Tveten, J. Van de Walle,   D. Voulot, N. Warr, D. Weisshaar, F. Wenander, K. Wrzosek, and M. Zielinska,  \emph{Electric quadrupole moments of the 2+ states in  100,102,104Cd},  \emph{Physical Review C}, 80:054302, 2009. 

\item G. Hagen, T. Papenbrock, D. J. Dean, Morten Hjorth-Jensen, and B. V. Asokan,  \emph{Ab initio computation of neutron-rich oxygen isotopes},  \emph{Physical Review C}, 80:021306, 2009. 

\item Micho Honma, Takahuro Otsuka, T. Mizusaki, and Morten Hjorth-Jensen,  \emph{New effective interaction for fpg-shell nuclei}.  \emph{Physical Review C}, 80:064323, 2009. 

\item Koshiroh Tsukiyama, Morten Hjorth-Jensen, and Gaute Hagen,  \emph{Gamow shell-model calculations of drip-line oxygen isotopes}.  \emph{Physical Review C}, 80:051301(R), 2009. 

\item David J. Dean, Gaute Hagen, Morten Hjorth-Jensen, and Thomas Papenbrock,  * Computational aspects of nuclear coupled-cluster theory*.  \emph{Computational Science and Discovery}, 1:015008, 2008. 

\item David J. Dean, Gaute Hagen, Morten Hjorth-Jensen, Thomas Papenbrock, and Achim Schwenk,  \emph{Comment on Ab initio study of 40Ca with an   importance-truncated no-core shell model}.  \emph{Physical Review Letters}, 101:119201, 2008. 

\item A. Ekstrom, J. Cederkall, C. Fahlander, Morten Hjorth-Jensen, F. Ames, P. A.   Butler, T. Davinson, J. Eberth, F. Fincke, A. Gorgen, M. Gorska, D. Habs,   A. M. Hurst, M. Huyse, O. Ivanov, J. Iwanicki, O. Kester, U. Koster, B. A.   Marsh, J. Mierzejewski, P. Reiter, H. Scheit, D. Schwalm, Sunniva Siem,   G. Sletten, I. Stefanescu, Gry Merete Tveten, J. V. de Walle, P. Van Duppen,   D. Voulot, N. Warr, D. Weisshaar, F. Wenander, and M. Zielinska.  \emph{Transition strengths in 106Sn and 108Sn},  \emph{Physical Review Letters}, 101:01250, 2008. 

\item Gaute Hagen, Thomas Papenbrock, David J. Dean, and Morten Hjorth-Jensen,  \emph{Medium-Mass Nuclei from Chiral Nucleon-Nucleon Interactions},  \emph{Physical Review Letters}, 101:092502, 2008. 

\item N. Hoteling, W. B. Walters, R. V. F. Janssens, R. Broda, M. P. Carpenter,   B. Fornal, A. A. Hecht, Morten Hjorth-Jensen, W. Krolas, T. Lauritsen,   T. Pawlat, D. Seweryniak, J. R. Stone, X. Wang, A. Wohr, J. Wrzesinski, and   S. Zhu,  \emph{Rotation-aligned coupling in 61Fe},  \emph{Physical Review C}, 77:044314, 2008. 

\item J. Cederkall, A. Ekstrom, C. Fahlander, A. M. Hurst, Morten Hjorth-Jensen,   F. Ames, A. Banu, P. A. Butler, T. Davinson, U. D. Pramanik, J. Eberth,   S. Franchoo, G. Georgiev, M. Gorska, D. Habs, M. Huyse, O. Ivanov,   J. Iwanicki, O. Kester, U. Koster, B. A. Marsh, O. Niedermaier, T. Nilsson,   P. Reiter, H. Scheit, D. Schwalm, T. Sieber, G. Sletten, I. Stefanescu, J. V.   de Walle, P. Van Duppen, N. Warr, D. Weisshaar, and F. Wenander, \emph{Sub-barrier Coulomb excitation of 110Sn and its   implications for the 100Sn shell closure},  \emph{Physical Review Letters}, 98:172501, 2007. 

\item Gaute Hagen, David J. Dean, Morten Hjorth-Jensen, and Thomas Papenbrock,  \emph{Complex coupled-cluster approach to an ab-initio description of   open quantum systems},  \emph{Physics Letters B}, 656:169, 2007. 

\item Gaute Hagen, David J. Dean, Morten Hjorth-Jensen, Thomas Papenbrock, and Achim   Schwenk,  \emph{Benchmark calculations for 3H, 4He, 16O, and 40Ca with ab initio coupled-cluster theory}.  \emph{Physical Review C}, 76:044305, 2007. 

\item Maxim Kartamychev, Torgeir Engeland, Morten Hjorth-Jensen, and Eivind Osnes,  \emph{Effective interactions and shell model studies of heavy tin isotopes},  \emph{Physical Review C}, 76:024313, 2007. 

\item Simen Kvaal, Morten Hjorth-Jensen, and Halvor Moll Nilsen,  \emph{Effective interactions, large-scale diagonalization, and   one-dimensional quantum dots},  \emph{Physical Review B}, 76:085421, 2007. 

\item C. Vaman, C. Andreoiu, D. Bazin, A. Becerril, B. A. Brown, C. M. Campbell,   A. Chester, J. M. Cook, D. C. Dinca, A. Gade, D. Galaviz, T. Glasmacher,   Morten Hjorth-Jensen, M. Horoi, D. Miller, V. Moeller, W. F. Mueller,   A. Schiller, K. Starosta, A. Stolz, J. R. Terry, A. Volya, V. Zelevinsky, and   H. Zwahlen.  \emph{Z=50 shell gap near 100Sn from intermediate-energy coulomb excitations in even-mass 106-112Sn isotopes},  \emph{Physical Review Letters}, 99:162501, 2007. 

\item Jeffrey Grour, Piotr Piecuch, Morten Hjorth-Jensen, Marta Wloch, and   David Jarvis Dean, \emph{Coupled-cluster calculations for valence systems around 16O}, \emph{Physical Review C}, 74:024310, 2006. 

\item Gaute Hagen, Morten Hjorth-Jensen, and Michel Nicolas,  \emph{Gamow shell model and realistic nucleon-nucleon interactions},  \emph{Physical Review C}, 73:064307, 2006. 

\item Nathan Hoteling, W. B. Walters, R. V. F. Janssens, R. Broda, M. F. Carpenter,   B. Fornal, A. A. Hecht, Morten Hjorth-Jensen, W. Krolas, T. Lauritzen,   T. Pawlat, D. Seweryniak, X. Wang, A. Wohr, J. Wrzesinski, and S. Zhu.  \emph{Yrast structure of 64Fe}.  \emph{Physical Review C}, 74:064313, 2006. 

\item J. Leske, K. H. Speidel, S. Schielke, J. Gerber, P. Maier-Komor, Torgeir   Engeland, and Morten Hjorth-Jensen,  \emph{Experimental g-factor and B(E2) value of the 4+ state in Coulomb-excited 66Zn compared to shell-model predictions}.  \emph{Physical Review C}, 73:064305, 2006.   

\item A. Banu, J. Gerl, C. Fahlander, M. Gorska, H. Grawe, H. J. Wollersheim,   E. Caurier, Torgeir Engeland, A. Gniady, Morten Hjorth-Jensen, F. Nowacki,   T. Beck, F. Becker, P. Bednarczyk, M. A. Bentley, A. Burger, F. Cristancho,   G. de Angelis, Z. Dombradi, P. Doornenbal, H. Geissel, J. Grebosz,   G. Hammond, M. Hellstrom, J. Jolie, I. Kojouharov, N. Kurz, R. Lozeva,   S. Mandal, N. Marginean, S. Muralithar, J. Nyberg, J. Pochodzalla,   W. Prokopowicz, P. Reiter, D. Rudolph, C. Rusu, N. Saito, H. Schaffner,   D. Sohler, H. Weick, C. Wheldon, and M. Winkler,  \emph{108Sn studied with intermediate-energy Coulomb excitation},  \emph{Physical Review C}, 72:061305, 2005. 

\item Boyd Alexander Brown, Nick Stone, Irena Stone, Ian Towner, and Morten Hjorth-Jensen,  \emph{Magnetic moments of the 2+ states around 132Sn},  \emph{Physical Review C}, 71:044317, 2005. 

\item Paul Ellis, Torgeir Engeland, Morten Hjorth-Jensen, Maximx Kartamyshev, and   Eivind Osnes, \emph{Model calculation of effective three-body forces},   \emph{Physical Review C}, 71:034301, 2005. 

\item Gaute Hagen, Morten Hjorth-Jensen, and Jan S. Vaagen, \emph{Effective interaction techniques for the Gamow shell model},  \emph{Physical Review C}, 71:044314, 2005. 

\item J. K. Leske, Karl-heinz Speidel, S. Schielke, J. Gerber, P. Maier-komor, Morten   Hjorth-Jensen, and Torgeir Engeland,  \emph{Physical Review C}, 72:044301, 2005. 

\item Jon Kristian Nilsen, Jordi Mur-Petit, Muntsa Guilleumas, Morten Hjorth-Jensen,   and Artur Polls,  \emph{Vortices in atomic Bose-Einstein condensates in the  large-gas-parameter region},  \emph{Physical Review A}, 71:053610, 2005. 

\item D. Sohler, M. Palacz, Z. Dombradi, Morten Hjorth-Jensen, C. Fahlander, L. O.   Norlin, J. Nyberg, T. Back, K. Lagergren, D. Rudolph, A. Algora, C. Andreoiu,   G. de Angelis, A. Atac, D. Bazzacco, J. Cederkall, B. Cederwall, B. Fant,   E. Farnea, A. Gadea, M. Gorska, H. Grawe, N. Hashimito-Saitoh, A. Johnson,   A. Kerek, W. Klamra, J. Kownacki, S. M. Lenzi, A. Likar, M. Lipoglavsek,   M. Moszynski, D. R. Napoli, C. Rossi-Alvarez, H. A. Roth, T. Saitoh,   D. Seweryniak, O. Skeppstedt, J. Timar, M. Weisflog, and M. Wolinska, \emph{Maximally aligned states in the proton drip line nucleus 106Sb},  \emph{Nuclear Physics A}, 753:251, 2005. 

\item Marta Wloch, David J. Dean, Jeffrey Grour, Morten Hjorth-Jensen, Karol   Kowalski, Thomas Papenbrock, and Piotr Piecuch, \emph{Ab-initio coupled-cluster study of 16O},  \emph{Physical Review Letters}, 94:212501, 2005. 

\item David J. Dean, Torgeir Engeland, Morten Hjorth-Jensen, Maxim Kartamychev, and   Eivind Osnes, \emph{Effective interactions and the nuclear shell-model},  \emph{Progress in Particle and Nuclear Physics}, 53:419, 2004. 

\item Haavar Gausemel, Birger Fogelberg, Torgeir Engeland, Morten Hjorth-Jensen,   Per Hoff, Hendryk Mach, K. A. Mezilev, and Jon Petter Omtvedt, \emph{Decay of 127In and 129In},  \emph{Physical Review C}, 69:054307, 2004. 

\item Gaute Hagen, Jan S. Vaagen, and Morten Hjorth-Jensen,  \emph{The contour deformation method in momentum space, applied to   subatomic physics},  \emph{Journal of Physics A: Mathematical and General}, 37:8991, 2004. 

\item Karol Kowalski, David J. Dean, Morten Hjorth-Jensen, Thomas Papenbrock, and   Piotr Piecuch, \emph{Coupled cluster calculations of ground and excited states of nuclei},  \emph{Physical Review Letters}, 92:132501, 2004. 

\item David J. Dean and Morten Hjorth-Jensen,  \emph{Pairing in nuclear systems: from neutron stars to finite nuclei}, \emph{Reviews of Modern Physics}, 75:607, 2003. 

\item I. Dillmann, K. L. Kratz, A. Wohr, O. Arndt, B. A. Brown, Per Hoff, Morten   Hjorth-Jensen, U. Koster, A. Ostrowski, B. Pfeiffer, D. Seweryniak,   J. Shergur, and W. B. Walters,  \emph{N=82 shell-quenching of the classical r-process waiting-point 130Cd}.  \emph{Physical Review Letters}, 91:162503, 2003. 

\item Magne Guttormsen, Rositsa Chankova, Morten Hjorth-Jensen, John Bernhard   Rekstad, Sunniva Siem, Andreas Schiller, and David J. Dean, \emph{Free energy and criticality in the nucleon pair breaking   process},  \emph{Physical Review C}, 68:034311, 2003. 

\item A. Schiller, Emel Algin, Lee Bernstein, P. E. Garrett, Magne Guttormsen, Morten   Hjorth-Jensen, C. W. Johnson, Gary Mitchell, John Bernhard Rekstad, Sunniva   Siem, Alexander Voinov, and William Younes, \emph{Level densities in 56,57Fe and 96,97Mo},  \emph{Physical Review C}, 68:054326, 2003. 

\item N. Fotiades, J. A. Cizewski, J. A. Becker, A. Bernstein, D. P. Mcnabb, William   Younes, R. M. Clark, P. Fallon, I. Y. Lee, A. O. Macchiavelli, Anne Holt, and   Morten Hjorth-Jensen, \emph{High-spin excitations in 92,93,94,95Zr}, \emph{Physical Review C}, 65:044303, 2002. 

\item M. Lipoglavsek, C. Baktash, Jan Blomqvist, David J. Dean, Torgeir Engeland,   C. Fahlander, Morten Hjorth-Jensen, Robert V. F. Janssens, A. Likar, Eivind   Osnes, and S. D. Paul, \emph{Break-up of the Doubly-magic 100Sn core},  \emph{Physical Review C}, 66:011302, 2002. 

\item M. Lipoglavsek, C. Baktash, M. P. Carpenter, David J. Dean, Torgeir Engeland,   C. Fahlander, Morten Hjorth-Jensen, and Eivind Osnes, *Excited states of the proton emitter 105Sb, \emph{Physical Review C}, 65:051037, 2002. 

\item M. Lipoglavsek, C. Baktash, M. P. Carpenter, David J. Dean, Torgeir Engeland,   Morten Hjorth-Jensen, and Eivind Osnes, \emph{Core excitations in 102In},  \emph{Physical Review C}, 65:021302(R), 2002. 

\item J. J. Ressler, W. B. Walters, C. N. Davids, David J. Dean, Andreas Heinz,   Morten Hjorth-Jensen, D. Seweryniak, and J. Shergur, \emph{First observation of 109Te $\beta^+$ and electron capture  decay of 109Sb},  \emph{Physical Review C}, 66:024308, 2002. 

\item Andreas Schiller, Magne Guttormsen, Morten Hjorth-Jensen, John Bernhard   Rekstad, and Sunniva Siem, \emph{Model for pairing phase transition in atomic nuclei},   \emph{Physical Review C}, page 024315, 2002. 

\item J. Shergur, B. A. Brown, V. N. Fedosseev, U. K?ster, K. L. Kratz,   D. Seweryniak, W. B. Walters, A. Wohr, D. Fedorov, M. Hannawald, Morten   Hjorth-Jensen, V. Mishin, B. Pfeiffer, J. J. Ressler, H. O. U. Fynbo, and Per   Hoff, \emph{Beta decay studies of 135-137Sn using selective reonace   laser ionization techniques}, \emph{Physical Review C}, 65:034313, 2002. 

\item Magne Guttormsen, Morten Hjorth-Jensen, Elin Melby, John Bernhard Rekstad,   Andreas Schiller, and Sunniva Siem, \emph{Heat capacity and pairing transition in nuclei},  \emph{Physical Review C}, 64:034319, 2001. 

\item Andreas Schiller, Amund Bjerve, Magne Guttormsen, Morten Hjorth-Jensen, Finn   Ingebretsen, Elin Melby, John Bernhard Rekstad, Sunniva Siem, and   Stein Westad Odegaard, \emph{The critical temperature for quenching of pair correlations},  \emph{Physical Review C}, 63:021306, 2001. 

\item Teemu Siiskonen, Morten Hjorth-Jensen, and Jouni Suhonen, \emph{Renormalization of the weak hadronic current in the nuclear medium},  \emph{Physical Review C}, 63:024315, 2001. 

\item Torgeir Engeland, Morten Hjorth-Jensen, and Eivind Osnes, \emph{Shell model studies of the proton drip line nucleus 106Sb},  \emph{Physical Review C}, 61:00010(R), 2000. 

\item Magne Guttormsen, Amund Bjerve, Morten Hjorth-Jensen, Elin Melby, John Bernhard   Rekstad, Andreas Schiller, Sunniva Siem, and Alexandar Belic, \emph{Entropy in hot 161,162Dy and 171,172Yb nuclei},  \emph{Physical Review C}, C62:024306, 2000. 

\item Magne Guttormsen, Morten Hjorth-Jensen, Elin Melby, John Bernhard Rekstad,   Andreas Schiller, and Sunniva Siem, \emph{Energy shifted level density in the rare earth region}, \emph{Physical Review C}, 61:067302, 2000. 

\item Magne Guttormsen, Morten Hjorth-Jensen, Elin Melby, John Bernhard Rekstad,   Andreas Schiller, and Sunniva Siem, \emph{Entropy of thermally excited particles in nuclei},  \emph{Physical Review C}, 63:024315, 2000. 

\item Henning Heiselberg and Morten Hjorth-Jensen, \emph{Phases of dense matter in neutron stars},  \emph{Physics Reports}, 328:237, 2000. 

\item Anne Holt, Torgeir Engeland, Morten Hjorth-Jensen, and Eivind Osnes, \emph{Applications of realistic effective interactions to the structure of Zr isotopes},  \emph{Physical Review C}, 61:024315, 2000. 

\item M. Tomaselli, M. Hjorth-Jensen, S. Fritzsche, P. Egelhof, S. R. Neumaier,   M. Mutterer, T. Kuhl, A. Dax, and H. Wang, \emph{Matter and charge distributions of 6He and 5,6,7,9Li   within the dynamic-correlation model},  \emph{Physical Review C}, 62:067305, 2000. 

\item Isaac Vidanya, Artur Polls, Angels Ramos, Lars Engvik, and Morten   Hjorth-Jensen, \emph{Properties of $beta$-stable neutron star matter with hyperons},  \emph{Physical Review C}, 62:024315, 2000. 

\item Isaac Vidanya, Artur Polls, Angels Ramos, Morten Hjorth-Jensen, and V. G. J. Stoks, \emph{Strange nuclear matter within the Brueckner-Hartree-Fock   theory},  \emph{Physical Review C}, 61:024315, 2000. 

\item David J. Dean, M. T. Ressell, Morten Hjorth-Jensen, S. E. Koonin, K. Langanke,  and A. P. Zuker, \emph{Shell model Monte Carlo studies of neutron-rich nuclei in the   1s0d-1p0f shells},  \emph{Physical Review C}, 59:2474, 1999. 

\item Henning Heiselberg and Morten Hjorth-Jensen, \emph{Phase transitions in neutron stars and maximum masses},  \emph{Astrophysical Journal Letters}, 525:L45, 1999. 

\item S. M. Vincent, P. H. Regan, S. Mohammadi, D. Blumenthal, M. Carpenter, C. N.   Davids, W. Gelletly, S. S. Ghugre, D. J. Henderson, R. V. F. Janssens,   M. Hjorth-Jensen, B. Kharraja, C. J. Lister, C. J. Pearson, D. Seweryniak,   J. Schwartz, J. Simpson, and D. D. Warner, \emph{Near yrast study of the fpg shell nuclei 58Ni,  61Cu and 61Zn},  \emph{Physical Review C}, 60:064308, 1999. 

\item Elin Melby, Lisbeth Bergholt, Magne Guttormsen, Morten Hjorth-Jensen, Finn   Ingebretsen, Svein Messelt, John Bernhard Rekstad, Andreas Schiller, Sunniva   Siem, and Stein Westad Odegaard, \emph{Observation of thermodynamical properties in the 162Dy,  166Er, 172Yb nuclei},  \emph{Physical Review Letters}, 83:3150, 1999. 

\item Teemu Siiskonen, Jouni Suhonen, and Morten Hjorth-Jensen, \emph{Shell-model effective operators for muon capture in 20Ne},  \emph{Journal of Physics G: Nuclear and Particle Physics}, 25:L55,   1999. 

\item Teemu Siiskonen, Jouni Suhonen, and Morten Hjorth-Jensen, \emph{Towards the solution of the $C_P/C_A$ anomaly in shell-model calculations of muon capture}, \emph{Physical Review C}, 59:R1839, 1999. 

\item Marcello Baldo, Oystein Elgaroy, Lars Engvik, Morten Hjorth-Jensen, and   Hans-Josef Schulze, \emph{Modern nucleon-nucleon potentials and $^3P_2$--$^3F_2$ pairing   in neutron matter},  \emph{Physical Review C}, 58:1921, 1998. 

\item Oystein Elgaroy, Lars Engvik, Morten Hjorth-Jensen, and Eivind Osnes, \emph{Minimal relativity and $^3S_1$--$^3D_1$ pairing in symmetric   nuclear matter},  \emph{Physical Review C}, 57:1069, 1998. 

\item Oystein Elgaroy and Morten Hjorth-Jensen, \emph{Nucleon-nucleon phase shifts and pairing in infinite matter},  \emph{Physical Review C}, 57:1174, 1998. 

\item R. Grzywacz, R. Beraud, C. Borcea, A. Ensallem, M. Glogowski, H. Grawe,   D. Guillemaud-Mueller, Morten Hjorth-Jensen, M. Houry, M. Lewitowicz, A. C.   Mueller, A. Nowak, and A. Plochocki, \emph{New island of $mu s$-isomers in neutron-rich nuclei around the   $Z=28$ and $N=40$ shell closures}, \emph{Physical Review Letters}, 81:766, 1998. 

\item Henning Heiselberg and Morten Hjorth-Jensen, \emph{Phase transitions in rotating neutron stars},  \emph{Physical Review Letters}, 80:5485, 1998. 

\item Anne Holt, Torgeir Engeland, Morten Hjorth-Jensen, and Eivind Osnes, \emph{Shell-model calculations of heavy Sn isotopes},  \emph{Nuclear Physics A}, 634:41, 1998. 

\item Artur Polls, Herbert Muther, Ruprecht Machleidt, and Morten Hjorth-Jensen, \emph{Phaseshift equivalent NN potentials and the deuteron},  \emph{Physics Letters B}, 432:1, 1998. 

\item Jouni Suhonen, Jussi Toivanen, Torgeir Engeland, Morten Hjorth-Jensen, Anne   Holt, and Eivind Osnes, \emph{Study of odd-mass $N=82$ isotones: comparison of the microscopic   quasiparticle-phonon model and the nuclear shell model},  \emph{Nuclear Physics A}, 628:41, 1998. 

\item Isaac Vidanya, Artur Polls, Angels Ramos, and Morten Hjorth-Jensen, \emph{Hyperon properties in finite nuclei using realistic $YN$ interactions},  \emph{Nuclear Physics A}, 644:201, 1998. 

\item G. N. White, N. J. Stone, J. Rikovska, Y. Koh, J. Copell, T. J. Giles, I. S. Towner, B. A. Brown, S. Ohya, Birger Fogelberg, L. Jacobsson, P. Rahkila, and Morten Hjorth-Jensen, \emph{Ground state magnetic dipole moment of 135I},  \emph{Nuclear Physics A}, 644:277, 1998. 

\item Fabio V. de Blasio, Morten Hjorth-Jensen, Oystein Elgaroy, Lars Engvik,   Gianluca Lazzari, Marcello Baldo, and Hans-Josef Schulze, \emph{Coherence lengths of neutron superfluids},  \emph{Physical Review C}, 56:2332, 1997. 

\item Lars Engvik, Morten Hjorth-Jensen, Ruprecht Machleidt, Herbert Muther, and   Artur Polls, \emph{Modern nucleon-nucleon potentials and symmetry energy in   infinite matter},  \emph{Nuclear Physics A}, 627:85, 1997. 

\item Lars Engvik, Morten Hjorth-Jensen, Eivind Osnes, and T. Kuo, \emph{Ring-diagram calculations of nuclear matter with different model   spaces},  \emph{Nuclear Physics A}, 622:553, 1997. 

\item Anne Holt, Torgeir Engeland, Morten Hjorth-Jensen, Eivind Osnes, and Jouni   Suhonen, \emph{The structure of the $N=82$ isotones with realistic effective   interactions},  \emph{Nuclear Physics A}, 618:107, 1997. 

\item N. Sandulescu, Roberto Liotta, Jan Blomqvist, Torgeir Engeland, Morten   Hjorth-Jensen, Anne Holt, and Eivind Osnes, \emph{Generalized seniority scheme in light tin isotopes},  \emph{Physical Review C}, 55:2708, 1997. 

\item Lars Engvik, Morten Hjorth-Jensen, Eivind Osnes, G. Bao, and Erlend Ostgaard, \emph{Asymmetric Nuclear Matter and Neutron Star Properties},  \emph{Astrophysical Journal}, 469:794, 1996. 

\item Alessandro Drago, Umberto Tambini, and Morten Hjorth-Jensen, \emph{Massive quarks in neutron stars},  \emph{Physics Letters B}, 380:13, 1996. 

\item Oystein Elgaroy, Lars Engvik, Morten Hjorth-Jensen, and Eivind Osnes,  \emph{Model-space approach to $^1S_0$ neutron and proton pairing in   neutron star matter with the Bonn meson-exchange potentials},  \emph{Nuclear Physics A}, 604:466, 1996. 

\item Oystein Elgaroy, Lars Engvik, Morten Hjorth-Jensen, and Eivind Osnes, \emph{Superfluidity in beta-stable neutron star matter},  \emph{Physical Review Letters}, 77:1428, 1996. 

\item Oystein Elgaroy, Lars Engvik, Morten Hjorth-Jensen, and Eivind Osnes, \emph{Triplet pairing of neutrons in $beta$-stable neutron star   matter}.  \emph{Nuclear Physics A}, 607:425, 1996. 

\item Oystein Elgaroy, Lars Engvik, Eivind Osnes, Fabio V. de Blasio, Gianluca Lazzari, and Morten Hjorth-Jensen, \emph{Emissivities of neutrinos in neutron stars},  \emph{Physical Review Letters}, 76:1994, 1996. 

\item Oystein Elgaroy, Lars Engvik, Eivind Osnes, Fabio V. de Blasio, Gianluca   Lazzari, and Morten Hjorth-Jensen, \emph{Superfluidity and neutron star crust matter},  \emph{Physical Review D. Particles and fields}, 54:1848, 1996. 

\item Morten Hjorth-Jensen, Herbert Muther, Artur Polls, and Angels Ramos, \emph{Self-energy of $Lambda$ in finite nuclei},  \emph{Nuclear Physics A}, 605:458, 1996. 

\item Morten Hjorth-Jensen, Eivind Osnes, Herbert Muther, and Artur Polls, \emph{Comparison of the effective interaction to various orders in   different mass regions}, \emph{Journal of Physics G: Nuclear and Particle Physics}, 22:321,   1996. 

\item Morten Hjorth-Jensen, T. Kuo, and Eivind Osnes, \emph{Realistic effective interactions for nuclear systems},  \emph{Physics Reports}, 261:125, 1995. 

\item G. Bao, Lars Engvik, Morten Hjorth-Jensen, Eivind Osnes, and Erlend Ostgaard, \emph{New equations of state for neutron stars},  \emph{Nuclear Physics A}, 575:707, 1994. 

\item P. J. Ellis, Torgeir Engeland, Morten Hjorth-Jensen, Anne Holt, and Eivind   Osnes, \emph{Convergence properties of the effective interaction},  \emph{Nuclear Physics A}, 573:216, 1994. 

\item Lars Engvik, Morten Hjorth-Jensen, Eivind Osnes, G. Bao, and Erlend Ostgaard, \emph{Asymmetric nuclear matter and neutron star properties},  \emph{Physical Review Letters}, 73:2650, 1994. 

\item Morten Hjorth-Jensen, Herbert Muther, and Artur Polls, \emph{Width of the $\Delta$ resonance in nuclei},  \emph{Physical Review C}, 50:501, 1994. 

\item Torgeir Engeland, Morten Hjorth-Jensen, Anne Holt, and Eivind Osnes, \emph{The structure of the neutron deficient Sn isotopes},  \emph{Physical Review C}, 48:R535, 1993. 

\item Morten Hjorth-Jensen, Marcello Borromeo, Herbert Muther, and Artur Polls, \emph{Isobar contributions to the imaginary part of the optical-model   potential for finite nuclei},  \emph{Nuclear Physics A}, 551:580, 1993. 

\item Morten Hjorth-Jensen, Mariana Kirchbach, Dan Olof Riska, and Kazuo Tsushima, \emph{Nuclear renormalization of the isoscalar axial coupling   constants},  \emph{Nuclear Physics A}, 563:525, 1993. 

\item Morten Hjorth-Jensen, Torgeir Engeland, Anne Holt, and Eivind Osnes, \emph{Effective interactions for valence-hole nuclei with modern   meson-exchange potential models},  \emph{Nuclear Physics A}, 541:105, 1992. 

\item Morten Hjorth-Jensen, Eivind Osnes, and T. Kuo, \emph{Effective interactions for valence-hole nuclei with modern   meson-exchange potential models},  \emph{Nuclear Physics A}, 540:145, 1992. 

\item Morten Hjorth-Jensen, Eivind Osnes, and Herbert Muther, \emph{Folded-Diagram effective interaction with the Bonn meson-exchange potential model},  \emph{Annals of Physics}, 213:102, 1992. 

\item Morten Hjorth-Jensen and Kjell Aashamar, \emph{Oscillator strengths and lifetimes for low-lying terms in the Al isoelectronic sequence}.  \emph{Physica Scripta}, 42:309, 1990. 

\item Morten Hjorth-Jensen and Eivind Osnes, \emph{Number-conserving sets and effective interactions through third order for mass-18 with the Bonn potential},  \emph{Physica Scripta}, 41:207, 1990. 

\item Morten Hjorth-Jensen, Eivind Osnes, Herbert Muther, and K. W. Schmid, \emph{Choice of single-particle potential and the convergence of the   effective interaction}, \emph{Physics Letters B}, 248:243, 1990. 

\item Morten Hjorth-Jensen and Eivind Osnes, \emph{Effective interactions through third order for mass-18 nuclei with the Paris potential}, \emph{Physics Letters B}, 228:281, 1989.   
\end{enumerate}

\noindent
\paragraph{Contributions to Conference and Workshop Proceedings (refereed and non-refereed).}
\begin{enumerate}
\item Osnes, E, Engeland, T, and Hjorth-Jensen, M, \emph{Large-scale shell-model study of Sn Isotopes}, European Journal of Physics Web of Conferences \textbf{95},01010 (2015)

\item Malthe-Sørenssen, Anders; Hjorth-Jensen, Morten; Langtangen, Hans Petter; Mørken, Knut Martin. \emph{Integrasjon av beregninger i fysikkundervisningen}, UNIPED, 38:303, 2015.

\item Engeland, Torgeir; Hjorth-Jensen, Morten; Kartamyshev, Maxim; Osnes, Eivind.  The Kuo–Brown effective interaction: From 18O to the Sn isotopes. Nuclear Physics A, 928:, 2014 

\item Takayanagi, K, Tsunoda, N, Hjorth-Jensen, M, Otsuka, T, Effective Hamiltonian in non-degenerate model space. Journal of Physics, Conference Series, 445:012003, 2013, DOI: 10.1088/1742-6596/445/1/012003.  

\item Hagen G., Papenbrock T., Hjorth-Jensen M., Jansen G., Machleidt R., Living at the edge of stability: the role of continuum and three-nucleon forces. Edited by: Hamilton, JH; Ramayya, AV.  FISSION AND PROPERTIES OF NEUTRON-RICH NUCLEI, ICFN5, Pages: 400-400, Published: 2013. Conference: 5th International Conference on Fission and Properties of Neutron-Rich Nuclei. Date: NOV 04-10, 2012, (World Scientific, Singapore, 2013)

\item DiJulio D.D. et al, Shell model based Coulomb excitation gamma-ray intensity calculations in Sn-107, PHYSICA SCRIPTA, Volume: T150, Article Number: 014012, DOI: 10.1088/0031-8949/2012/T150/014012, Published: OCT 2012

\item DiJulio D.D. et al, Sub-barrier Coulomb excitation of Sn-107. Edited by:Freeman, S; Andreyev, A; Bruce, A; Deacon, A; Jenkins, D; Joss, D; MacGregor, D; Regan, P; Simpson, J; Tungate, G; Wadsworth, R; Watts, D, RUTHERFORD CENTENNIAL CONFERENCE ON NUCLEAR PHYSICS, 2011, Journal of Physics Conference Series, Volume: 381, Article Number: 012073, DOI: 10.1088/1742-6596/381/1/012073, Published: 2012

\item Brown B.A., Signoracci A., and Hjorth-Jensen, M., Configuration interactions constrained by energy density functionals. Edited by:Covello, A; Gargano, A, 10TH INTERNATIONAL SPRING SEMINAR ON NUCLEAR PHYSICS: NEW QUESTS IN NUCLEAR STRUCTURE, Journal of Physics Conference Series, Volume: 267, Article Number: 012028, DOI: 10.1088/1742-6596/267/1/012028, Published: 2011

\item Tsunoda, Naofumi; Otsuka, Takahuro; Tsukiyama, Koshiroh; Hjorth-Jensen, Morten. Tensor force in effective interaction of nuclear force. Journal of Physics, Conference Series 2011 ;Volume 267.

\item Barbieri, Carlo; Hjorth-Jensen, Morten; Giusti, C; Pacati, FD. ONE- AND TWO-NUCLEON STRUCTURE FROM GREEN'S FUNCTION THEORY. Modern Physics Letters A 2010 ;Volume 25.(21-23) p. 1927-1930

\item Ekstrom, A; Cederkall, Joakim; Fahlander, Claes; Hjorth-Jensen, Morten; Engeland, Torgeir; Butler, PA; Davinson, T; Eberth, J; Finke, F; Görgen, Andreas; Gorska, M; Hurst, AM; Ivanov, O; Iwanicki, J; Koster, U; Marsh, BA; Mierzejewski, J; Reiter, P; Siem, Sunniva; Sletten, G; Stefanescu, I; Tveten, Gry Merete; Van de Walle, J; Voulot, D; Warr, N; Weisshaar, D; Wenander, F; Zielinska, M; Blazhev, A.  Coulomb excitation of the odd-odd isotopes 106, 108In. European Physical Journal A 2010 ;Volume 44. p. 355-361

\item Honma, Micho; Otsuka, Takahuro; Mizusaki, T.; Hjorth-Jensen, Morten.  Recent Progress in Shell-Model Calculations for pfg-shell Nuclei. AIP Conference Proceedings 2010 ;Volume 1235. p. 384-390

\item Otsuka, Takaharu; Tsunoda, Naofumi; Tsukiyama, Koshiroh; Suzuki, Toshio; Honma, Michio; Utsuno, Yutaka; Hjorth-Jensen, Morten; Holt, Jason; Schwenk, Achim.  Hadronic Interaction and Exotic Nuclei. AIP Conference Proceedings 2009 ;Volume 1165. p. 47-52

\item Algin, E; Schiller, A; Voinov, A; Agvaanluvsan, U; Belgya, T; Bernstein, LA; Brune, CR; Chankova, Rosita; Garrett, PE; Grimes, SM; Guttormsen, Magne Sveen; Hjorth-Jensen, Morten; Hornish, MJ; Johnson, CW; Massey, T; Mitchell, GE; Rekstad, John Bernhard; Siem, Sunniva; Younes, W.  Bulk properties of iron isotopes. Physics of Atomic Nuclei 2007 ;Volume 70. p. 1634-1639

\item Hjorth-Jensen, Morten.  Computational Quantum Mechanics. META 2007 ;Volume 2. p. 10-15

\item Hjorth-Jensen, Morten.  High-performance computing and basic education in computational Science. META 2007 (1) p. 18-19

\item Gorska, M.; Grawe, H.; Banu, A.; Burger, A.; Doornenbal, P.; Gerl, J,; Hjorth-Jensen, Morten; Hübel, H.; Nowacki, F.; Otsuka, Takahuro; reiter, P.  Nuclear structure far off stability – New results from RISING. Journal of Physics, Conference Series 2006 ;Volume 49. p. 59-64

\item Guttormsen, Magne; Agvaanluvsan, Undraa; Chankova, Rositsa; Hjorth-Jensen, Morten; Rekstad, John Bernhard; Schiller, Andreas; Siem, Sunniva; Larsen, Ann-Cecilie; Syed, Naeem Ul Hasan; Voinov, Alexander.  Single particle entropy in heated nuclei. AIP Conference Proceedings 2006 ;Volume 831. p. 162-166

\item Honma, Micho; Otsuka, Takahuro; Mizusaki, T.; Hjorth-Jensen, Morten.  Effective interaction for f5pg9-shell nuclei and two-neutrino double beta-decay matrix elements. Journal of Physics, Conference Series 2006 ;Volume 49. p. 45-50

\item Papenbrock, T.; Dean, David Jarvis; Gour, J. R.; Hagen, G.; Hjorth-Jensen, Morten; Piecuch, P.; Wloch, M.  Coupled-cluster theory for nuclei. International journal of modern physics B 2006 ;Volume 20. p. 5338-5345

\item Schiller, Andreas; Agvaanluvsan, Undraa; Algin, Emel; Bagheri, Asadolla; Chankova, Rosita; Guttormsen, Magne; Hjorth-Jensen, Morten; Rekstad, John Bernhard; Siem, Sunniva; Sunde, Ann-Cecilie; Voinov, Alexander.  Nuclear thermodynamics below particle threshold. AIP Conference Proceedings 2005 (777) p. 216-228

\item Wloch, Marta; Dean, David J.; Grour, Jeffrey; Piecuch, Piotr; Hjorth-Jensen, Morten; Papenbrock, Thomas; Kowalski, Karol.  Ab Initio Coupled-Cluster calculations for Nuclei using Methods of Quantum Chemistry. European Physical Journal A  Volume: 25  485-488   Published: 2005

\item Dean, DJ, Hjorth-Jensen, M, Kowalski, K, Piecuch, P, Wloch, M, Coupled-cluster theory for nuclei, Condensed Matter Theories, VOL 20, Volume: 20  Pages: 89-97, Published: 2006

\item Barrett, BR; Dean, DJ; Hjorth-Jensen, Morten; Vary, JP.  Nuclear forces and the quantum many-body problem - Preface. Journal of Physics G: Nuclear and Particle Physics 2005 ;Volume 31.

\item Honma. M., Otsuka. T, Mizusaki T, Hjorth-Jensen M, Brown BA, Effective interaction for nuclei of A=50-100 and Gamow-Teller properties, Edited by:Suzuki, T; Otsuka, T; Ichimura, M, International Symposium on correlation dynamics in nuclei, Journal of Physics Conference Series, Volume: 20  Pages: 7-12, DOI: 10.1088/1742-6596/20/1/002, Published: 2005. Conference: International Symposium on Correlation Dynamics in Nuclei, Univ Tokyo, Sanjo Kaikan, JAPAN

\item Piecuch, P, Wloch, M,  Gour, JR, Dean, DJ, Hjorth-Jensen M,  Papenbrock T., Bridging quantum chemistry and nuclear structure theory: Coupled-cluster calculations for closed- and open-shell nuclei, Edited by:Zelevinsky, V, Nuclei and Mesoscopic Physics, AIP Conference Proceedings, Volume: 777  Pages: 28-45, Published: 2005, Conference: Workshop on Nuclei and Mesoscopic Physics, Michigan State Univ, NSCL, E Lansing, MI, OCT 23-26, 2004

\item Schiller A. et al,  Nuclear thermodynamics below particle threshold, Edited by:Zelevinsky, V, Nuclei and Mesoscopic Physics, AIP Conference Proceedings, Volume: 777, Pages: 216-228, Published: 2005. Conference: Workshop on Nuclei and Mesoscopic Physics, Michigan State Univ, NSCL, E Lansing, MI, OCT 23-26, 2004

\item Hagen, G; Hjorth-Jensen, M; Vaagen, Jan S.  State-dependent interactions for the Gamow shell model. Journal of Physics G: Nuclear and Particle Physics 2005 ;Volume 31. 

\item Wloch, Marta; Grour, Jeffrey; Piecuch, Piotr; Dean, David J.; Hjorth-Jensen, Morten; Papenbrock, Thomas.  Coupled-cluster calculations for ground and excited states of closed- and open-shell nuclei using methods of quantum chemistry. Journal of Physics G: Nuclear and Particle Physics 2005 ;Volume 31.  S1291-S1299

\item Belic, Alexandar; Dean, David J.; Hjorth-Jensen, Morten.  Pairing correlations and transitions in nuclear systems. Nuclear Physics A 2004 ;Volume 731. p. 381-391

\item Dean, DJ, Gour, JR, Hagen, G, Hjorth-Jensen, M, Kowalski K, Papenbrock, T, Piecuch, P, Wloch, M, Nuclear structure calculations with coupled cluster methods from quantum chemistry, Nuclear Physics A, Volume: 752  Pages: 299C-308C, DOI: 10.1016/j.nuclphysa.2005.02.041, Published: Apr 18 2005

\item Dean, DJ, Hjorth-Jensen, M, Kowalski, K, Papenbrock, T, Wloch, M, Piecuch, P, Coupled cluster approaches to nuclei, ground states and excited states, Edited by:Covello, A, Key topics in nuclear structure, Pages: 147-157, Published: 2005, Conference: 8th International Spring Seminar on Nuclear Physics, Location: Paestum, Italy, May 23-27, 2004

\item Brown, B.A.; Clement, R.; Schatz, H.; Giansiracusa, J.; Richter, W.A.; Hjorth-Jensen, Morten; Kratz, K.L.; Pfeiffer, B.; Walters, W.B.  Nuclear structure theory for the astrophysical rp-process and r-process. Nuclear Physics A 2003 ;Volume 719. p. 177-184

\item Dean, David J.; Hjorth-Jensen, Morten.  Toward coupled-cluster implementations in nuclear structure. AIP Conference Proceedings 2003 ;Volume 656. p 197-204

\item Schiller A. et al, Radiative strength functions and level densities, Pages: 432-440 (2003), Conference: 11th International Symposium on Capture Gamma-Ray Spectroscopy and Related Topics, Pruhonice, Czech Republic, sep 02-06, 2002

\item Hjorth-Jensen, Morten.  Pairing correlations, from neutron stars to finite nuclei. Progress of Theoretical Physics Supplement 2002 ;Volume 146. p. 289-298

\item Schiller, Andreas; Guttormsen, Magne; Hjorth-Jensen, Morten; Melby, Elin; Rekstad, John Bernhard; Siem, Sunniva. Level density and thermal properties in rare earth nuclei. Physics Atomic Nuclei 2001 ;Volume 64.(7) p. 1186-1193

\item Elgarøy, Øystein; Engeland, Torgeir; Hjorth-Jensen, Morten; Osnes, Eivind.  Pairing correlations in nuclear systems, from infinite nuclear matter to finite nuclei. International journal of modern physics B 2001 ;Volume 15. p. 1501-1509

\item Vidanya, Isaac; Polls, Artur; Ramos, Angels; Engvik, Lars; Hjorth-Jensen, Morten. Hyperon effects on the properties of beta-stable neutron star matter. Nuclear Physics A 2001 ;Volume 691. p. 443-446

\item Rekstad, John Bernhard; Bergholt, Lisbeth; Guttormsen, Magne; Hjorth-Jensen, Morten; Ingebretsen, Finn; Melby, Elin; Messelt, Svein; Schiller, Andreas; Siem, Sunniva; Ødegård, Stein Westad. Measurement of level densities and gamma ray strength functions. AIP Conference Proceedings 2000 ;Volume 529.(1) p. 144-151

\item Engeland T, Hjorth-Jensen M, Osnes E., Effective interactions in medium heavy nuclei, NUCLEAR PHYSICS A 701  Pages: 416C-421C (2002). Conference: 5th International Conference on Radioactive Nuclear Beams, Divonne, France, March 27-APR 01, 2000

\item Engeland T, Hjorth-Jensen M, Holt A, and Osnes E,   Large-scale realistic nuclear structure studies in the Sn-region. Edited by:Covello, A. Conference: 7th International Spring Seminar on Nuclear Physics, Maiori, Italy, MAY 27-31, 2001

\item Schiller et al,  Level density and thermal properties in rare earth nuclei. Conference: International Conference on Nuclear Structure and Related Topics, DUBNA, RUSSIA, JUN 06-10, 2000. PHYSICS OF ATOMIC NUCLEI 64  Pages: 1186-1193 DOI: 10.1134/1.1389540 Published: JUL 2001

\item Lipoglavsek, M, Baktash, C, Carpenter, MP,  et al, First observation of excitation across the Sn-100 core. Conference: Conference on Nuclear Structure 2000 (NS2000), E LANSING, MICHIGAN, AUG 15-19, 2000. Nuclear Physics A 682  Pages: 399C-403C (2001), DOI: 10.1016/S0375-9474(00)00666-7

\item Siiskonen, Teemu; Suhonen, Jouni; Hjorth-Jensen, Morten.  Effective Shell-Model Transition Operators for Muon-Capture Calculations.  Conference: 2nd International Conference on Nonaccelerator New Physics (NANP 99) Location: Joint inst nuclear res, Dubna, Russia Date: JUN 28-JUL 03, 1999. Physics of Atomic Nuclei  Volume: 63   Issue: 7   Pages: 1182-1186   Published: JUL 2000

\item Suhonen, J; Aunola, M; Kortelainen, M; et al., Refined shell-model matrix elements for muon-capture processes. Conference: Workshop on Calculation of Double-Beta-Decay Matrix Elements (MEDEX 99) Location: PRAGUE, CZECH REPUBLIC Date: JUL 20-23, 1999, CZECHOSLOVAK JOURNAL OF PHYSICS  Volume: 50   Issue: 4   Pages: 567-575   Published: APR 2000

\item Siem, S; Schiller, A; Guttormsen, M; et al., Level density and thermal properties in rare earth nuclei. Conference: International Symposium on Exotic Nuclear Structures (ENS 2000) Location: DEBRECEN, HUNGARY Date: MAY 15-20, 2000 ACTA PHYSICA HUNGARICA NEW SERIES-HEAVY ION PHYSICS  Volume: 12   Issue: 2-4   Pages: 299-302   Published: 2000

\item Rekstad, J; Bergholt, L; Guttormsen, M; et al., Measurements of level densities and gamma ray strength functions. Edited by: Wender, S. Conference: 10th International Symposium on Capture Gamma-Ray Spectroscopy and Related Topics Location: SANTA FE, NM Date: AUG 30-SEP 03, 1999. AIP CONFERENCE PROCEEDINGS   Volume: 529   Pages: 144-151   Published: 2000

\item Melby, E; Bergholt, L; Guttormsen, M; et al., Experimental temperature and heat capacity in rare earth nuclei. Conference: International Conference on Achievements and Perspectives in Nuclear Structure Location: IRAKLION, GREECE Date: JUL 11-17, 1999, PHYSICA SCRIPTA  Volume: T88   Pages: 138-140   Published: 2000

\item Stone, N.J.; White, G.N.; Rikovska, J.; Ohya, S.; Giles, T.J.; Towner, I.S.; Brown, B.A.; Fogelberg, Birger; Jacobsson, L.; Hjorth-Jensen, Morten.  NMR/ON nuclear magnetic dipole moments near 132Sn: I. At the shell closure: meson exchange current effects. Hyperfine Interactions 1999 ;Volume 120. p. 645-649

\item White, G.N.; Stone, N.J.; Rikovska, J.; Ohya, S.; Giles, T.J.; Towner, I.S.; Brown, B.A.; Fogelberg, Birger; Jacobsson, L.; Hjorth-Jensen, Morten.  New on-line NMR/ON nuclear magnetic dipole moments near 132Sn: II variation with proton and neutron number: shell model treatment of `collective' effects. Hyperfine Interactions 1999 ;Volume 120. p. 651-655

\item Elgaroy, O; Hjorth-Jensen, M, Properties of pairing correlations in infinite nuclear matter.Edited by: daProvidencia, J; Malik, FB. Conference: 21st International Workshop on Condensed Matter Theories Location: LUSO, PORTUGAL Date: SEP 22-26, 1997. CONDENSED MATTER THEORIES, Volume: 13   Pages: 381-391   Published: 1998

\item Drago, Alessandro; Hjorth-Jensen, Morten; Tambini, Ubaldo. Neutron stars and massive quark matter. Progress in Particle and Nuclear Physics 1996 ;Volume 36. p. 407-408

\item Engeland, T, Hjorth-Jensen, M, Holt, A., and Osnes, E.,  Extensive nuclear structure calculations in the tin isotopes. Edited by: Klapdor Kleingrothaus, HV and Stoica, S. Conference: International Workshop on Double-Beta Decay and Related Topics Location: ECT*, TRENTO, ITALY Date: APR 24-MAY 05, 1995, (World Scientific, Singapore, 1996), Pages: 421-451

\item Engeland, Torgeir; Hjorth-Jensen, Morten; Holt, Anne; Osnes, Eivind.  Large shell model calculations with realistic effective interactions. Physica Scripta 1995 ;T56. p. 58-66 Conference: International Symposium on New Nuclear Structure Phenomena in the Vicinity of Closed Shells Location: STOCKHOLM, SWEDEN, AUG 30-SEP 03, 1994 

\item Hjorth-Jensen, Morten; Engeland, Torgeir; Holt, Anne; Osnes, Eivind.  Perturbative many-body approaches to finite nuclei. Physics reports 1994 ;242. p. 37-69 By: Conference: International Conference on Realistic Nuclear Structure, to Celebrate the 60th Birthday of TTS Kuo, Suny Stony Brook, NY, USA,  May 28-30, 1992

\item Holt, Anne; Engeland, Torgeir; Hjorth-Jensen, Morten; Osnes, Eivind.  The structure of the neutron deficient Sn isotopes. Nuclear Physics A 1994 ;Volume 570. p. 137c-144c Conference: International Symposium on Nuclear Structure Physics Today Location: Chung Yuan Christian Univ, Chungli, Taiwan Date: MAY 11-15, 1993 

\item Hjorth-Jensen, Morten,  Microscopic nuclear-structure calculations with modern meson-exchange potentials, Edited by A. Covello, 3rd international spring seminar on nuclear physics, Ischia, Italy, May 21-25, 1990, (World Scientific Singapore, 1991), pages 87-97 
\end{enumerate}

\noindent
\paragraph{Talks, lectures and seminars at workshops, conferences, schools  and institute colloquiua.}
\begin{enumerate}
\item Hjorth-Jensen, Morten, \href{{https://mhjensenseminars.github.io/MachineLearningTalk/doc/pub/quantumcomputing/html/quantumcomputing-reveal.html}}{Quantum Computing and Quantum Mechanics for Many Interacting Particles}, Gemini center at Sintef seminar, Oslo, March 3, 2021

\item Hjorth-Jensen, Morten, \href{{https://www.youtube.com/watch?v=gzi9n8nGZSM&ab_channel=NITheCSKZN}}{Machine Learning and Quantum Mechanics for Many Interacting Particles, NITheP Colloquium, South Africa}, Monday, 8 February 2021

\item Hjorth-Jensen, Morten, Machine Learning meets Nuclear Physics, XAI seminar series: Explaining what goes on inside DNN/AI, SINTEF/University of Oslo, Norway, December 8, 2020.

\item Hjorth-Jensen, Morten, Machine Learning meets Nuclear Physics, University of the Western Cape, South Africa, November 30- December 4, 2020, online workshop \textbf{Tastes of Nuclear Physics}  \href{{http://nuclear.uwc.ac.za/index.php/tnp2020/}}{\nolinkurl{http://nuclear.uwc.ac.za/index.php/tnp2020/}}

\item Hjorth-Jensen, Morten, Machine Learning meets Nuclear Physics, Institute colloquium at the Department of Physics, University of Padova, Italy, October 13, 2020.

\item Hjorth-Jensen, Morten, Machine Learning and Quantum Mechanics for Many Interacting Particles, UiO, March 3, 2020  \href{{https://www.mn.uio.no/math/english/research/groups/statistics-data-science/events/seminars/hjorth-jensen.html}}{\nolinkurl{https://www.mn.uio.no/math/english/research/groups/statistics-data-science/events/seminars/hjorth-jensen.html}}

\item Hjorth-Jensen, Morten, Lecture on Nuclear Physics at the NS3 school, FRIB, Michigan State University, May 15, 2019. Main organizer Artemis Spyrou, Michigan State University.

\item Morten Hjorth-Jensen, Solving Quantum Mechanical Many-body Problems with Machine Learning Algorithms, Chalmers Tekniska Høgskola, Gøteborg, Sverige, October 28, 2019.

\item Hjorth-Jensen, Morten, Integrating a Computational Perspective in Physics (and Science) Courses, October 23, 2019. Ole Rømer Colloquium, Department of Physics and Astronomy, University of Århus, Denmark \href{{https://phys.au.dk/en/news/item/artikel/ole-roemer-colloquium-morten-hjort-jensen-tba/}}{\nolinkurl{https://phys.au.dk/en/news/item/artikel/ole-roemer-colloquium-morten-hjort-jensen-tba/}}

\item Morten Hjorth-Jensen, Solving Quantum Mechanical Many-body Problems with Machine Learning Algorithms, University of Surrey, Guildford, UK, October 1, 2019.

\item Hjorth-Jensen, Morten, Machine Learning and Quantum Mechanics for Many Interacting Particles, University of Ohio, Athens, April 16, 2019  \href{{https://mhjensenseminars.github.io/MachineLearningTalk/doc/pub/unitn/html/uniohio-reveal.html}}{\nolinkurl{https://mhjensenseminars.github.io/MachineLearningTalk/doc/pub/unitn/html/uniohio-reveal.html}}

\item Hjorth-Jensen, Morten, Machine Learning and Quantum Mechanics for Many Interacting Particles, University of Trento, Italy, March 12, 2019, 2019  \href{{https://mhjensenseminars.github.io/MachineLearningTalk/doc/pub/unitn/html/unitn-reveal.html}}{\nolinkurl{https://mhjensenseminars.github.io/MachineLearningTalk/doc/pub/unitn/html/unitn-reveal.html}}

\item Hjorth-Jensen, Morten, "Integrating Computations in Physics Courses, Workshop on New Horizons in Teaching Science: 18th-19th, June 2018, University of Messina, Italy"

\item Hjorth-Jensen, Morten, \href{{https://indico.fnal.gov/event/15794/page/11}}{Nuclear Structure studies from decay spectroscopy, Decay Station Workshop, NSCL/FRIB Michigan State University, January 25-26, 2018}

\item Hjorth-Jensen, Morten, \href{{http://www.nucleartheory.net/NPG/recent_seminars.htm}}{Computing in Science Education; how to integrate computing in Science courses across disciplines, seminar at the University of Surrey, UK, November 28 2017}

\item Hjorth-Jensen, Morten, \href{{https://www.sif.it/attivita/congresso/103}}{Computing in Physics Education, Invited talk at the 103rd National congress of the Italian Physical Society}, Trento, September 11-15, 2017, Italy

\item Alex Brown, Alexandra Gade, Morten Hjorth-Jensen, Gustav Jansen, Robert Grzywacz, Nuclear Talent course on Nucleartheory for Nuclear Structure Experiments, July 3-21 2017. \href{{https://github.com/NuclearTalent/NuclearStructure}}{Main organizer and teacher with in total fifteen hours of lectures}. 

\item Hjorth-Jensen, Morten, \href{{https://icer-acres.msu.edu/summer-2017/schedule/}}{High performance computing in Nuclear Physics}, Lecture at the \emph{Advanced Computational Research Experience} at Michigan State University, East Lansing, Michigan, June 1, 2017.

\item Hjorth-Jensen, Morten, \href{{https://icer-acres.msu.edu/summer-2017/schedule/}}{How to write good code}, Lecture at the \emph{Advanced Computational Research Experience} at Michigan State University, East Lansing, Michigan, May 24, 2017.

\item Hjorth-Jensen, Morten, \href{{http://www.dnva.no/c26754/kalender/index.html?year=2017&month=3&day=16}}{Minnetalen over Hans Petter Langtangen}, Det Norske Vitenskapsakademiet, Oslo, Norway, March 16, 2017.

\item Hjorth-Jensen, Morten, \href{{https://science.nd.edu/events/2017/01/30/nuclear-physics-seminar-prof-morten-hjorth-jensen/}}{Living on the edge of stability, challenges to nuclear theory in the FRIB era}, Nuclear Physics seminar, Unversity of Notre Dame, Notre Dame, IN 46556, USA, January 30, 2017 

\item Hjorth-Jensen, Morten, \href{{http://rafael.ujf.cas.cz/school}}{Computational Nuclear Physics and Post Hartree-Fock Methods. Configuration Interaction Theory, Many-Body Perturbation Theory and Coupled Cluster Theory}, five lectures at 28th Indian-Summer School on Ab Initio Methods in Nuclear Physics, Prague, Czech Republic, August 29 - September 2, 2016.

\item Hjorth-Jensen, Morten, \href{{http://compphysics.github.io/CompPhysUTunis/doc/web/course.html}}{Computational Physics and Quantum Mechanical Systems}, one week course on Computational Physics at the University of Tunis El Manar, Tunis, Tunisia, May 16-20, 2016. In total 15 hours of lectures and 15 hours of computer lab and exercises. 

\item Hjorth-Jensen, Morten, \href{{https://t2.lanl.gov/seminars/?section=abstract&number=-10&year=2016}}{Correlations in many-body systems; from condensed matter physics to nuclear physics}, T-2, Nuclear and Particle Physics, Astrophysics and Cosmology, Los Alamos National Laboratory, New Mexico, USA, Tuesday, April 12, 2016

\item Hjorth-Jensen, Morten, \href{{http://mhjensenseminars.github.io/EducationalSeminars/doc/pub/cse/html/cse-reveal.html}}{Integrating a Computational Perspective in the Basic Science Education}, Department of Physics Colloquium at Central  Michigan University,  Kalamazoo, Michigan, USA, April 4, 2016

\item Co-organizer with Giuseppina Orlandini and Alejandro Kievsky of Nuclear Talent course \href{{https://groups.nscl.msu.edu/jina/talent/wiki/Course_3}}{Few-body methods and nuclear reactions}, ECT*, Trento, Italy, July 20-August 7 2015

\item Carlo Barbieri, Wim Dickhoff, Gaute Hagen, Morten Hjorth-Jensen, and Artur Polls, Nuclear Talent course on Many-body methods for nuclear physics, GANIL, Caen, France, July 5-25 2015. \href{{http://nucleartalent.github.io/Course2ManyBodyMethods/doc/web/course.html}}{Main organizer and teacher with in total five hours of lectures}. 

\item Hjorth-Jensen, Morten, ECT* \href{{http://www.ectstar.eu/node/1287}}{Doctoral Training Program 2015 on Computational Nuclear Physics}, April 13- May 22, ECT*, Trento, Italy. I taught the last week of the lecture series. In total I have ten one hour lectures. 

\item Hjorth-Jensen, Morten, \href{{http://clarkfest15.physics.wustl.edu/Docs/intro.php}}{Correlations in many-body systems, from condensed matter physics to nuclear physics}, invited talk at Clarkfest 15,  conference in honor of John W Clark, Wayman Crow Professor of Physics, Washington University in St. Louis, Missouri, April 27-28 2015.

\item Hjorth-Jensen, Morten, \href{{http://www.event.iastate.edu/event/35628}}{Correlations in many-body systems, from condensed matter physics to nuclear physics}, Nuclear Physics Seminar, Iowa State University, Ames, Iowa, April 22 2015.

\item Hjorth-Jensen, Morten, Nuclear physics education and the national FRIB theory center, plus some cool ways to organize your lectures, special seminar, Iowa State University, Ames, Iowa, April 23 2015.

\item Hjorth-Jensen, Morten, Integrating a Computational Perspective in the Basic Science Education, Special Lectures and Events, Notre Dame University, South Bend, Indiana, March 30 2015.

\item Hjorth-Jensen, Morten, Computing in Science Education.  Integrating a Computational Perspective in the Basic Science Education, Physics Colloquium, Central Michigan University, Mt Pleasant, March 19 2015.

\item Hjorth-Jensen, Morten, From Nuclei to Neutron Stars: Why Is Matter Stable? Physics Colloquium, Ohio University, Athens, Ohio,  February 27 2015.

\item Hjorth-Jensen, Morten, Computing in Science Education.  Integrating a Computational Perspective in the Basic Science Education, condensed matter seminar, Ohio University, Athens, Ohio,  February 26 2015.

\item Hjorth-Jensen, Morten, Theory challenges around 78Ni and 132Sn, invited talk at  RIBSS Center retreat and CSAC, Michigan State University, June 11-13 2014.

\item Hjorth-Jensen, Morten, Living at the edge of stability, understanding the limits of the nuclear landscape, Institute colloquium, Department of Physics, Lousiana State University, Baton Rouge, Lousiana, April 3 2014.

\item Hjorth-Jensen, Morten, Computing in Science education, how to introduce a computational perspective in the basic science education, special colloquium Department of Physics, Lousiana State University, Baton Rouge, Lousiana, April 4 2014.

\item Hjorth-Jensen, Morten, Correlations in Nuclei and Quantum Dots, invited talk at  The Fourth Conference on NUCLEI and MESOSCOPIC PHYSICS, Michigan State University, May 5-9 2014.

\item Hjorth-Jensen, Morten, Nuclear Talent School in Nuclear Astrophysics, co-organizer with Richard Cyburt and Hendrik Schatz of the Nuclear Talent course on Nuclear Astrophysics,  Michigan State University, May 26 - June 13, 2014. 

\item Hjorth-Jensen, Morten, Nuclear Talent course on Density Functional theories, co-organizer with Scott Bogner, Nicolas Schunck, Dario Vretenar and Peter Ring, European Center for Theoretical Nuclear Physics and Related Areas, Trento, Italy, July 13 -August 1 2014.

\item Hjorth-Jensen, Morten.  Living at the edge of stability, understanding the limits of the nuclear landscape. Institute colloquium Centre Etudes Nucléaires de Bordeaux Gradignan; 2013-12-10 - 2013-12-10

\item Hjorth-Jensen, Morten.  Educating the next generation of nuclear scientists; how can a center like the ECT* aid in developing modern nuclear physics educational programs?. ECT* 20th anniversary colloquium; 2013-09-14 - 2013-09-14

\item Hjorth-Jensen, Morten.  Living at the edge of stability, understanding the limits of the nuclear landscape; computational and algorithmic challenges. XXV IUPAP Conference on Computational Physics, August 20, 2013- August 24, 2013, Moscow, Russia; 2013-08-20 - 2013-08-24

\item Hjorth-Jensen, Morten.  Living at the edge of stability, understanding the nuclear landscape. Theory seminar National Superconducting Cyclotron Laboratory; 2013-03-19 - 2013-03-19

\item Hjorth-Jensen, Morten.  Living on the edge of stability, the limits of nuclear landscape. Physics Division seminar, Argonne National Laboratory, Illinois, USA; 2013-06-05 - 2013-06-05

\item Hjorth-Jensen, Morten.  Living on the edge of stability, the limits of the nuclear landscape. Institute colloquium; 2013-03-22 - 2013-03-22

\item Hjorth-Jensen, Morten.  Living on the edge of stability, understanding the limits of the nuclear landscape. Nuclear Theory in the Supercomputing Era; 2013-05-13 - 2013-05-17

\item Hjorth-Jensen, Morten.  Computing in Science Education. Seminar at college of engineering; 2012-03-15 - 2012-03-15

\item Hjorth-Jensen, Morten.  Computing in Science Education, a new way to teach science?. Institute seminar The Ohio State University; 2012-02-28 - 2012-02-28

\item Hjorth-Jensen, Morten.  Evolution of shell structure in neutron-rich isotopes. Research seminar National Superconducting Cyclotron Laboratory; 2012-03-15 - 2012-03-15

\item Hjorth-Jensen, Morten.  Evolution of shell structure in neutron-rich isotopes and the stability of nuclear matter. Exotic Nuclear Structure from Nucleons; 2012-10-10 - 2012-10-12

\item Hjorth-Jensen, Morten.  Introduction to computational nuclear physics. High-performance computing and computational tools for nuclear physics; 2012-06-24 - 2012-07-13

\item Hjorth-Jensen, Morten.  Lecture 2: Configuration interaction theory. High-performance computing and computational tools for nuclear physics; 2012-06-24 - 2012-07-13

\item Hjorth-Jensen, Morten.  Lectures 3-5: Configuration interaction theory and computational nuclear physics. High-performance computing and computational tools for nuclear physics; 2012-06-24 - 2012-07-13

\item Hjorth-Jensen, Morten.  Shell Structure in Neutron-rich isotopes and the stability of nuclear matter. Berkeley Lab Colloquia 2012; 2012-05-30 - 2012-05-30

\item Hjorth-Jensen, Morten.  Understanding the stability of nuclear matter. Nuclear strcuture seminar The Ohio State University; 2012-02-29 - 2012-02-29

\item Hjorth-Jensen, Morten.  Understanding the stability of nuclear matter. Triangle Nuclear Theory Colloquium; 2012-05-01 - 2012-05-01

\item Hjorth-Jensen, Morten.  Why is matter stable?. Theory of Nuclear Physics Related to the RI Facilities; 2012-05-11 - 2012-05-12

\item Hjorth-Jensen, Morten.  Why is matter stable? Understanding the limits of stability of nuclear matter. Nobel Symposium 152; 2012-06-10 - 2012-06-15

\item Hjorth-Jensen, Morten. Computational environment for Nuclear Structure, Lectures I-V. Lecture series in Nuclear Physics at Universidad Complutense Madrid; 2011-01-17 - 2011-02-09

\item Hjorth-Jensen, Morten.  Computers in Science Education; a new way to teach Science?. Institute seminar; 2011-03-21 - 2011-03-21

\item Hjorth-Jensen, Morten.  Computers in Science Education; a new way to teach Science?. Seminar at Universidad Complutense Madrid; 2011-01-24 - 2011-01-24

\item Hjorth-Jensen, Morten.  From few to many nucleons; a tale on recent advances (and challenges) in nuclear many-body theory. Institute seminar; 2011-03-25 - 2011-03-25

\item Hjorth-Jensen, Morten. Linking nuclear forces with many-body methods, Lecture II. Second MSU--UT/ORNL winter school in nuclear physics; 2011-01-03 - 2011-01-07

\item Hjorth-Jensen, Morten.  Many-body interactions and nuclear structure. Institute seminar National Superconducting Cyclotron laboratory; 2011-01-05 - 2011-01-05

\item Hjorth-Jensen, Morten.  Many-body interactions and nuclear structure. Seminar at Universidad Complutense Madrid; 2011-01-18 - 2011-01-18

\item Hjorth-Jensen, Morten.  Many-body interactions and nuclear structure at the limits of stability. Institute seminar; 2011-03-22 - 2011-03-22

\item Hjorth-Jensen, Morten.  Many-body interactions and nuclear structure at the limits of stability. Nordic Nuclear Physics conference 2011; 2011-06-13 - 2011-06-17

\item Hjorth-Jensen, Morten.  Many-body interactions and nuclear structure at the limits of stability. Nuclear Physics in Astrophysics - V; 2011-04-03 - 2011-04-09

\item Hjorth-Jensen, Morten.  Many-body methods, Lecture III. Second MSU--UT/ORNL winter school in nuclear physics; 2011-01-03 - 2011-01-07

\item Hjorth-Jensen, Morten.  Many-body methods, Lectures IV and V. Second MSU--UT/ORNL winter school in nuclear physics; 2011-01-03 - 2011-01-07

\item Hjorth-Jensen, Morten.  Nuclear structure at the limits of stability. Division of Nuclear Physics Meeting 2011; 2011-10-25 - 2011-10-29

\item Hjorth-Jensen, Morten.  Parallel programming with MPI. The 10th Annual Meeting on High Performance Computing and Infrastructure in Norway; 2011-05-23 - 2011-05-27

\item Hjorth-Jensen, Morten.  Renormalization of nuclear forces, Lecture set I. Second MSU--UT/ORNL winter school in nuclear physics; 2011-01-03 - 2011-01-07

\item Hjorth-Jensen, Morten.  Computers in Science Education. Institute seminar at the university of Trento, Italy; 2010-05-05 - 2010-05-05

\item Hjorth-Jensen, Morten.  Deriving nuclear forces. CERN/Isolde course on nuclear structure theory; 2010-03-01 - 2010-03-04

\item Hjorth-Jensen, Morten.  From few to many nucleons; a tale on recent advances (and challenges) in nuclear many-body theory. Institute seminar; 2010-07-22 - 2010-07-22

\item Hjorth-Jensen, Morten.  From few to many nucleons; a tale on recent advances (andchallenges) in nuclear many-body theory. Spiral2 week 2010; 2010-01-25 - 2010-01-29

\item Hjorth-Jensen, Morten.  High-performance computing and quantum mechanical problems. Future needs for eInfrastructure for Norwegian research, March 19 2010; 2010-03-19 - 2010-03-19

\item Hjorth-Jensen, Morten.  Many-body interactions and nuclear structure. New faces of atomic nuclei; 2010-11-15 - 2010-11-17

\item Hjorth-Jensen, Morten.  Many-body methods for nuclear structure studies. CERN/Isolde course on nuclear structure theory; 2010-03-01 - 2010-03-04

\item Hjorth-Jensen, Morten.  Many-body theory for exotic nuclei and coupled-cluster theory. CERN/Isolde course on nuclear structure theory; 2010-03-01 - 2010-03-04

\item Hjorth-Jensen, Morten.  Modern theory of effective interactions. Zakopane Conference On Nuclear Physics 2010; 2010-08-30 - 2010-09-05

\item Hjorth-Jensen, Morten.  Overview of nuclear forces. CERN/Isolde course on nuclear structure theory; 2010-03-01 - 2010-03-04

\item Hjorth-Jensen, Morten.  Renormalizing nuclear forces. CERN/Isolde course on nuclear structure theory; 2010-03-01 - 2010-03-04

\item Hjorth-Jensen, Morten.  Role of many-body forces in nuclei. CERN/Isolde course on nuclear structure theory; 2010-03-01 - 2010-03-04

\item Hjorth-Jensen, Morten.  Role of the tensor force in nuclear spectra. CERN/Isolde course on nuclear structure theory; 2010-03-01 - 2010-03-04

\item Hjorth-Jensen, Morten.  Shell structure and modern effective interactions. International Nuclear Physics Conference 2010; 2010-07-04 - 2010-07-09

\item Hjorth-Jensen, Morten.  Theory of shell-model studies for nuclei. CERN/Isolde course on nuclear structure theory; 2010-03-01 - 2010-03-04

\item Hjorth-Jensen, Morten.  Ab initio methods in nuclear physics. Overview and recent achievements. Assemblée Générale des Théoriciens, 15 et 16 octobre, IPN-Orsay; 2009-10-15 - 2009-10-16

\item Hjorth-Jensen, Morten.  Can we do ab initio calculations for nuclei beyond A=16?. 7th Biennal Yale Nuclear structure workshop; 2009-06-18 - 2009-06-21

\item Hjorth-Jensen, Morten.  Computers in Science Education. Institutt kollokvium; 2009-04-28 - 2009-04-28

\item Hjorth-Jensen, Morten.  Datamaskiner i realfagsopplæringen, en ny måte å undervise realfag på?. Institutt kollokvium; 2009-02-13 - 2009-02-13

\item Hjorth-Jensen, Morten.  From QCD to the nuclear many-body problem: theory and experiments at Isolde. New Opportunities in the Physics Landscape at CERN Search; 2009-05-10 - 2009-05-13

\item Hjorth-Jensen, Morten.  Lecture 1: Models for the nuclear forces. 20th Chris Engelbrecht Summer School in Theoretical Physics; 2009-01-19 - 2009-01-28

\item Hjorth-Jensen, Morten.  Lecture 1: Nuclear interactions. Postgraduate Nuclear Physics Summer School '09; 2009-09-12 - 2009-09-23

\item Hjorth-Jensen, Morten.  Lecture 1: Nuclear interactions and the Shell Model. 8th CNS-EFES International Summer School; 2009-08-26 - 2009-09-01

\item Hjorth-Jensen, Morten.  Lecture 2: Constructing effective interactions for the shell model. Postgraduate Nuclear Physics Summer School '09; 2009-09-12 - 2009-09-23

\item Hjorth-Jensen, Morten.  Lecture 2: Nuclear interactions and the shell model. 8th CNS-EFES International Summer School; 2009-08-26 - 2009-09-01

\item Hjorth-Jensen, Morten.  Lecture 2: Renormalization of nuclear forces. 20th Chris Engelbrecht Summer School in Theoretical Physics; 2009-01-19 - 2009-01-28

\item Hjorth-Jensen, Morten.  Lecture 3: Effective interactions. 20th Chris Engelbrecht Summer School in Theoretical Physics; 2009-01-19 - 2009-01-28

\item Hjorth-Jensen, Morten.  Lecture 3: Nuclear interactions and the shell model. 8th CNS-EFES International Summer School; 2009-08-26 - 2009-09-01

\item Hjorth-Jensen, Morten.  Lecture 3: Shell model studies. Postgraduate Nuclear Physics Summer School '09; 2009-09-12 - 2009-09-23

\item Hjorth-Jensen, Morten.  Lecture 4: Nuclear interactions and the shell model. 8th CNS-EFES International Summer School; 2009-08-26 - 2009-09-01

\item Hjorth-Jensen, Morten.  Lecture 4: Nuclear many-body methods. 20th Chris Engelbrecht Summer School in Theoretical Physics; 2009-01-19 - 2009-01-28

\item Hjorth-Jensen, Morten.  Lecture 5: Nuclear interactions and the shell model. 8th CNS-EFES International Summer School; 2009-08-26 - 2009-09-01

\item Hjorth-Jensen, Morten.  Lecture 5: Nuclear many-body methods. 20th Chris Engelbrecht Summer School in Theoretical Physics; 2009-01-19 - 2009-01-28

\item Hjorth-Jensen, Morten.  Lecture 6: Nuclear interactions and the shell model. 8th CNS-EFES International Summer School; 2009-08-26 - 2009-09-01

\item Hjorth-Jensen, Morten.  Many-body methods and multiscale physics: A nuclear physics story. Seminar at CTCC, University of oslo; 2009-11-04 - 2009-11-04

\item Hjorth-Jensen, Morten. School on Nuclear Physics at the University of Oslo. 15 lectures in total. Nuclear Physics School; 2009-08-10 - 2009-08-14

\item Hjorth-Jensen, Morten.  Shell structure around 100Sn. Gordon conference:Frontiers Of Nuclear Structure Through Spectroscopy And Reactions; 2009-06-21 - 2009-06-26

\item Hjorth-Jensen, Morten.  Shell-model interactions around 100Sn. American Physical Society April meeting; 2009-05-01 - 2009-05-05

\item Hjorth-Jensen, Morten.  Structure of very neutron-rich nuclei and some key questions in nuclear structure theory. HRIBF, Upgrade for the FRIB Era An HRIBF Users Workshop; 2009-11-13 - 2009-11-14

\item Hjorth-Jensen, Morten. Effective interactions and convergence criteria for configuration interaction methods. Effective Field Theories and the Many-Body Problem; 2009-03-23 - 2009-06-05

\item Hjorth-Jensen, Morten.  CENS, a computational environment for nuclear structure. April Meeting of the American Physical Society; 2008-04-11 - 2008-04-15

\item Hjorth-Jensen, Morten.  Cens lecture 1: Effective interactions for the nuclear shell model. Lecture series at the University of Padova and Legnaro National Laboratory, Padova Italy; 2008-07-15 - 2008-07-18

\item Hjorth-Jensen, Morten.  Cens lecture 2: Nuclear structure studies. Lecture series at the University of Padova and Legnaro national Laboratory, Padova, Italy; 2008-07-15 - 2008-07-18

\item Hjorth-Jensen, Morten.  Cens lecture 3, challenges for nuclear structure studies. Lecture series at the University of Padova and Legnaro national Laboratory, Padova, Italy; 2008-07-15 - 2008-07-18

\item Hjorth-Jensen, Morten.  Computers in Science Education. Guest lecture at Michigan State University; 2008-03-30 - 2008-03-30

\item Hjorth-Jensen, Morten.  Computers in Science Education. Forelesning ved UniK, Kjeller; 2008-10-23 - 2008-10-23

\item Hjorth-Jensen, Morten.  Computers in Science education, a new way to teach science?. eNORIA: Workshop on eScience in Higher Education; 2008-10-07 - 2008-10-07

\item Hjorth-Jensen, Morten.  From nuclear forces to the nuclear many-body problem. Carnegie 2008 Conference NUCLEAR STRUCTURE AT THE EXTREMES; 2008-05-08 - 2008-05-10

\item Hjorth-Jensen, Morten.  From stable to weakly bound nuclei. Lectures series at Lund University; 2008-05-04 - 2008-05-07

\item Hjorth-Jensen, Morten.  From the nucleon-nucleon interaction to effective interactions for the nuclear shell model. Lectures series at Lund University; 2008-05-04 - 2008-05-07

\item Hjorth-Jensen, Morten.  Nuclear many-body methods, shell model and many-body perturbation theory. Lectures series at Lund University; 2008-05-04 - 2008-05-07

\item Hjorth-Jensen, Morten.  Trends in Nuclear Structure Theory. Workshop at the University of Lund; 2008-05-07 - 2008-05-07

\item Hjorth-Jensen, Morten.  Trends in Nuclear Structure Theory. Physics Division Seminar; 2008-04-17 - 2008-04-17

\item Hjorth-Jensen, Morten.  Trends in nuclear structure theory. Lecture series at the University of Padova and Legnaro National Laboratory, Padova Italy; 2008-07-16 - 2008-07-16

\item Hjorth-Jensen, Morten; Langtangen, Hans Petter; Malthe-Sørenssen, Anders; Mørken, Knut Martin; Vistnes, Arnt Inge.  Computers in Science Education, a new way to teach physics and mathematics?. April Meeting of the American Physical Society; 2008-04-11 - 2008-04-15

\item Hjorth-Jensen, Morten; Mørken, Knut Martin.  Computers in Science Education A New Way to Teach Science?. ”I POSE OG SEKK” - Kvalitet i både forskning og utdanning. Er det mulig?; 2008-11-12 - 2008-11-13

\item Hjorth-Jensen, Morten; Mørken, Knut Martin.  Computers in Science Education A New Way to Teach Science?. Møte i Nasjonalt råd for teknologisk utdanning; 2008-11-11 - 2008-11-11

\item Hjorth-Jensen, Morten.  Challenges for nuclear many-body theories. CORRELATIONS IN NUCLEI: BEYOND-MEAN-FIELD AND SHELL-MODEL APPROACHES; 2007-06-04 - 2007-06-08

\item Hjorth-Jensen, Morten.  Computeres in Science Education, a new way to teach science?. Institute seminar; 2007-05-15 - 2007-05-15

\item Hjorth-Jensen, Morten.  Computers in Science Education, a new way to teach science?. EUPEN's 9th General Forum - EGF2007; 2007-09-06 - 2007-09-08

\item Hjorth-Jensen, Morten.  Computers in Science Education: realfagsundervisning på en ny måte?. Pedagogisk modul for MN-fak; 2007-04-11 - 2007-04-11

\item Hjorth-Jensen, Morten.  Coupled Cluster theories: from stable to weakly bound nuclei. CORRELATIONS IN NUCLEI: BEYOND-MEAN-FIELD AND SHELL-MODEL APPROACHES; 2007-06-04 - 2007-06-08

\item Hjorth-Jensen, Morten.  Examples from the physical sciences and sociology. eScience Winther School 2007; 2007-01-28 - 2007-02-02

\item Hjorth-Jensen, Morten.  How to Integrate Parallel Computing in Science Education?. High-Performance and Parallel Computing; 2007-10-24 - 2007-10-24

\item Hjorth-Jensen, Morten.  Introduction to Monte Carlo methods and applications in the physical sciences. eScience Winther School 2007; 2007-01-28 - 2007-02-02

\item Hjorth-Jensen, Morten.  Lecture 1: Models for the nuclear interactions. Lectures in Nuclear Physics, From basic nuclear interactions to nuclear structure; 2007-02-19 - 2007-02-19

\item Hjorth-Jensen, Morten.  Lecture 1: Models for the nuclear interactions. ISOLDE Spring School in Nuclear Theory; 2007-05-21 - 2007-05-26

\item Hjorth-Jensen, Morten.  Lecture 1: Models for the nuclear iteractions. ECT* Doctoral Training Programme 2007; 2007-04-16 - 2007-04-16

\item Hjorth-Jensen, Morten.  Lecture 2: Renormalization of the nucleon-nucleon interaction. Lectures in Nuclear Physics, From basic nuclear interactions to nuclear structure; 2007-02-20 - 2007-02-20

\item Hjorth-Jensen, Morten.  Lecture 2: Renormalization of the nucleon-nucleon interaction. ISOLDE Spring School in Nuclear Theory; 2007-05-21 - 2007-05-26

\item Hjorth-Jensen, Morten.  Lecture 2: Renormalization of the nucleon-nucleon interaction. ECT* Doctoral Training Programme 2007; 2007-04-17 - 2007-04-17

\item Hjorth-Jensen, Morten.  Lecture 3: Many-body methods for nuclear structure. Lectures in Nuclear Physics, From basic nuclear interactions to nuclear structure; 2007-02-21 - 2007-02-21

\item Hjorth-Jensen, Morten.  Lecture 3: Many-body methods for nuclear structure. ISOLDE Spring School in Nuclear Theory; 2007-05-21 - 2007-05-26

\item Hjorth-Jensen, Morten.  Lecture 3: Many-body methods for nuclear structure. ECT* Doctoral Training Programme 2007; 2007-04-18 - 2007-04-18

\item Hjorth-Jensen, Morten.  Lecture 4: Effective interactions for various mass areas. Lectures in Nuclear Physics, From basic nuclear interactions to nuclear structure; 2007-02-22 - 2007-02-22

\item Hjorth-Jensen, Morten.  Lecture 4: Effective interactions for various mass areas. ISOLDE Spring School in Nuclear Theory; 2007-05-21 - 2007-05-26

\item Hjorth-Jensen, Morten.  Lecture 4: Effective interactions for various mass areas. ECT* Doctoral Training Programme 2007; 2007-04-19 - 2007-04-19

\item Hjorth-Jensen, Morten.  Lecture 5: From stable to weakly bound nuclei. Lectures in Nuclear Physics, From basic nuclear interactions to nuclear structure; 2007-02-23 - 2007-02-23

\item Hjorth-Jensen, Morten.  Lecture 5: From stable to weakly bound nuclei. ECT* Doctoral Training Programme 2007; 2007-04-20 - 2007-04-20

\item Hjorth-Jensen, Morten.  Random numbers, Markov chains, Diffusion and the Metropolis algorithm. eScience Winther School 2007; 2007-01-28 - 2007-02-02

\item Hjorth-Jensen, Morten.  Trends in Nuclear Theory. SVENSKT KÄRNFYSIKERMÖTE XXVII, 13-14 NOVEMBER, 2007; 2007-11-13 - 2007-11-14

\item Hjorth-Jensen, Morten.  Two and three-body correlations in nuclei. CORRELATIONS IN NUCLEI: BEYOND-MEAN-FIELD AND SHELL-MODEL APPROACHES; 2007-06-04 - 2007-06-08

\item Hjorth-Jensen, Morten; Dean, David J.; Hagen, Gaute; Papenbrock, Thomas.  Complex Coupled-cluster Approach to an Ab-initio Description of Open Quantum Systems. Recent progress in many-body theories 14; 2007-07-16 - 2007-07-20

\item Hjorth-Jensen, Morten; Jansen, Gustav.  CENS: computational environment for nuclear structure. Many-body physics workshop; 2007-12-05 - 2007-12-07

\item Hjorth-Jensen, Morten; Kvaal, Simen. Similarity Transformations, Flow Equations and Many-Body Perturbation Theory: Role of Many-Body Forces. Many-body physics workshop; 2007-12-05 - 2007-12-07

\item Hjorth-Jensen, Morten; Mørken, Knut Martin.  A unified renewal of mathematics and science education. HPCIA07 (opening of new supercomputer i Tromsø); 2007-12-12 - 2007-12-13

\item Hjorth-Jensen, Morten; Mørken, Knut Martin. Computers in Science Education, realfag på en ny måte?. Realfag – nøkkelen til fremtidens kunnskapssamfunn; 2007-03-23 - 2007-03-23

\item Hjorth-Jensen, Morten; Mørken, Knut Martin.  Computers in Science Education: Realfagsundervisning på en ny måte?. Presentasjon for Abelia og NHO; 2007-08-14 - 2007-08-14

\item Kartamychev, Maxim; Hjorth-Jensen, Morten; Engeland, Torgeir; Osnes, Eivind.  Three-body effective interactions in nuclear structure studies. Many-body methods for 21st century; 2007-10-26 - 2007-10-30

\item Kartamychev, Maxim; Hjorth-Jensen, Morten; Engeland, Torgeir; Osnes, Eivind.  Three-body interactions in nuclear structure studies. Norwegian Physical Society Subatomic and Astrophysics Division Annual Meeting 2007; 2007-01-04 - 2007-01-06

\item Kartamyshev, Maxim; Hjorth-Jensen, Morten; Engeland, Torgeir; Osnes, Eivind.  Realistic three-nucleon effective interactions in nuclear structure studies. RPMBT14; 2007-07-16 - 2007-07-20

\item Kartamyshev, Maxim; Hjorth-Jensen, Morten; Engeland, Torgeir; Osnes, Eivind.  Three-body effective interactions in nuclear structure studies. Workshp at ORNL; 2007-12-05 - 2007-12-07

\item Hjorth-Jensen, Morten.  Basis, model space and wave functions for the shell model. Nuclear shell model applications; 2006-02-13 - 2006-02-17

\item Hjorth-Jensen, Morten.  Effective Interactions for Weakly Bound Systems and Shell Model Studies. 1st Southern Mediterranean Summer Workshop on Subatomic Physics; 2006-05-29 - 2006-06-03

\item Hjorth-Jensen, Morten.  Experimental and theoretical challenges for nuclei in the mass region A=56 to A=78. Nuclear Physics seminar; 2006-09-01 - 2006-09-01

\item Hjorth-Jensen, Morten.  From nucleon-nucleon interactions to effective interactions. Nuclear shell model applications; 2006-02-13 - 2006-02-17

\item Hjorth-Jensen, Morten.  Gamma and Beta decay. Nuclear shell model applications; 2006-02-13 - 2006-02-17

\item Hjorth-Jensen, Morten.  Green's Function Approach to Effective Interactions for Nuclear Systems. 1st Southern Mediterranean Summer Workshop on Subatomic Physics; 2006-05-29 - 2006-06-03

\item Hjorth-Jensen, Morten.  Hva er lys?. Upop aften; 2006-01-16 - 2006-01-16

\item Hjorth-Jensen, Morten.  Methods for studying weakly bound and unbound nuclei. Seminar; 2006-12-01 - 2006-12-01

\item Hjorth-Jensen, Morten.  Nuclear Physics in Norway 2006-2011. OECD Global Science working group on Nuclear Physics; 2006-03-06 - 2006-03-07

\item Hjorth-Jensen, Morten.  Nucleon-Nucleon interactions, from QCD to mesonic degrees of freedom. Nuclear Shell Model applications; 2006-02-13 - 2006-02-17

\item Hjorth-Jensen, Morten.  Spectroscopic factors. Nuclear shell model applications; 2006-02-13 - 2006-02-17

\item Hagen, Gaute; Dean, David J.; Hjorth-Jensen, Morten; Papenbrock, Thomas.  Building nuclei from the ground up. International Symposium on Nuclear Astrophysics - Nuclei in the Cosmos - IX; 2006-06-25 - 2006-06-30

\item Hagen, Gaute; Dean, David J.; Hjorth-Jensen, Morten; Papenbrock, Thomas.  Coupled-cluster calculation of the 3-5He isotopes with Gamow-Hartree-Fock basis. Nuclei in the Cosmos 9; 2006-06-25 - 2006-06-30

\item Kartamychev, Maxim; Hjorth-Jensen, Morten; Engeland, Torgeir; Osnes, Eivind.  Realistic Three-Nucleon Effective Interaction from the Folded-Diagram Theory. Nuclei in the Cosmos - IX; 2006-06-25 - 2006-06-30

\item Kartamychev, Maxim; Hjorth-Jensen, Morten; Engeland, Torgeir; Osnes, Eivind. Realistic Three-Nucleon Effective Interaction from the Folded-Diagram Theory. DNP 06; 2006-10-25 - 2006-10-28

\item Hjorth-Jensen, Morten.  Ab Initio nuclear structure methods: Monte Carlo methods and no-core shell model approaches. ISOLDE Physics Group Seminar; 2005-03-14 - 2005-03-14

\item Hjorth-Jensen, Morten. CHALLENGES FOR NUCLEAR STRUCTURE: FROM STABLE TO WEAKLY BOUND NUCLEI. International Symposium on Correlation Dynamics in Nuclei; 2005-01-31 - 2005-02-05

\item Hjorth-Jensen, Morten.  Computational Environment for Nuclear Structure: CENS. Lecture Series at Michigan State University; 2005-04-11 - 2005-04-12

\item Hjorth-Jensen, Morten.  Computers in Science Education. CMA workshop on 'Computers, computations and science education'; 2005-09-30 - 2005-09-30

\item Hjorth-Jensen, Morten.  From the nucleon-nucleon interaction to a renormalized interaction for nuclear systems. Lecture series at Michigan State University; 2005-04-07 - 2005-04-08

\item Hjorth-Jensen, Morten.  High-Performance Computing in Physics. High-Performance Computing in Physics workshop; 2005-11-04 - 2005-11-04

\item Hjorth-Jensen, Morten.  Kvalitetsreformen, nye Muligheter for Samarbeid mellom Universitet og Næringsliv. Industridag, rom for muligheter; 2005-09-16 - 2005-09-16

\item Hjorth-Jensen, Morten.  Large Scale Shell Model and Coupled Cluster Calculations. Microscopic Approaches to Many-Body Theories; 2005-08-30 - 2005-09-03

\item Hjorth-Jensen, Morten.  Shell model approaches. 2nd VISTARS Workshop in Russbach; 2005-03-05 - 2005-03-12

\item Hjorth-Jensen, Morten.  Variational and Diffusion Monte Carlo Calculations for Bose-Einstein condensation. Nonlinear PDE for Bose-Einstein condensed gases; 2005-11-11 - 2005-11-11

\item Honma, Micho; Otsuka, Takahuro; Mizusaki, T.; Hjorth-Jensen, Morten; Brown, Boyd Alexander. Effective Interactions for nuclei with A=50-100 and Gamow-Teller properties. International Symposium on Correlation Dynamics in Nuclei; 2005-01-31 - 2005-02-04

\item Dean, David J.; Hjorth-Jensen, Morten; Kowalski, Karol; Piecuch, Piotr; Wloch, Marta. Coupled Cluster Theory for Nuclei. International Workshop on Condensed Matter Theories CMT28; 2004-09-27 - 2004-10-02

\item Hjorth-Jensen, Morten. CENS: A computational Environment for Nuclear Structure. Isolde Lecture series; 2004-11-11 - 2005-11-25

\item Hjorth-Jensen, Morten.  Challenges for Nuclear Structure; from Stable to Weakly Bound Nuclei. Theory seminar University of Tuebingen; 2004-12-07 - 2004-12-07

\item Hjorth-Jensen, Morten.  Challenges for Nuclear Structure Studies. Isolde workshop 2004; 2004-12-13 - 2004-12-15

\item Hjorth-Jensen, Morten.  Coupled Cluster approaches to nuclei, ground state and excited states. 8th INTERNATIONAL SPRING SEMINAR ON NUCLEAR PHYSICS; 2004-05-23 - 2004-05-27

\item Hjorth-Jensen, Morten.  Effective Interactions for the Nuclear many-body problem. Workshop on Nuclear structure Studies for Light Nuclei; 2004-07-04 - 2004-07-08

\item Hjorth-Jensen, Morten. Fra Supernovaer og nøytronstjerner til nøytronrike kjerner; en reise fra giga/megameter til femtometer skala. Foredrag ved Norsk Astronomisk selskap; 2004-01-14 - 2004-01-14

\item Hjorth-Jensen, Morten. From non-linear PDEs to Monte-Carlo methods, a biased tour of open problems in computational quantum mechanics. CMA workshop on Mathematical Aspects of the Schroedinger Equation; 2004-06-14 - 2004-06-14

\item Hjorth-Jensen, Morten. Mathematics for Neutron Stars. Foredrag ved CMA; 2004-05-11 - 2004-05-11

\item Hjorth-Jensen, Morten. Nuclear Many-Body Approaches and Experiment; workshop summary. Insitute of Nuclear Theory workshop series; 2004-10-04 - 2004-10-08

\item Hjorth-Jensen, Morten.  Nuclear structure and the coupled-cluster method. International Nuclear Physics Conference, INPC2004; 2004-06-27 - 2004-07-02

\item Hjorth-Jensen, Morten.  Nuclear Structure for Radioactive Ion Beam Physic. ISOLDE PHYSICS GROUP SEMINAR SERIES; 2004-09-21 - 2004-09-21

\item Hjorth-Jensen, Morten.  Selected Nuclear Structure Topics. Workshop on Nuclear structure Studies for Light Nuclei; 2004-07-04 - 2004-07-08

\item Hjorth-Jensen, Morten.  Shell-Model Approaches and Effective Interactions for Weakly Bound Systems. Insitute Seminar Max-Planck Institut fuer Kern Chemie; 2004-12-06 - 2004-12-06

\item Hjorth-Jensen, Morten. Økt innsikt og læring ved hjelp av IKT i Fysikk. Det Umuliges kunst? IKT i utdanning - kvalitetetsreformen i praksis; 2004-04-28 - 2004-04-28

\item Vistnes, Arnt Inge; Hjorth-Jensen, Morten. Numerical methods as an integrated part of physics education. 9th Workshop on Multimedia in Physics Teaching and Learning; 2004-09-09 - 2004-09-11

\item Ovrum, Eirik; Leinaas, Jon Magne; Hjorth-Jensen, Morten. Quantum Computation of Energy Levels in a Spin Chain: A Detailed Simulation for a Small no of Spins. Gordon Research Conference; 2004-02-22 - 2004-02-28

\item Hjorth-Jensen, Morten. Bruk av numeriske verktøy i undervisningen. Pedagogisk modul i 'Undervisning i matematiske og naturvitenskapelige fag'; 2003-05-23 - 2003-05-23

\item Hjorth-Jensen, Morten. Challenges for shell-model studies and emergent phenomena in nuclei. APS april meeting; 2003-04-04 - 2003-04-07

\item Hjorth-Jensen, Morten.  Computational quantum mechanics. CMA seminar; 2003-05-06 - 2003-05-06

\item Hjorth-Jensen, Morten. Effective interactions for weakly bound systems. DNP fall meeting; 2003-10-29 - 2003-11-01

\item Hjorth-Jensen, Morten. Effective interactions for weakly bound systems. Mini/workshop on nuclear many/body physics; 2003-04-02 - 2003-04-02

\item Hjorth-Jensen, Morten. Effective interactions from Greens functions. Recent advances in the nuclear shell model; 2003-06-29 - 2003-07-12

\item Hjorth-Jensen, Morten. Many-body methods and the nuclear shell-model. 10th Nordic Nuclear Physics Meeting; 2003-05-12 - 2003-05-16

\item Hjorth-Jensen, Morten. Neutron stars and challenges for RIA physics. RIA theory working group workshop; 2003-11-02 - 2003-11-03

\item Hjorth-Jensen, Morten. Pairing correlations in nuclear systems. COMEX1; 2003-06-10 - 2003-06-13

\item Hjorth-Jensen, Morten. Pairing correlations in nuclear systems. Foredrag ved Oak Ridge National lab; 2003-08-12 - 2003-08-12

\item Hjorth-Jensen, Morten. Complex scaling and effective interactions for weakly bound nuclei. ; 2002

\item Hjorth-Jensen, Morten. Effective interactions and the nuclear shell model. Continuum aspects of the nuclear shell model; 2002-06-03

\item Hjorth-Jensen, Morten. Effective interactions for the nuclear shell model. Advanced computational methods for solving the nuclear many-body problem; 2002-03-12

\item Hjorth-Jensen, Morten. Effective interactions of the nuclear shell model. ; 2002

\item Hjorth-Jensen, Morten. Pairing correlations in nuclear systems. ; 2002

\item Hjorth-Jensen, Morten. Pairing correlations in nuclear systems, from neutron stars to finite nuclei. ; 2002

\item Hjorth-Jensen, Morten. Theory of effective interactions. ; 2002

\item Hjorth-Jensen, Morten. Brukerinformasjon om tungregneberegninger. Møte mellom Usit of Hewlett Packard; 2001-02-14

\item Hjorth-Jensen, Morten. Effective interactions for finite nuclei. Nato advanced workshop on the nuclear many-body problem; 2001-06-02

\item Hjorth-Jensen, Morten. Effective Interactions for the nuclear shell model. ISOL01; 2001-03-11

\item Hjorth-Jensen, Morten. Effective interactions for the nuclear shell-model. International workshop on continuum aspects of the nuclear shell model; 2001-09-24

\item Hjorth-Jensen, Morten. From finite nuclei to neutron stars and dense matter. Annual Meeting of the Norwegian physics society; 2001-06-14

\item Hjorth-Jensen, Morten. Kvantedatamaskinen, den neste teknologiske revolusjonen?. Faglig pedagogisk dag universitetet i oslo; 2001-01-03

\item Hjorth-Jensen, Morten. Nye trender i kvantefysikk. Fysikk kurs for gymnaslærere; 2001-11-27

\item Hjorth-Jensen, Morten. Pairing correlations in nuclear systems. ; 2001

\item Hjorth-Jensen, Morten. Pairing correlations in nuclear systems, from neutrons starts to finite nuclei. Yukawa International seminar 2001, Physics of unstable nuclei; 2001-11-05

\item Hjorth-Jensen, Morten. Phases of dense matter in neutron stars. Graduate programme in nuclear physics, Copenhagen-Giessen; 2001-01-25

\item Hjorth-Jensen, Morten, Effective interactions for medium heavy nuclei. 5th international conference on radioactive nuclear beams; 2000-04-03

\item Hjorth-Jensen, Morten. Effective interactions for finite nuclei. Physics with Radioactive Beams; 2000-11-27

\item Hjorth-Jensen, Morten. Effective interactions for nuclear systems. Nuclear structure for the 21st century; 2000-10-15

\item Hjorth-Jensen, Morten. Kvantedatamaskinen, den neste teknologiske revolusjonen?. IAESTE næringslivsdager; 2000-09-13

\item Hjorth-Jensen, Morten. Nuclear structure from finite nuclei to neutron stars. Twelfth summer school in nuclear physics; 2000-07-03

\item Hjorth-Jensen, Morten. Phases of dense matter in neutron stars. EOS2000; 2000-02-20

\item Dean, David J.; Hjorth-Jensen, Morten; Liotta, Roberto; Zuker, A.P.. Advances in shell model studies in nuclei far from stability. Advances in shell model studies in nuclei far from stability; 1999-01-01

\item Hjorth-Jensen, Morten. Effective interactions for finite nuclei. Advances in nuclear many-body theory; 1999-08-01

\item Hjorth-Jensen, Morten.  Faseoverganger i endelige systemer?. ; 1999

\item Hjorth-Jensen, Morten. From finite nuclei to neutron stars. NFR meeting on Cern related Physics; 1999-10-01

\item Hjorth-Jensen, Morten. Pairing correlations, from finite nuclei to infinite matter. Recent progress in Many-Body theories 10; 1999-09-10

\item Hjorth-Jensen, Morten.  Phases of dense matter in neutron stars. ; 1999

\item Hjorth-Jensen, Morten. Properties of Pairing Correlations in Infinite Nuclear Matter. Condensed Matter theories 21; 1998-01-01

\item Hjorth-Jensen, Morten.  Realistic Effective Interactions and Large-Scale Nuclear Structure Calculation. Highlights of modern nuclear structure; 1998-05-01

\item Hjorth-Jensen, Morten. Nuclear structure from $N\approx Z$ to $N >>Z$. Highlights of modern nuclear structure; 1998-05-01

\item Engeland, Torgeir; Hjorth-Jensen, Morten; Holt, Anne; Osnes, Eivind. Extensive Shell-Model calculations in the tin isotopes. workshop on double-beta decay; 1996-01-01

\item Engeland, Torgeir; Hjorth-Jensen, Morten; Holt, Anne; Osnes, Eivind. Realistic Large basis shell-model calculation in the low-mass tin isotopes. symposium on frontiers of nuclear structure physics; 1996-01-01

\item Hjorth-Jensen, Morten.  Conference: International Conference on Realistic Nuclear Structure, to Celebrate the 60th Birthday of TTS (TOM) Kuo Location: SUNY Stony Brook, phys dept, STONY BROOK, NY, May 28-30, 1992

\item Hjorth-Jensen, Morten,  Microscopic nuclear-structure calculations with modern meson-exchange potentials, 3rd international spring seminar on nuclear physics, Ischia, Italy, May 21-25, 1990
\end{enumerate}

\noindent
\paragraph{Selected research grants as PI and co-PI, financed and pending applications.}
\begin{enumerate}
\item 2020-205, Quantum Strategic Innovation and Development (QSID) Center, pending, 125MUSD, Michigan State University, co-PI, Department of Energy, USA

\item 2020-2025, QLCI – CI: Institute for Quantum Computing and Control (IQC2) at MSU, pending, 25MUSD, Michigan State University, co-PI, National Science Foundation, USA

\item 2020-2025, AI Institute:  Transdisciplinary Institute for Physics-Informed Machine Learning, pending, 25MUSD, Michigan State University, co-PI, National Science Foundation, USA

\item 2020-2022 750 kUSD from the Department of Energy, From Quarks to Stars; A Quantum Computing Approach to the Nuclear Many-Body Problem. PI, grant number DoE-0000248785  

\item 2020-2023 600 kUSD from the National Science Foundation for the project From nuclei to neutron stars, CO-PI with Scott Bogner, Grant number PHY-013877. 

\item 2016-present Co-PI  at the Norwegian center of excellence in Education \emph{Center for Computing in Science Education}, University of Oslo wit annual funding from NOKUT of 5MNOK

\item 2017-2020 600 kUSD from the National Science Foundation for the project From nuclei to neutron stars, CO-PI with Scott Bogner. Grant number PHY-1713901

\item 2014-2017 600 kUSD from the National Science Foundation for the project Computational Nuclear Many-body Physics, CO-PI with Scott Bogner. Grant number PHY-1404159

\item 2010-2015 15 MNOK from the Research Council of Norway, Multi-scale physics on the computer, Collaboration between Universities. Grant number ISP-Fysikk/216699 in Trondheim, Ås, and Oslo on research and education in computational physics

\item 2008-2012 1.5 MNOK for organizing the MSU-UT/ORNL-UiO network on Computational Nuclear Many-Body Theory. Sponsor, SiU, the Norwegian center for internationalization in higher education

\item 2003-2013 Co-PI of the Norwegian Center of Excellence Mathematics for Applications, with annual funding from the Research Council of Norway of 15 MNOK.
\end{enumerate}

\noindent

% ------------------- end of main content ---------------

\end{document}

